\documentclass{beamer}

\usetheme{default}
\usecolortheme{default}
\usefonttheme[onlymath]{serif}

%%%%% PACKAGES

% small tweaks and nicer typography
\usepackage{microtype}

% changes language to German
% gives proper date, and correct hyphenation
\usepackage[ngerman]{babel}
\uselanguage{German}
\languagepath{German}

% display quotes correctly
\usepackage{csquotes}

% basic math stuff
\usepackage{mathtools}
\usepackage{amssymb}
\usepackage{amsthm}
\usepackage{tikz}

% allow for any font-size, alternative mathpazo
\usepackage{mathptmx}

% images
\usepackage{graphicx}
\graphicspath{ {./images/} }
\usepackage{subcaption}


% tikz
\usepackage{tikz}
\usetikzlibrary{positioning}
\usetikzlibrary{babel}
\tikzset{>=stealth}

\newcommand{\tikzmark}[3][]{\tikz[remember picture,baseline] \node [anchor=base,#1](#2) {$#3$};}

% Tikz librarys
\usetikzlibrary{datavisualization}
\usetikzlibrary{datavisualization.formats.functions}
%\usetikzlibrary{external}
%\tikzexternalize[prefix=../Resources/]


%%%%% CONFIGURATION

% prevents automatic line breaks inside of equations
% since it looks bad
\binoppenalty = \maxdimen
\relpenalty   = \maxdimen

% theorem-like environments
\newcounter{everything}
%\newtheorem{corollary}[everything]{Korollar}
%\newtheorem{lemma}[everything]{Lemma}
\newtheorem{proposition}[everything]{Proposition}



%%%%% CUSTOM COMMANDS


% real numbers via \R
% complex numbers via \C
% general field via \K
\def\C{\mathbb{C}}
\def\R{\mathbb{R}}
\def\K{\mathbb{K}}
\def\Q{\mathbb{Q}}
\def\Z{\mathbb{Z}}
\def\N{\mathbb{N}}
\def\H{\mathbb{H}}
\def\e{\varepsilon}
\def\ev{e}


\newcommand{\cA}{\mathcal{A}}
\newcommand{\cB}{\mathcal{B}}
\newcommand{\cC}{\mathcal{C}}
\newcommand{\cD}{\mathcal{D}}
\newcommand{\cE}{\mathcal{E}}
\newcommand{\cF}{\mathcal{F}}
\newcommand{\cG}{\mathcal{G}}
\newcommand{\cH}{\mathcal{H}}
\newcommand{\cI}{\mathcal{I}}
\newcommand{\cJ}{\mathcal{J}}
\newcommand{\cK}{\mathcal{K}}
\newcommand{\cL}{\mathcal{L}}
\newcommand{\cM}{\mathcal{M}}
\newcommand{\cN}{\mathcal{N}}
\newcommand{\cO}{\mathcal{O}}
\newcommand{\cP}{\mathcal{P}}
\newcommand{\cQ}{\mathcal{Q}}
\newcommand{\cR}{\mathcal{R}}
\newcommand{\cS}{\mathcal{S}}
\newcommand{\cT}{\mathcal{T}}
\newcommand{\cU}{\mathcal{U}}
\newcommand{\cV}{\mathcal{V}}
\newcommand{\cW}{\mathcal{W}}
\newcommand{\cX}{\mathcal{X}}
\newcommand{\cY}{\mathcal{Y}}
\newcommand{\cZ}{\mathcal{Z}}

\newcommand{\bA}{\mathbb{A}}
\newcommand{\bB}{\mathbb{B}}
\newcommand{\bC}{\mathbb{C}}
\newcommand{\bD}{\mathbb{D}}
\newcommand{\bE}{\mathbb{E}}
\newcommand{\bF}{\mathbb{F}}
\newcommand{\bG}{\mathbb{G}}
\newcommand{\bH}{\mathbb{H}}
\newcommand{\bI}{\mathbb{I}}
\newcommand{\bJ}{\mathbb{J}}
\newcommand{\bK}{\mathbb{K}}
\newcommand{\bL}{\mathbb{L}}
\newcommand{\bM}{\mathbb{M}}
\newcommand{\bN}{\mathbb{N}}
\newcommand{\bO}{\mathbb{O}}
\newcommand{\bP}{\mathbb{P}}
\newcommand{\bQ}{\mathbb{Q}}
\newcommand{\bR}{\mathbb{R}}
\newcommand{\bS}{\mathbb{S}}
\newcommand{\bT}{\mathbb{T}}
\newcommand{\bU}{\mathbb{U}}
\newcommand{\bV}{\mathbb{V}}
\newcommand{\bW}{\mathbb{W}}
\newcommand{\bX}{\mathbb{X}}
\newcommand{\bY}{\mathbb{Y}}
\newcommand{\bZ}{\mathbb{Z}}

\newcommand{\hu}{\hat{u}}
\newcommand{\hv}{\hat{v}}
\newcommand{\hA}{\hat{A}}
\newcommand{\hC}{\hat{C}}
\newcommand{\hR}{\hat{R}}
\newcommand{\hb}{\hat{b}}


\newcommand{\tiS}{\tilde{S}}
\newcommand{\tiu}{\tilde{u}}
\newcommand{\tih}{\tilde{h}}
\newcommand{\tie}{\tilde{\e}}
\newcommand{\tisigma}{\tilde{\sigma}}


%%%%%%%%%%%%%%%%%%% Math operators %%%%%%%%%%%%%


\newcommand{\dif}[1]{\,\mathrm{d} #1}
\newcommand{\norm}[1]{\lVert #1 \rVert}
\newcommand{\bnorm}[1]{\left\lVert #1\right\rVert}
\newcommand{\vii}[2]{\ensuremath{\begin{pmatrix}#1 \\ #2 \end{pmatrix}}}
\newcommand{\mii}[4]{\ensuremath{\begin{pmatrix}#1&#2 \\ #3&#4 \end{pmatrix}}}
\newcommand{\mc}[1]{\mathcal{#1}}


% Hom(V,W) via \Hom(V,W)
\DeclareMathOperator{\End}{End}
\DeclareMathOperator{\Hom}{Hom}
\DeclareMathOperator{\Id}{Id}
\DeclareMathOperator{\diver}{Div}
\DeclareMathOperator{\Tr}{Tr}
\DeclareMathOperator{\Image}{Image}
\DeclareMathOperator{\Span}{Span}         % linear span
\DeclareMathOperator{\Vspan}{Span}
\DeclareMathOperator{\Erf}{erf}

% inner product (scalar product) via \inner{v, w}
% norm via \norm{x}
% absolute value via \abs{x}
% use the star-version for automatic scaling
\DeclarePairedDelimiter{\abs}{|}{|}
\DeclarePairedDelimiter{\inner}{\langle}{\rangle}

% \vect{ x // y // z } for a column vector with entries x, y, z
% similarly for larger vectors
% in this code:  1 = number of arguments
%               #1 = first argument
\newcommand{\vect}[1]{\begin{pmatrix} #1 \end{pmatrix}}
\newcommand{\conj}{\overline}


% \conj{z} for complex conjugation
% \newcommand{\conj}{\overline}

%%%%% allow for proofs over multiple slides

\makeatletter
\newenvironment<>{proofs}[1][\proofname]{%
    \par
    \def\insertproofname{#1\@addpunct{.}}%
    \usebeamertemplate{proof begin}#2}
  {\usebeamertemplate{proof end}}
\makeatother


%%%%% TITLE PAGE

\subject{Adaptive Finite Elemente für Lineare Elastizität}
\title{Adaptive Finite Elemente für Lineare Elastizität}
%\subtitle{Blatt 0}
\author{Theo Koppenhöfer}
\date{7. April 2022}




\begin{document}

\frame[plain]



\section{Einleitung}

% Frame 1
\frame[plain]{\titlepage}

\begin{frame}
	\begin{center}
		\Large{{Es wird spannend...}}
	\end{center}
\end{frame}

% Frame 3
\frame{ \frametitle{Outline} \tableofcontents }

\subsection*{Verzerrung, Spannung}
\subsubsection*{Geometrische Betrachtungen}
\begin{frame}[allowframebreaks]
	Wir nehmen an, der Körper nimmt in Referenzkonfiguration (Lagrange-Koordinaten) das Gebiet $\overline{\Omega}\subseteq\R^d$ ein. Wir definieren
	\begin{itemize}
		\item Deformation: Eine Abbildung $\chi\colon\overline{\Omega}\to\R^d$ mit $\det\chi>0$
		\item Verschiebung: Eine Abbildung $u$, gegeben durch $\chi=\Id+u$
	\end{itemize}
	Die Menge an zulässigen Verschiebungen bezeichnen wir mit $V$ (wir verwenden für eine generische Verschiebung den Buchstaben $v$).
	
	\framebreak
	\begin{figure}[h]
	\centering
	% Graphic for TeX using PGF
% Title: /mnt/12CCB7B3CCB79009/Filing/Education/University/Bonn/Courses/Bachelorarbeit/Resources/Deformation1.dia
% Creator: Dia v0.97+git
% CreationDate: Mon Apr  4 15:27:10 2022
% For: theo
% \usepackage{tikz}
% The following commands are not supported in PSTricks at present
% We define them conditionally, so when they are implemented,
% this pgf file will use them.
\ifx\du\undefined
  \newlength{\du}
\fi
\setlength{\du}{15\unitlength}
\begin{tikzpicture}[even odd rule]
\pgftransformxscale{0.5}
\pgftransformyscale{-0.5}
\definecolor{dialinecolor}{rgb}{0.000000, 0.000000, 0.000000}
\pgfsetstrokecolor{dialinecolor}
\pgfsetstrokeopacity{1.000000}
\definecolor{diafillcolor}{rgb}{1.000000, 1.000000, 1.000000}
\pgfsetfillcolor{diafillcolor}
\pgfsetfillopacity{1.000000}
\pgfsetlinewidth{0.100000\du}
\pgfsetdash{}{0pt}
\pgfsetbuttcap
{
\definecolor{diafillcolor}{rgb}{0.000000, 0.000000, 0.000000}
\pgfsetfillcolor{diafillcolor}
\pgfsetfillopacity{1.000000}
% was here!!!
\pgfsetarrowsend{latex}
\definecolor{dialinecolor}{rgb}{0.000000, 0.000000, 0.000000}
\pgfsetstrokecolor{dialinecolor}
\pgfsetstrokeopacity{1.000000}
\draw (0.000000\du,0.000000\du)--(5.000000\du,0.000000\du);
}
\pgfsetlinewidth{0.100000\du}
\pgfsetdash{}{0pt}
\pgfsetbuttcap
{
\definecolor{diafillcolor}{rgb}{0.000000, 0.000000, 0.000000}
\pgfsetfillcolor{diafillcolor}
\pgfsetfillopacity{1.000000}
% was here!!!
\pgfsetarrowsend{latex}
\definecolor{dialinecolor}{rgb}{0.000000, 0.000000, 0.000000}
\pgfsetstrokecolor{dialinecolor}
\pgfsetstrokeopacity{1.000000}
\draw (0.000000\du,0.000000\du)--(0.000000\du,-5.000000\du);
}
\pgfsetlinewidth{0.100000\du}
\pgfsetdash{}{0pt}
\pgfsetbuttcap
{
\definecolor{diafillcolor}{rgb}{0.000000, 0.000000, 0.000000}
\pgfsetfillcolor{diafillcolor}
\pgfsetfillopacity{1.000000}
% was here!!!
\definecolor{dialinecolor}{rgb}{0.000000, 0.000000, 0.000000}
\pgfsetstrokecolor{dialinecolor}
\pgfsetstrokeopacity{1.000000}
\draw (7.000000\du,-5.000000\du)--(18.000000\du,-5.000000\du);
}
\pgfsetlinewidth{0.100000\du}
\pgfsetdash{}{0pt}
\pgfsetbuttcap
{
\definecolor{diafillcolor}{rgb}{0.000000, 0.000000, 0.000000}
\pgfsetfillcolor{diafillcolor}
\pgfsetfillopacity{1.000000}
% was here!!!
\definecolor{dialinecolor}{rgb}{0.000000, 0.000000, 0.000000}
\pgfsetstrokecolor{dialinecolor}
\pgfsetstrokeopacity{1.000000}
\draw (18.000000\du,-5.000000\du)--(18.000000\du,-13.000000\du);
}
\pgfsetlinewidth{0.100000\du}
\pgfsetdash{}{0pt}
\pgfsetbuttcap
{
\definecolor{diafillcolor}{rgb}{0.000000, 0.000000, 0.000000}
\pgfsetfillcolor{diafillcolor}
\pgfsetfillopacity{1.000000}
% was here!!!
\definecolor{dialinecolor}{rgb}{0.000000, 0.000000, 0.000000}
\pgfsetstrokecolor{dialinecolor}
\pgfsetstrokeopacity{1.000000}
\draw (18.000000\du,-13.000000\du)--(10.000000\du,-13.000000\du);
}
\pgfsetlinewidth{0.100000\du}
\pgfsetdash{}{0pt}
\pgfsetbuttcap
{
\definecolor{diafillcolor}{rgb}{0.000000, 0.000000, 0.000000}
\pgfsetfillcolor{diafillcolor}
\pgfsetfillopacity{1.000000}
% was here!!!
\definecolor{dialinecolor}{rgb}{0.000000, 0.000000, 0.000000}
\pgfsetstrokecolor{dialinecolor}
\pgfsetstrokeopacity{1.000000}
\draw (10.000000\du,-13.000000\du)--(7.000000\du,-5.000000\du);
}
\pgfsetlinewidth{0.100000\du}
\pgfsetdash{}{0pt}
\pgfsetbuttcap
{
\definecolor{diafillcolor}{rgb}{0.000000, 0.000000, 0.000000}
\pgfsetfillcolor{diafillcolor}
\pgfsetfillopacity{1.000000}
% was here!!!
\definecolor{dialinecolor}{rgb}{0.000000, 0.000000, 0.000000}
\pgfsetstrokecolor{dialinecolor}
\pgfsetstrokeopacity{1.000000}
\draw (6.000000\du,-14.000000\du)--(8.000000\du,-19.000000\du);
}
\pgfsetlinewidth{0.100000\du}
\pgfsetdash{}{0pt}
\pgfsetbuttcap
{
\definecolor{diafillcolor}{rgb}{0.000000, 0.000000, 0.000000}
\pgfsetfillcolor{diafillcolor}
\pgfsetfillopacity{1.000000}
% was here!!!
\definecolor{dialinecolor}{rgb}{0.000000, 0.000000, 0.000000}
\pgfsetstrokecolor{dialinecolor}
\pgfsetstrokeopacity{1.000000}
\draw (-6.000000\du,-17.000000\du)--(3.000000\du,-21.000000\du);
}
\pgfsetlinewidth{0.100000\du}
\pgfsetdash{}{0pt}
\pgfsetbuttcap
\definecolor{dialinecolor}{rgb}{0.000000, 0.000000, 0.000000}
\pgfsetstrokecolor{dialinecolor}
\pgfsetstrokeopacity{1.000000}
\pgfpathmoveto{\pgfpoint{6.000114\du}{-13.999887\du}}
\pgfpatharc{315}{254}{12.113788\du and 12.113788\du}
\pgfusepath{stroke}
\pgfsetlinewidth{0.100000\du}
\pgfsetdash{}{0pt}
\pgfsetbuttcap
\definecolor{dialinecolor}{rgb}{0.000000, 0.000000, 0.000000}
\pgfsetstrokecolor{dialinecolor}
\pgfsetstrokeopacity{1.000000}
\pgfpathmoveto{\pgfpoint{8.000010\du}{-18.999976\du}}
\pgfpatharc{337}{247}{3.810623\du and 3.810623\du}
\pgfusepath{stroke}
\pgfsetlinewidth{0.050000\du}
\pgfsetdash{}{0pt}
\pgfsetbuttcap
\definecolor{dialinecolor}{rgb}{0.000000, 0.000000, 0.000000}
\pgfsetstrokecolor{dialinecolor}
\pgfsetstrokeopacity{1.000000}
\pgfpathmoveto{\pgfpoint{6.790392\du}{-15.949101\du}}
\pgfpatharc{323}{253}{8.050467\du and 8.050467\du}
\pgfusepath{stroke}
\pgfsetlinewidth{0.050000\du}
\pgfsetdash{}{0pt}
\pgfsetbuttcap
{
\definecolor{diafillcolor}{rgb}{0.000000, 0.000000, 0.000000}
\pgfsetfillcolor{diafillcolor}
\pgfsetfillopacity{1.000000}
% was here!!!
\definecolor{dialinecolor}{rgb}{0.000000, 0.000000, 0.000000}
\pgfsetstrokecolor{dialinecolor}
\pgfsetstrokeopacity{1.000000}
\draw (-0.832811\du,-17.414717\du)--(4.380734\du,-21.303853\du);
}
\pgfsetlinewidth{0.050000\du}
\pgfsetdash{}{0pt}
\pgfsetbuttcap
{
\definecolor{diafillcolor}{rgb}{0.000000, 0.000000, 0.000000}
\pgfsetfillcolor{diafillcolor}
\pgfsetfillopacity{1.000000}
% was here!!!
\definecolor{dialinecolor}{rgb}{0.000000, 0.000000, 0.000000}
\pgfsetstrokecolor{dialinecolor}
\pgfsetstrokeopacity{1.000000}
\draw (3.150000\du,-16.050000\du)--(6.798305\du,-20.462959\du);
}
\pgfsetlinewidth{0.050000\du}
\pgfsetdash{}{0pt}
\pgfsetbuttcap
\definecolor{dialinecolor}{rgb}{0.000000, 0.000000, 0.000000}
\pgfsetstrokecolor{dialinecolor}
\pgfsetstrokeopacity{1.000000}
\pgfpathmoveto{\pgfpoint{7.580030\du}{-17.873140\du}}
\pgfpatharc{324}{245}{6.035271\du and 6.035271\du}
\pgfusepath{stroke}
\pgfsetlinewidth{0.050000\du}
\pgfsetdash{}{0pt}
\pgfsetbuttcap
{
\definecolor{diafillcolor}{rgb}{0.000000, 0.000000, 0.000000}
\pgfsetfillcolor{diafillcolor}
\pgfsetfillopacity{1.000000}
% was here!!!
\definecolor{dialinecolor}{rgb}{0.000000, 0.000000, 0.000000}
\pgfsetstrokecolor{dialinecolor}
\pgfsetstrokeopacity{1.000000}
\draw (11.000000\du,-13.000000\du)--(11.000000\du,-5.000000\du);
}
\pgfsetlinewidth{0.050000\du}
\pgfsetdash{}{0pt}
\pgfsetbuttcap
{
\definecolor{diafillcolor}{rgb}{0.000000, 0.000000, 0.000000}
\pgfsetfillcolor{diafillcolor}
\pgfsetfillopacity{1.000000}
% was here!!!
\definecolor{dialinecolor}{rgb}{0.000000, 0.000000, 0.000000}
\pgfsetstrokecolor{dialinecolor}
\pgfsetstrokeopacity{1.000000}
\draw (15.000000\du,-13.000000\du)--(15.000000\du,-5.000000\du);
}
\pgfsetlinewidth{0.050000\du}
\pgfsetdash{}{0pt}
\pgfsetbuttcap
{
\definecolor{diafillcolor}{rgb}{0.000000, 0.000000, 0.000000}
\pgfsetfillcolor{diafillcolor}
\pgfsetfillopacity{1.000000}
% was here!!!
\definecolor{dialinecolor}{rgb}{0.000000, 0.000000, 0.000000}
\pgfsetstrokecolor{dialinecolor}
\pgfsetstrokeopacity{1.000000}
\draw (8.139259\du,-7.981492\du)--(18.000000\du,-8.000000\du);
}
\pgfsetlinewidth{0.050000\du}
\pgfsetdash{}{0pt}
\pgfsetbuttcap
{
\definecolor{diafillcolor}{rgb}{0.000000, 0.000000, 0.000000}
\pgfsetfillcolor{diafillcolor}
\pgfsetfillopacity{1.000000}
% was here!!!
\definecolor{dialinecolor}{rgb}{0.000000, 0.000000, 0.000000}
\pgfsetstrokecolor{dialinecolor}
\pgfsetstrokeopacity{1.000000}
\draw (8.952426\du,-10.297251\du)--(17.946967\du,-10.310575\du);
}
\pgfsetlinewidth{0.100000\du}
\pgfsetdash{}{0pt}
\pgfsetbuttcap
{
\definecolor{diafillcolor}{rgb}{0.000000, 0.000000, 0.000000}
\pgfsetfillcolor{diafillcolor}
\pgfsetfillopacity{1.000000}
% was here!!!
\pgfsetarrowsend{to}
\definecolor{dialinecolor}{rgb}{0.000000, 0.000000, 0.000000}
\pgfsetstrokecolor{dialinecolor}
\pgfsetstrokeopacity{1.000000}
\draw (0.000000\du,0.000000\du)--(7.000000\du,-5.000000\du);
}
\pgfsetlinewidth{0.100000\du}
\pgfsetdash{}{0pt}
\pgfsetbuttcap
{
\definecolor{diafillcolor}{rgb}{0.000000, 0.000000, 0.000000}
\pgfsetfillcolor{diafillcolor}
\pgfsetfillopacity{1.000000}
% was here!!!
\pgfsetarrowsend{to}
\definecolor{dialinecolor}{rgb}{0.000000, 0.000000, 0.000000}
\pgfsetstrokecolor{dialinecolor}
\pgfsetstrokeopacity{1.000000}
\draw (0.000000\du,0.000000\du)--(6.000000\du,-14.000000\du);
}
\pgfsetlinewidth{0.100000\du}
\pgfsetdash{}{0pt}
\pgfsetbuttcap
{
\definecolor{diafillcolor}{rgb}{0.000000, 0.000000, 0.000000}
\pgfsetfillcolor{diafillcolor}
\pgfsetfillopacity{1.000000}
% was here!!!
\pgfsetarrowsend{to}
\definecolor{dialinecolor}{rgb}{0.000000, 0.000000, 0.000000}
\pgfsetstrokecolor{dialinecolor}
\pgfsetstrokeopacity{1.000000}
\draw (7.000000\du,-5.000000\du)--(6.000000\du,-14.000000\du);
}
% setfont left to latex
\definecolor{dialinecolor}{rgb}{0.000000, 0.000000, 0.000000}
\pgfsetstrokecolor{dialinecolor}
\pgfsetstrokeopacity{1.000000}
\definecolor{diafillcolor}{rgb}{0.000000, 0.000000, 0.000000}
\pgfsetfillcolor{diafillcolor}
\pgfsetfillopacity{1.000000}
\node[anchor=base west,inner sep=0pt,outer sep=0pt,color=dialinecolor] at (5.350004\du,0.099999\du){$x_1$};
% setfont left to latex
\definecolor{dialinecolor}{rgb}{0.000000, 0.000000, 0.000000}
\pgfsetstrokecolor{dialinecolor}
\pgfsetstrokeopacity{1.000000}
\definecolor{diafillcolor}{rgb}{0.000000, 0.000000, 0.000000}
\pgfsetfillcolor{diafillcolor}
\pgfsetfillopacity{1.000000}
\node[anchor=base west,inner sep=0pt,outer sep=0pt,color=dialinecolor] at (-0.549997\du,-5.450003\du){$x_2$};
% setfont left to latex
\definecolor{dialinecolor}{rgb}{0.000000, 0.000000, 0.000000}
\pgfsetstrokecolor{dialinecolor}
\pgfsetstrokeopacity{1.000000}
\definecolor{diafillcolor}{rgb}{0.000000, 0.000000, 0.000000}
\pgfsetfillcolor{diafillcolor}
\pgfsetfillopacity{1.000000}
\node[anchor=base west,inner sep=0pt,outer sep=0pt,color=dialinecolor] at (6.7\du,-12.000000\du){$u(x)$};
% setfont left to latex
\definecolor{dialinecolor}{rgb}{0.000000, 0.000000, 0.000000}
\pgfsetstrokecolor{dialinecolor}
\pgfsetstrokeopacity{1.000000}
\definecolor{diafillcolor}{rgb}{0.000000, 0.000000, 0.000000}
\pgfsetfillcolor{diafillcolor}
\pgfsetfillopacity{1.000000}
\node[anchor=base west,inner sep=0pt,outer sep=0pt,color=dialinecolor] at (4.9\du,-2.5\du){$x$};
% setfont left to latex
\definecolor{dialinecolor}{rgb}{0.000000, 0.000000, 0.000000}
\pgfsetstrokecolor{dialinecolor}
\pgfsetstrokeopacity{1.000000}
\definecolor{diafillcolor}{rgb}{0.000000, 0.000000, 0.000000}
\pgfsetfillcolor{diafillcolor}
\pgfsetfillopacity{1.000000}
\node[anchor=base west,inner sep=0pt,outer sep=0pt,color=dialinecolor] at (1\du,-9\du){$\chi(x)$};
% setfont left to latex
\definecolor{dialinecolor}{rgb}{0.000000, 0.000000, 0.000000}
\pgfsetstrokecolor{dialinecolor}
\pgfsetstrokeopacity{1.000000}
\definecolor{diafillcolor}{rgb}{0.000000, 0.000000, 0.000000}
\pgfsetfillcolor{diafillcolor}
\pgfsetfillopacity{1.000000}
\node[anchor=base west,inner sep=0pt,outer sep=0pt,color=dialinecolor] at (12.500000\du,-8.7\du){$\Omega$};
% setfont left to latex
\definecolor{dialinecolor}{rgb}{0.000000, 0.000000, 0.000000}
\pgfsetstrokecolor{dialinecolor}
\pgfsetstrokeopacity{1.000000}
\definecolor{diafillcolor}{rgb}{0.000000, 0.000000, 0.000000}
\pgfsetfillcolor{diafillcolor}
\pgfsetfillopacity{1.000000}
\node[anchor=base west,inner sep=0pt,outer sep=0pt,color=dialinecolor] at (8.7\du,-19\du){$\chi(\Omega)$};
\node[anchor=base west,inner sep=0pt,outer sep=0pt,color=dialinecolor] at (-9\du,-14.5\du){(Euler Koordinaten)};
\node[anchor=base west,inner sep=0pt,outer sep=0pt,color=dialinecolor] at (7\du,-3\du){(Lagrange Koordinaten)};
\end{tikzpicture}
	\caption{Eine Deformation in 2D}
	\end{figure}
	\framebreak
	
	
	Man berechnet für die Länge eines kleines deformierten Wegstückes bei $x$
	\begin{align*}
		&\norm{\chi(x+z)-\chi(x)}^2+o(\norm{z}^2) \\
		&= \norm{\nabla\chi(x)z}^2 \\
		&= \norm{\nabla(\Id+u(x))z}^2 \\
		&= \norm{z+\nabla u(x)z}^2 \\
		&= \norm{z}^2 + z^\top(\nabla u(x)+\nabla u(x)^\top)z+z^\top\nabla u(x)^\top\nabla u(x) z \\
		&= \norm{z}^2 + 2z^\top E z 
	\end{align*}
	mit Euler-Lagrange Verrzerrungstensor $E$ gegeben durch
	\begin{align*}
		2E\coloneqq \nabla u(x)+\nabla u(x)^\top+\nabla u(x)^\top\nabla u(x)
	\end{align*}
	\framebreak
	\begin{figure}[h]
	\centering
	% Graphic for TeX using PGF
% Title: /mnt/12CCB7B3CCB79009/Filing/Education/University/Bonn/Courses/Bachelorarbeit/Resources/StrainTensor1.dia
% Creator: Dia v0.97+git
% CreationDate: Mon Apr  4 16:31:15 2022
% For: theo
% \usepackage{tikz}
% The following commands are not supported in PSTricks at present
% We define them conditionally, so when they are implemented,
% this pgf file will use them.
\ifx\du\undefined
  \newlength{\du}
\fi
\setlength{\du}{15\unitlength}
\begin{tikzpicture}[even odd rule]
\pgftransformxscale{0.650000}
\pgftransformyscale{-0.650000}
\definecolor{dialinecolor}{rgb}{0.000000, 0.000000, 0.000000}
\pgfsetstrokecolor{dialinecolor}
\pgfsetstrokeopacity{1.000000}
\definecolor{diafillcolor}{rgb}{1.000000, 1.000000, 1.000000}
\pgfsetfillcolor{diafillcolor}
\pgfsetfillopacity{1.000000}
\pgfsetlinewidth{0.100000\du}
\pgfsetdash{}{0pt}
\pgfsetbuttcap
{
\definecolor{diafillcolor}{rgb}{0.000000, 0.000000, 0.000000}
\pgfsetfillcolor{diafillcolor}
\pgfsetfillopacity{1.000000}
% was here!!!
\pgfsetarrowsend{latex}
\definecolor{dialinecolor}{rgb}{0.000000, 0.000000, 0.000000}
\pgfsetstrokecolor{dialinecolor}
\pgfsetstrokeopacity{1.000000}
\draw (0.000000\du,0.000000\du)--(5.000000\du,0.000000\du);
}
\pgfsetlinewidth{0.100000\du}
\pgfsetdash{}{0pt}
\pgfsetbuttcap
{
\definecolor{diafillcolor}{rgb}{0.000000, 0.000000, 0.000000}
\pgfsetfillcolor{diafillcolor}
\pgfsetfillopacity{1.000000}
% was here!!!
\pgfsetarrowsend{latex}
\definecolor{dialinecolor}{rgb}{0.000000, 0.000000, 0.000000}
\pgfsetstrokecolor{dialinecolor}
\pgfsetstrokeopacity{1.000000}
\draw (0.000000\du,0.000000\du)--(0.000000\du,-5.000000\du);
}
\pgfsetlinewidth{0.100000\du}
\pgfsetdash{}{0pt}
\pgfsetbuttcap
{
\definecolor{diafillcolor}{rgb}{0.000000, 0.000000, 0.000000}
\pgfsetfillcolor{diafillcolor}
\pgfsetfillopacity{1.000000}
% was here!!!
\pgfsetarrowsend{to}
\definecolor{dialinecolor}{rgb}{0.000000, 0.000000, 0.000000}
\pgfsetstrokecolor{dialinecolor}
\pgfsetstrokeopacity{1.000000}
\draw (8.000000\du,-4.000000\du)--(12.000000\du,-5.000000\du);
}
\pgfsetlinewidth{0.100000\du}
\pgfsetdash{}{0pt}
\pgfsetbuttcap
{
\definecolor{diafillcolor}{rgb}{0.000000, 0.000000, 0.000000}
\pgfsetfillcolor{diafillcolor}
\pgfsetfillopacity{1.000000}
% was here!!!
\pgfsetarrowsend{to}
\definecolor{dialinecolor}{rgb}{0.000000, 0.000000, 0.000000}
\pgfsetstrokecolor{dialinecolor}
\pgfsetstrokeopacity{1.000000}
\draw (2.148705\du,-8.358339\du)--(4.000000\du,-12.000000\du);
}
\pgfsetlinewidth{0.100000\du}
\pgfsetdash{{\pgflinewidth}{0.200000\du}}{0cm}
\pgfsetbuttcap
{
\definecolor{diafillcolor}{rgb}{0.000000, 0.000000, 0.000000}
\pgfsetfillcolor{diafillcolor}
\pgfsetfillopacity{1.000000}
% was here!!!
\pgfsetarrowsend{to}
\definecolor{dialinecolor}{rgb}{0.000000, 0.000000, 0.000000}
\pgfsetstrokecolor{dialinecolor}
\pgfsetstrokeopacity{1.000000}
\draw (4.000000\du,-12.000000\du)--(7.000000\du,-18.000000\du);
}
\pgfsetlinewidth{0.050000\du}
\pgfsetdash{}{0pt}
\pgfsetmiterjoin
\pgfsetbuttcap
\definecolor{diafillcolor}{rgb}{1.000000, 1.000000, 1.000000}
\pgfsetfillcolor{diafillcolor}
\pgfsetfillopacity{0.000000}
\definecolor{dialinecolor}{rgb}{0.000000, 0.000000, 0.000000}
\pgfsetstrokecolor{dialinecolor}
\pgfsetstrokeopacity{1.000000}
\pgfpathmoveto{\pgfpoint{10.584766\du}{-21.209025\du}}
\pgfpathcurveto{\pgfpoint{14.334754\du}{-19.159032\du}}{\pgfpoint{8.084774\du}{-4.959077\du}}{\pgfpoint{2.184793\du}{-5.359075\du}}
\pgfpathcurveto{\pgfpoint{-3.715189\du}{-5.759074\du}}{\pgfpoint{6.834778\du}{-23.259019\du}}{\pgfpoint{10.584766\du}{-21.209025\du}}
\pgfpathclose
\pgfusepath{fill,stroke}
\pgfsetlinewidth{0.050000\du}
\pgfsetdash{}{0pt}
\pgfsetmiterjoin
\pgfsetbuttcap
\definecolor{diafillcolor}{rgb}{1.000000, 1.000000, 1.000000}
\pgfsetfillcolor{diafillcolor}
\pgfsetfillopacity{0.000000}
\definecolor{dialinecolor}{rgb}{0.000000, 0.000000, 0.000000}
\pgfsetstrokecolor{dialinecolor}
\pgfsetstrokeopacity{1.000000}
\pgfpathmoveto{\pgfpoint{15.048146\du}{-4.895365\du}}
\pgfpathcurveto{\pgfpoint{18.663996\du}{-3.129485\du}}{\pgfpoint{9.338473\du}{-0.986923\du}}{\pgfpoint{7.269864\du}{-2.982328\du}}
\pgfpathcurveto{\pgfpoint{5.201255\du}{-4.977734\du}}{\pgfpoint{11.432296\du}{-6.661246\du}}{\pgfpoint{15.048146\du}{-4.895365\du}}
\pgfpathclose
\pgfusepath{fill,stroke}
% setfont left to latex
\definecolor{dialinecolor}{rgb}{0.000000, 0.000000, 0.000000}
\pgfsetstrokecolor{dialinecolor}
\pgfsetstrokeopacity{1.000000}
\definecolor{diafillcolor}{rgb}{0.000000, 0.000000, 0.000000}
\pgfsetfillcolor{diafillcolor}
\pgfsetfillopacity{1.000000}
\node[anchor=base west,inner sep=0pt,outer sep=0pt,color=dialinecolor] at (11.159005\du,-3.938847\du){};
% setfont left to latex
\definecolor{dialinecolor}{rgb}{0.000000, 0.000000, 0.000000}
\pgfsetstrokecolor{dialinecolor}
\pgfsetstrokeopacity{1.000000}
\definecolor{diafillcolor}{rgb}{0.000000, 0.000000, 0.000000}
\pgfsetfillcolor{diafillcolor}
\pgfsetfillopacity{1.000000}
\node[anchor=base west,inner sep=0pt,outer sep=0pt,color=dialinecolor] at (8.114833\du,-3.053250\du){$x$};
% setfont left to latex
\definecolor{dialinecolor}{rgb}{0.000000, 0.000000, 0.000000}
\pgfsetstrokecolor{dialinecolor}
\pgfsetstrokeopacity{1.000000}
\definecolor{diafillcolor}{rgb}{0.000000, 0.000000, 0.000000}
\pgfsetfillcolor{diafillcolor}
\pgfsetfillopacity{1.000000}
\node[anchor=base west,inner sep=0pt,outer sep=0pt,color=dialinecolor] at (12.402875\du,-4.287354\du){$x+z$};
% setfont left to latex
\definecolor{dialinecolor}{rgb}{0.000000, 0.000000, 0.000000}
\pgfsetstrokecolor{dialinecolor}
\pgfsetstrokeopacity{1.000000}
\definecolor{diafillcolor}{rgb}{0.000000, 0.000000, 0.000000}
\pgfsetfillcolor{diafillcolor}
\pgfsetfillopacity{1.000000}
\node[anchor=base west,inner sep=0pt,outer sep=0pt,color=dialinecolor] at (10.435460\du,-3.487484\du){$z$};
% setfont left to latex
\definecolor{dialinecolor}{rgb}{0.000000, 0.000000, 0.000000}
\pgfsetstrokecolor{dialinecolor}
\pgfsetstrokeopacity{1.000000}
\definecolor{diafillcolor}{rgb}{0.000000, 0.000000, 0.000000}
\pgfsetfillcolor{diafillcolor}
\pgfsetfillopacity{1.000000}
\node[anchor=base west,inner sep=0pt,outer sep=0pt,color=dialinecolor] at (15.349558\du,-2.032635\du){$\Omega$};
% setfont left to latex
\definecolor{dialinecolor}{rgb}{0.000000, 0.000000, 0.000000}
\pgfsetstrokecolor{dialinecolor}
\pgfsetstrokeopacity{1.000000}
\definecolor{diafillcolor}{rgb}{0.000000, 0.000000, 0.000000}
\pgfsetfillcolor{diafillcolor}
\pgfsetfillopacity{1.000000}
\node[anchor=base west,inner sep=0pt,outer sep=0pt,color=dialinecolor] at (1.247838\du,-6.970235\du){$\chi(x)$};
% setfont left to latex
\definecolor{dialinecolor}{rgb}{0.000000, 0.000000, 0.000000}
\pgfsetstrokecolor{dialinecolor}
\pgfsetstrokeopacity{1.000000}
\definecolor{diafillcolor}{rgb}{0.000000, 0.000000, 0.000000}
\pgfsetfillcolor{diafillcolor}
\pgfsetfillopacity{1.000000}
\node[anchor=base west,inner sep=0pt,outer sep=0pt,color=dialinecolor] at (7.500360\du,-17.879076\du){$\chi(x+z)$};
% setfont left to latex
\definecolor{dialinecolor}{rgb}{0.000000, 0.000000, 0.000000}
\pgfsetstrokecolor{dialinecolor}
\pgfsetstrokeopacity{1.000000}
\definecolor{diafillcolor}{rgb}{0.000000, 0.000000, 0.000000}
\pgfsetfillcolor{diafillcolor}
\pgfsetfillopacity{1.000000}
\node[anchor=base west,inner sep=0pt,outer sep=0pt,color=dialinecolor] at (6.730380\du,-13.326092\du){$2z^\top Ez$};
\pgfsetlinewidth{0.100000\du}
\pgfsetdash{}{0pt}
\pgfsetmiterjoin
\pgfsetbuttcap
{
\definecolor{diafillcolor}{rgb}{0.000000, 0.000000, 0.000000}
\pgfsetfillcolor{diafillcolor}
\pgfsetfillopacity{1.000000}
% was here!!!
\definecolor{dialinecolor}{rgb}{0.000000, 0.000000, 0.000000}
\pgfsetstrokecolor{dialinecolor}
\pgfsetstrokeopacity{1.000000}
\pgfpathmoveto{\pgfpoint{6.951524\du}{-17.899953\du}}
\pgfpathcurveto{\pgfpoint{7.978614\du}{-17.387152\du}}{\pgfpoint{5.395871\du}{-15.082960\du}}{\pgfpoint{6.468010\du}{-14.347179\du}}
\pgfusepath{stroke}
}
\pgfsetlinewidth{0.100000\du}
\pgfsetdash{}{0pt}
\pgfsetmiterjoin
\pgfsetbuttcap
{
\definecolor{diafillcolor}{rgb}{0.000000, 0.000000, 0.000000}
\pgfsetfillcolor{diafillcolor}
\pgfsetfillopacity{1.000000}
% was here!!!
\definecolor{dialinecolor}{rgb}{0.000000, 0.000000, 0.000000}
\pgfsetstrokecolor{dialinecolor}
\pgfsetstrokeopacity{1.000000}
\pgfpathmoveto{\pgfpoint{6.425966\du}{-14.326156\du}}
\pgfpathcurveto{\pgfpoint{5.269737\du}{-15.061938\du}}{\pgfpoint{5.403991\du}{-10.855418\du}}{\pgfpoint{4.029420\du}{-11.929610\du}}
\pgfusepath{stroke}
}
% setfont left to latex
\definecolor{dialinecolor}{rgb}{0.000000, 0.000000, 0.000000}
\pgfsetstrokecolor{dialinecolor}
\pgfsetstrokeopacity{1.000000}
\definecolor{diafillcolor}{rgb}{0.000000, 0.000000, 0.000000}
\pgfsetfillcolor{diafillcolor}
\pgfsetfillopacity{1.000000}
\node[anchor=base west,inner sep=0pt,outer sep=0pt,color=dialinecolor] at (12.587382\du,-15.788603\du){$\chi(\Omega)$};
% setfont left to latex
\definecolor{dialinecolor}{rgb}{0.000000, 0.000000, 0.000000}
\pgfsetstrokecolor{dialinecolor}
\pgfsetstrokeopacity{1.000000}
\definecolor{diafillcolor}{rgb}{0.000000, 0.000000, 0.000000}
\pgfsetfillcolor{diafillcolor}
\pgfsetfillopacity{1.000000}
\node[anchor=base west,inner sep=0pt,outer sep=0pt,color=dialinecolor] at (-0.646234\du,-5.334980\du){$x_2$};
% setfont left to latex
\definecolor{dialinecolor}{rgb}{0.000000, 0.000000, 0.000000}
\pgfsetstrokecolor{dialinecolor}
\pgfsetstrokeopacity{1.000000}
\definecolor{diafillcolor}{rgb}{0.000000, 0.000000, 0.000000}
\pgfsetfillcolor{diafillcolor}
\pgfsetfillopacity{1.000000}
\node[anchor=base west,inner sep=0pt,outer sep=0pt,color=dialinecolor] at (5.481312\du,0.484568\du){$x_1$};
\end{tikzpicture}

	\caption{Eine Anschauliche Darstellung}
	\end{figure}
	
	\framebreak
	Für kleine Verzerrungen, wie wir sie im Folgenden annehmen, ist der letzte Term vernachlässigbar und wir haben
	\begin{align*}
		E\approx \frac{1}{2}\left(\nabla u+\nabla u^\top\right)\eqqcolon\e
	\end{align*}
	mit $\e$ dem linearisierten Verzerrungstensor.
\end{frame}

\subsubsection*{Kräfte, Cauchy-Tensor}
\begin{frame}
	Vorraussetzung des Models: Der statische deformierte Körper (Euler-Koordinaten) nimmt den Raum $\Omega$ ein und befindet sich im Kräftegleichgewicht. Wir definieren
	\begin{itemize}
	\item Volumenkräfte: Eine Abbildung $f\colon\Omega\to\R^d$
	\item Oberflächenkräfte: Eine Abbildung $\sigma\colon\Omega\to\R^{d\times d}$ (Cauchyscher Spannungstensor). $\sigma_{ij}$ bezeichnet die Kraft auf die Fläche j in Richtung i wirkt. Kraft, die auf Oberfläche in Richtung $n$ wirkt ist
	\begin{align*}
		\sigma n =\sum_j\sigma_{ij}n_je_i
	\end{align*}
	\end{itemize}
\end{frame}

\begin{frame}
	\begin{figure}[h]
	\centering
	% Graphic for TeX using PGF
% Title: /mnt/12CCB7B3CCB79009/Filing/Education/University/Bonn/Courses/Bachelorarbeit/Resources/StressTensor1.dia
% Creator: Dia v0.97+git
% CreationDate: Mon Apr  4 16:00:44 2022
% For: theo
% \usepackage{tikz}
% The following commands are not supported in PSTricks at present
% We define them conditionally, so when they are implemented,
% this pgf file will use them.
\ifx\du\undefined
  \newlength{\du}
\fi
\setlength{\du}{15\unitlength}
\begin{tikzpicture}[even odd rule]
\pgftransformxscale{0.5}
\pgftransformyscale{-0.5}
\definecolor{dialinecolor}{rgb}{0.000000, 0.000000, 0.000000}
\pgfsetstrokecolor{dialinecolor}
\pgfsetstrokeopacity{1.000000}
\definecolor{diafillcolor}{rgb}{1.000000, 1.000000, 1.000000}
\pgfsetfillcolor{diafillcolor}
\pgfsetfillopacity{1.000000}
\pgfsetlinewidth{0.100000\du}
\pgfsetdash{}{0pt}
\pgfsetbuttcap
{
\definecolor{diafillcolor}{rgb}{0.000000, 0.000000, 0.000000}
\pgfsetfillcolor{diafillcolor}
\pgfsetfillopacity{1.000000}
% was here!!!
\definecolor{dialinecolor}{rgb}{0.000000, 0.000000, 0.000000}
\pgfsetstrokecolor{dialinecolor}
\pgfsetstrokeopacity{1.000000}
\draw (3.000000\du,-2.000000\du)--(3.000000\du,-10.000000\du);
}
\pgfsetlinewidth{0.100000\du}
\pgfsetdash{}{0pt}
\pgfsetbuttcap
{
\definecolor{diafillcolor}{rgb}{0.000000, 0.000000, 0.000000}
\pgfsetfillcolor{diafillcolor}
\pgfsetfillopacity{1.000000}
% was here!!!
\definecolor{dialinecolor}{rgb}{0.000000, 0.000000, 0.000000}
\pgfsetstrokecolor{dialinecolor}
\pgfsetstrokeopacity{1.000000}
\draw (3.000000\du,-10.000000\du)--(9.000000\du,-7.000000\du);
}
\pgfsetlinewidth{0.100000\du}
\pgfsetdash{}{0pt}
\pgfsetbuttcap
{
\definecolor{diafillcolor}{rgb}{0.000000, 0.000000, 0.000000}
\pgfsetfillcolor{diafillcolor}
\pgfsetfillopacity{1.000000}
% was here!!!
\definecolor{dialinecolor}{rgb}{0.000000, 0.000000, 0.000000}
\pgfsetstrokecolor{dialinecolor}
\pgfsetstrokeopacity{1.000000}
\draw (3.000000\du,-2.000000\du)--(9.000000\du,1.000000\du);
}
\pgfsetlinewidth{0.100000\du}
\pgfsetdash{}{0pt}
\pgfsetbuttcap
{
\definecolor{diafillcolor}{rgb}{0.000000, 0.000000, 0.000000}
\pgfsetfillcolor{diafillcolor}
\pgfsetfillopacity{1.000000}
% was here!!!
\definecolor{dialinecolor}{rgb}{0.000000, 0.000000, 0.000000}
\pgfsetstrokecolor{dialinecolor}
\pgfsetstrokeopacity{1.000000}
\draw (9.000000\du,-7.000000\du)--(9.000000\du,1.000000\du);
}
\pgfsetlinewidth{0.100000\du}
\pgfsetdash{}{0pt}
\pgfsetbuttcap
{
\definecolor{diafillcolor}{rgb}{0.000000, 0.000000, 0.000000}
\pgfsetfillcolor{diafillcolor}
\pgfsetfillopacity{1.000000}
% was here!!!
\definecolor{dialinecolor}{rgb}{0.000000, 0.000000, 0.000000}
\pgfsetstrokecolor{dialinecolor}
\pgfsetstrokeopacity{1.000000}
\draw (9.000000\du,1.000000\du)--(15.000000\du,-2.000000\du);
}
\pgfsetlinewidth{0.100000\du}
\pgfsetdash{}{0pt}
\pgfsetbuttcap
{
\definecolor{diafillcolor}{rgb}{0.000000, 0.000000, 0.000000}
\pgfsetfillcolor{diafillcolor}
\pgfsetfillopacity{1.000000}
% was here!!!
\definecolor{dialinecolor}{rgb}{0.000000, 0.000000, 0.000000}
\pgfsetstrokecolor{dialinecolor}
\pgfsetstrokeopacity{1.000000}
\draw (15.000000\du,-2.000000\du)--(15.000000\du,-10.000000\du);
}
\pgfsetlinewidth{0.100000\du}
\pgfsetdash{}{0pt}
\pgfsetbuttcap
{
\definecolor{diafillcolor}{rgb}{0.000000, 0.000000, 0.000000}
\pgfsetfillcolor{diafillcolor}
\pgfsetfillopacity{1.000000}
% was here!!!
\definecolor{dialinecolor}{rgb}{0.000000, 0.000000, 0.000000}
\pgfsetstrokecolor{dialinecolor}
\pgfsetstrokeopacity{1.000000}
\draw (15.000000\du,-10.000000\du)--(9.000000\du,-7.000000\du);
}
\pgfsetlinewidth{0.100000\du}
\pgfsetdash{}{0pt}
\pgfsetbuttcap
{
\definecolor{diafillcolor}{rgb}{0.000000, 0.000000, 0.000000}
\pgfsetfillcolor{diafillcolor}
\pgfsetfillopacity{1.000000}
% was here!!!
\definecolor{dialinecolor}{rgb}{0.000000, 0.000000, 0.000000}
\pgfsetstrokecolor{dialinecolor}
\pgfsetstrokeopacity{1.000000}
\draw (3.000000\du,-10.000000\du)--(9.000000\du,-13.000000\du);
}
\pgfsetlinewidth{0.100000\du}
\pgfsetdash{}{0pt}
\pgfsetbuttcap
{
\definecolor{diafillcolor}{rgb}{0.000000, 0.000000, 0.000000}
\pgfsetfillcolor{diafillcolor}
\pgfsetfillopacity{1.000000}
% was here!!!
\definecolor{dialinecolor}{rgb}{0.000000, 0.000000, 0.000000}
\pgfsetstrokecolor{dialinecolor}
\pgfsetstrokeopacity{1.000000}
\draw (9.000000\du,-13.000000\du)--(15.000000\du,-10.000000\du);
}
\pgfsetlinewidth{0.100000\du}
\pgfsetdash{}{0pt}
\pgfsetbuttcap
{
\definecolor{diafillcolor}{rgb}{0.000000, 0.000000, 0.000000}
\pgfsetfillcolor{diafillcolor}
\pgfsetfillopacity{1.000000}
% was here!!!
\pgfsetarrowsend{to}
\definecolor{dialinecolor}{rgb}{0.000000, 0.000000, 0.000000}
\pgfsetstrokecolor{dialinecolor}
\pgfsetstrokeopacity{1.000000}
\draw (9.000000\du,-10.000000\du)--(11.000000\du,-9.000000\du);
}
\pgfsetlinewidth{0.100000\du}
\pgfsetdash{}{0pt}
\pgfsetbuttcap
{
\definecolor{diafillcolor}{rgb}{0.000000, 0.000000, 0.000000}
\pgfsetfillcolor{diafillcolor}
\pgfsetfillopacity{1.000000}
% was here!!!
\pgfsetarrowsend{to}
\definecolor{dialinecolor}{rgb}{0.000000, 0.000000, 0.000000}
\pgfsetstrokecolor{dialinecolor}
\pgfsetstrokeopacity{1.000000}
\draw (9.000000\du,-10.000000\du)--(7.000000\du,-9.000000\du);
}
\pgfsetlinewidth{0.100000\du}
\pgfsetdash{}{0pt}
\pgfsetbuttcap
{
\definecolor{diafillcolor}{rgb}{0.000000, 0.000000, 0.000000}
\pgfsetfillcolor{diafillcolor}
\pgfsetfillopacity{1.000000}
% was here!!!
\pgfsetarrowsend{to}
\definecolor{dialinecolor}{rgb}{0.000000, 0.000000, 0.000000}
\pgfsetstrokecolor{dialinecolor}
\pgfsetstrokeopacity{1.000000}
\draw (9.000000\du,-10.000000\du)--(9.000000\du,-12.000000\du);
}
\pgfsetlinewidth{0.100000\du}
\pgfsetdash{}{0pt}
\pgfsetbuttcap
{
\definecolor{diafillcolor}{rgb}{0.000000, 0.000000, 0.000000}
\pgfsetfillcolor{diafillcolor}
\pgfsetfillopacity{1.000000}
% was here!!!
\pgfsetarrowsend{to}
\definecolor{dialinecolor}{rgb}{0.000000, 0.000000, 0.000000}
\pgfsetstrokecolor{dialinecolor}
\pgfsetstrokeopacity{1.000000}
\draw (12.050000\du,-4.650000\du)--(12.050000\du,-6.650000\du);
}
\pgfsetlinewidth{0.100000\du}
\pgfsetdash{}{0pt}
\pgfsetbuttcap
{
\definecolor{diafillcolor}{rgb}{0.000000, 0.000000, 0.000000}
\pgfsetfillcolor{diafillcolor}
\pgfsetfillopacity{1.000000}
% was here!!!
\pgfsetarrowsend{to}
\definecolor{dialinecolor}{rgb}{0.000000, 0.000000, 0.000000}
\pgfsetstrokecolor{dialinecolor}
\pgfsetstrokeopacity{1.000000}
\draw (12.050000\du,-4.650000\du)--(14.050000\du,-3.650000\du);
}
\pgfsetlinewidth{0.100000\du}
\pgfsetdash{}{0pt}
\pgfsetbuttcap
{
\definecolor{diafillcolor}{rgb}{0.000000, 0.000000, 0.000000}
\pgfsetfillcolor{diafillcolor}
\pgfsetfillopacity{1.000000}
% was here!!!
\pgfsetarrowsend{to}
\definecolor{dialinecolor}{rgb}{0.000000, 0.000000, 0.000000}
\pgfsetstrokecolor{dialinecolor}
\pgfsetstrokeopacity{1.000000}
\draw (12.050000\du,-4.650000\du)--(10.050000\du,-3.650000\du);
}
\pgfsetlinewidth{0.100000\du}
\pgfsetdash{}{0pt}
\pgfsetbuttcap
{
\definecolor{diafillcolor}{rgb}{0.000000, 0.000000, 0.000000}
\pgfsetfillcolor{diafillcolor}
\pgfsetfillopacity{1.000000}
% was here!!!
\pgfsetarrowsend{to}
\definecolor{dialinecolor}{rgb}{0.000000, 0.000000, 0.000000}
\pgfsetstrokecolor{dialinecolor}
\pgfsetstrokeopacity{1.000000}
\draw (6.000000\du,-4.700000\du)--(6.000000\du,-6.700000\du);
}
\pgfsetlinewidth{0.100000\du}
\pgfsetdash{}{0pt}
\pgfsetbuttcap
{
\definecolor{diafillcolor}{rgb}{0.000000, 0.000000, 0.000000}
\pgfsetfillcolor{diafillcolor}
\pgfsetfillopacity{1.000000}
% was here!!!
\pgfsetarrowsend{to}
\definecolor{dialinecolor}{rgb}{0.000000, 0.000000, 0.000000}
\pgfsetstrokecolor{dialinecolor}
\pgfsetstrokeopacity{1.000000}
\draw (6.000000\du,-4.700000\du)--(8.000000\du,-3.700000\du);
}
\pgfsetlinewidth{0.100000\du}
\pgfsetdash{}{0pt}
\pgfsetbuttcap
{
\definecolor{diafillcolor}{rgb}{0.000000, 0.000000, 0.000000}
\pgfsetfillcolor{diafillcolor}
\pgfsetfillopacity{1.000000}
% was here!!!
\pgfsetarrowsend{to}
\definecolor{dialinecolor}{rgb}{0.000000, 0.000000, 0.000000}
\pgfsetstrokecolor{dialinecolor}
\pgfsetstrokeopacity{1.000000}
\draw (6.000000\du,-4.700000\du)--(4.000000\du,-3.700000\du);
}
\pgfsetlinewidth{0.100000\du}
\pgfsetdash{}{0pt}
\pgfsetbuttcap
{
\definecolor{diafillcolor}{rgb}{0.000000, 0.000000, 0.000000}
\pgfsetfillcolor{diafillcolor}
\pgfsetfillopacity{1.000000}
% was here!!!
\pgfsetarrowsend{latex}
\definecolor{dialinecolor}{rgb}{0.000000, 0.000000, 0.000000}
\pgfsetstrokecolor{dialinecolor}
\pgfsetstrokeopacity{1.000000}
\draw (0.000000\du,0.000000\du)--(4.000000\du,2.000000\du);
}
\pgfsetlinewidth{0.100000\du}
\pgfsetdash{}{0pt}
\pgfsetbuttcap
{
\definecolor{diafillcolor}{rgb}{0.000000, 0.000000, 0.000000}
\pgfsetfillcolor{diafillcolor}
\pgfsetfillopacity{1.000000}
% was here!!!
\pgfsetarrowsend{latex}
\definecolor{dialinecolor}{rgb}{0.000000, 0.000000, 0.000000}
\pgfsetstrokecolor{dialinecolor}
\pgfsetstrokeopacity{1.000000}
\draw (0.000000\du,0.000000\du)--(-4.000000\du,2.000000\du);
}
\pgfsetlinewidth{0.100000\du}
\pgfsetdash{}{0pt}
\pgfsetbuttcap
{
\definecolor{diafillcolor}{rgb}{0.000000, 0.000000, 0.000000}
\pgfsetfillcolor{diafillcolor}
\pgfsetfillopacity{1.000000}
% was here!!!
\pgfsetarrowsend{latex}
\definecolor{dialinecolor}{rgb}{0.000000, 0.000000, 0.000000}
\pgfsetstrokecolor{dialinecolor}
\pgfsetstrokeopacity{1.000000}
\draw (0.000000\du,0.000000\du)--(0.000000\du,-4.000000\du);
}
% setfont left to latex
\definecolor{dialinecolor}{rgb}{0.000000, 0.000000, 0.000000}
\pgfsetstrokecolor{dialinecolor}
\pgfsetstrokeopacity{1.000000}
\definecolor{diafillcolor}{rgb}{0.000000, 0.000000, 0.000000}
\pgfsetfillcolor{diafillcolor}
\pgfsetfillopacity{1.000000}
\node[anchor=base west,inner sep=0pt,outer sep=0pt,color=dialinecolor] at (12.9\du,-2.687500\du){$\sigma_{22}$};
% setfont left to latex
\definecolor{dialinecolor}{rgb}{0.000000, 0.000000, 0.000000}
\pgfsetstrokecolor{dialinecolor}
\pgfsetstrokeopacity{1.000000}
\definecolor{diafillcolor}{rgb}{0.000000, 0.000000, 0.000000}
\pgfsetfillcolor{diafillcolor}
\pgfsetfillopacity{1.000000}
\node[anchor=base west,inner sep=0pt,outer sep=0pt,color=dialinecolor] at (12.600000\du,-6.787500\du){$\sigma_{32}$};
% setfont left to latex
\definecolor{dialinecolor}{rgb}{0.000000, 0.000000, 0.000000}
\pgfsetstrokecolor{dialinecolor}
\pgfsetstrokeopacity{1.000000}
\definecolor{diafillcolor}{rgb}{0.000000, 0.000000, 0.000000}
\pgfsetfillcolor{diafillcolor}
\pgfsetfillopacity{1.000000}
\node[anchor=base west,inner sep=0pt,outer sep=0pt,color=dialinecolor] at (10\du,-2.4\du){$\sigma_{12}$};
% setfont left to latex
\definecolor{dialinecolor}{rgb}{0.000000, 0.000000, 0.000000}
\pgfsetstrokecolor{dialinecolor}
\pgfsetstrokeopacity{1.000000}
\definecolor{diafillcolor}{rgb}{0.000000, 0.000000, 0.000000}
\pgfsetfillcolor{diafillcolor}
\pgfsetfillopacity{1.000000}
\node[anchor=base west,inner sep=0pt,outer sep=0pt,color=dialinecolor] at (7\du,-2.4\du){$\sigma_{21}$};
% setfont left to latex
\definecolor{dialinecolor}{rgb}{0.000000, 0.000000, 0.000000}
\pgfsetstrokecolor{dialinecolor}
\pgfsetstrokeopacity{1.000000}
\definecolor{diafillcolor}{rgb}{0.000000, 0.000000, 0.000000}
\pgfsetfillcolor{diafillcolor}
\pgfsetfillopacity{1.000000}
\node[anchor=base west,inner sep=0pt,outer sep=0pt,color=dialinecolor] at (4.350000\du,2.512500\du){$x_2$};
% setfont left to latex
\definecolor{dialinecolor}{rgb}{0.000000, 0.000000, 0.000000}
\pgfsetstrokecolor{dialinecolor}
\pgfsetstrokeopacity{1.000000}
\definecolor{diafillcolor}{rgb}{0.000000, 0.000000, 0.000000}
\pgfsetfillcolor{diafillcolor}
\pgfsetfillopacity{1.000000}
\node[anchor=base west,inner sep=0pt,outer sep=0pt,color=dialinecolor] at (-5.2\du,2.5\du){$x_1$};
% setfont left to latex
\definecolor{dialinecolor}{rgb}{0.000000, 0.000000, 0.000000}
\pgfsetstrokecolor{dialinecolor}
\pgfsetstrokeopacity{1.000000}
\definecolor{diafillcolor}{rgb}{0.000000, 0.000000, 0.000000}
\pgfsetfillcolor{diafillcolor}
\pgfsetfillopacity{1.000000}
\node[anchor=base west,inner sep=0pt,outer sep=0pt,color=dialinecolor] at (-0.400000\du,-4.587500\du){$x_3$};
% setfont left to latex
\definecolor{dialinecolor}{rgb}{0.000000, 0.000000, 0.000000}
\pgfsetstrokecolor{dialinecolor}
\pgfsetstrokeopacity{1.000000}
\definecolor{diafillcolor}{rgb}{0.000000, 0.000000, 0.000000}
\pgfsetfillcolor{diafillcolor}
\pgfsetfillopacity{1.000000}
\node[anchor=base west,inner sep=0pt,outer sep=0pt,color=dialinecolor] at (5.300000\du,-7.087500\du){$\sigma_{31}$};
% setfont left to latex
\definecolor{dialinecolor}{rgb}{0.000000, 0.000000, 0.000000}
\pgfsetstrokecolor{dialinecolor}
\pgfsetstrokeopacity{1.000000}
\definecolor{diafillcolor}{rgb}{0.000000, 0.000000, 0.000000}
\pgfsetfillcolor{diafillcolor}
\pgfsetfillopacity{1.000000}
\node[anchor=base west,inner sep=0pt,outer sep=0pt,color=dialinecolor] at (4\du,-2.6\du){$\sigma_{11}$};
% setfont left to latex
\definecolor{dialinecolor}{rgb}{0.000000, 0.000000, 0.000000}
\pgfsetstrokecolor{dialinecolor}
\pgfsetstrokeopacity{1.000000}
\definecolor{diafillcolor}{rgb}{0.000000, 0.000000, 0.000000}
\pgfsetfillcolor{diafillcolor}
\pgfsetfillopacity{1.000000}
\node[anchor=base west,inner sep=0pt,outer sep=0pt,color=dialinecolor] at (9.650000\du,-11.287500\du){$\sigma_{33}$};
% setfont left to latex
\definecolor{dialinecolor}{rgb}{0.000000, 0.000000, 0.000000}
\pgfsetstrokecolor{dialinecolor}
\pgfsetstrokeopacity{1.000000}
\definecolor{diafillcolor}{rgb}{0.000000, 0.000000, 0.000000}
\pgfsetfillcolor{diafillcolor}
\pgfsetfillopacity{1.000000}
\node[anchor=base west,inner sep=0pt,outer sep=0pt,color=dialinecolor] at (11.4\du,-9.237500\du){$\sigma_{23}$};
% setfont left to latex
\definecolor{dialinecolor}{rgb}{0.000000, 0.000000, 0.000000}
\pgfsetstrokecolor{dialinecolor}
\pgfsetstrokeopacity{1.000000}
\definecolor{diafillcolor}{rgb}{0.000000, 0.000000, 0.000000}
\pgfsetfillcolor{diafillcolor}
\pgfsetfillopacity{1.000000}
\node[anchor=base west,inner sep=0pt,outer sep=0pt,color=dialinecolor] at (5.8\du,-9.9\du){$\sigma_{13}$};
\end{tikzpicture}

	\caption{Eine mögliche Visualisierung des Spannungstensors in 3D}
	\end{figure}
\end{frame}

\subsubsection*{Integralformulierung}
\begin{frame}
	\frametitle{Erste Formulierung des Problems}
	Wir bezeichnen $\Gamma\coloneqq \partial\Omega$ als den Rand des Gebietes mit $\Gamma_D\subseteq\Gamma$ als den Dirichlet- und $\Gamma_N\coloneqq\Gamma\setminus\Gamma_D$ als den Neumann-Rand. Die Gleichgewichtsbeziehung liefert die Formulierung: Finde eine Deformation $u$, so dass
	\vspace{1.5cm}
	\begin{align*}
		\int_\omega \tikzmark{forcevol}{f}\dif x+\int_{\partial\omega}\tikzmark{forcesurf}{\sigma n}\dif s&=0 &,\text{für alle }\omega\subseteq\Omega \\
		u &= w &,\text{auf }\Gamma_D \\
		\sigma n&= g &,\text{auf }\Gamma_N
	\end{align*}
	\begin{tikzpicture}[remember picture, overlay, node distance = 1cm]
		\node[,text width=3cm] (forcevoldescr) [above left=0.5cm and -2cm of forcevol ]{Volumenkräfte, die auf $\omega$ wirken};
		\draw[,->,thick] (forcevoldescr) to [in=90,out=-90] (forcevol);
		\node[,text width=6cm] (forcesurfdescr) [above right=0.8cm and -0.5cm of forcesurf ]{Oberflächenkräfte, die über die Oberfläche $\partial\omega$ auf $\omega$ wirken};
		\draw[,->,thick] (forcesurfdescr) to [in=90,out=-90] (forcesurf);
	\end{tikzpicture}
	wobei $\sigma$ von $u$ abhängt und $f,g$ als von $u$ unabhängig angenommen werden (tote Lasten). 
\end{frame}

\begin{frame}
	\begin{figure}[h]
	\centering
	% Graphic for TeX using PGF
% Title: /mnt/12CCB7B3CCB79009/Filing/Education/University/Bonn/Courses/Bachelorarbeit/Resources/BoundaryConditions1.dia
% Creator: Dia v0.97+git
% CreationDate: Mon Apr  4 20:43:29 2022
% For: theo
% \usepackage{tikz}
% The following commands are not supported in PSTricks at present
% We define them conditionally, so when they are implemented,
% this pgf file will use them.
\ifx\du\undefined
  \newlength{\du}
\fi
\setlength{\du}{15\unitlength}
\begin{tikzpicture}[even odd rule]
\pgftransformxscale{0.500000}
\pgftransformyscale{-0.500000}
\definecolor{dialinecolor}{rgb}{0.000000, 0.000000, 0.000000}
\pgfsetstrokecolor{dialinecolor}
\pgfsetstrokeopacity{1.000000}
\definecolor{diafillcolor}{rgb}{1.000000, 1.000000, 1.000000}
\pgfsetfillcolor{diafillcolor}
\pgfsetfillopacity{1.000000}
\pgfsetlinewidth{0.100000\du}
\pgfsetdash{}{0pt}
\pgfsetbuttcap
{
\definecolor{diafillcolor}{rgb}{0.000000, 0.000000, 0.000000}
\pgfsetfillcolor{diafillcolor}
\pgfsetfillopacity{1.000000}
% was here!!!
\definecolor{dialinecolor}{rgb}{0.000000, 0.000000, 0.000000}
\pgfsetstrokecolor{dialinecolor}
\pgfsetstrokeopacity{1.000000}
\draw (0.000000\du,0.000000\du)--(0.000000\du,-10.000000\du);
}
\pgfsetlinewidth{0.100000\du}
\pgfsetdash{}{0pt}
\pgfsetbuttcap
{
\definecolor{diafillcolor}{rgb}{0.000000, 0.000000, 0.000000}
\pgfsetfillcolor{diafillcolor}
\pgfsetfillopacity{1.000000}
% was here!!!
\definecolor{dialinecolor}{rgb}{0.000000, 0.000000, 0.000000}
\pgfsetstrokecolor{dialinecolor}
\pgfsetstrokeopacity{1.000000}
\draw (0.000000\du,-8.000000\du)--(17.000000\du,-8.000000\du);
}
\pgfsetlinewidth{0.100000\du}
\pgfsetdash{}{0pt}
\pgfsetbuttcap
{
\definecolor{diafillcolor}{rgb}{0.000000, 0.000000, 0.000000}
\pgfsetfillcolor{diafillcolor}
\pgfsetfillopacity{1.000000}
% was here!!!
\definecolor{dialinecolor}{rgb}{0.000000, 0.000000, 0.000000}
\pgfsetstrokecolor{dialinecolor}
\pgfsetstrokeopacity{1.000000}
\draw (17.000000\du,-8.000000\du)--(17.000000\du,-2.000000\du);
}
\pgfsetlinewidth{0.100000\du}
\pgfsetdash{}{0pt}
\pgfsetbuttcap
{
\definecolor{diafillcolor}{rgb}{0.000000, 0.000000, 0.000000}
\pgfsetfillcolor{diafillcolor}
\pgfsetfillopacity{1.000000}
% was here!!!
\definecolor{dialinecolor}{rgb}{0.000000, 0.000000, 0.000000}
\pgfsetstrokecolor{dialinecolor}
\pgfsetstrokeopacity{1.000000}
\draw (17.000000\du,-2.000000\du)--(0.000000\du,-2.000000\du);
}
\pgfsetlinewidth{0.100000\du}
\pgfsetdash{}{0pt}
\pgfsetbuttcap
{
\definecolor{diafillcolor}{rgb}{0.000000, 0.000000, 0.000000}
\pgfsetfillcolor{diafillcolor}
\pgfsetfillopacity{1.000000}
% was here!!!
\definecolor{dialinecolor}{rgb}{0.000000, 0.000000, 0.000000}
\pgfsetstrokecolor{dialinecolor}
\pgfsetstrokeopacity{1.000000}
\draw (0.000000\du,-10.000000\du)--(-1.000000\du,-9.000000\du);
}
\pgfsetlinewidth{0.100000\du}
\pgfsetdash{}{0pt}
\pgfsetbuttcap
{
\definecolor{diafillcolor}{rgb}{0.000000, 0.000000, 0.000000}
\pgfsetfillcolor{diafillcolor}
\pgfsetfillopacity{1.000000}
% was here!!!
\definecolor{dialinecolor}{rgb}{0.000000, 0.000000, 0.000000}
\pgfsetstrokecolor{dialinecolor}
\pgfsetstrokeopacity{1.000000}
\draw (0.000000\du,-8.000000\du)--(-1.000000\du,-7.000000\du);
}
\pgfsetlinewidth{0.100000\du}
\pgfsetdash{}{0pt}
\pgfsetbuttcap
{
\definecolor{diafillcolor}{rgb}{0.000000, 0.000000, 0.000000}
\pgfsetfillcolor{diafillcolor}
\pgfsetfillopacity{1.000000}
% was here!!!
\definecolor{dialinecolor}{rgb}{0.000000, 0.000000, 0.000000}
\pgfsetstrokecolor{dialinecolor}
\pgfsetstrokeopacity{1.000000}
\draw (0.000000\du,-6.000000\du)--(-1.000000\du,-5.000000\du);
}
\pgfsetlinewidth{0.100000\du}
\pgfsetdash{}{0pt}
\pgfsetbuttcap
{
\definecolor{diafillcolor}{rgb}{0.000000, 0.000000, 0.000000}
\pgfsetfillcolor{diafillcolor}
\pgfsetfillopacity{1.000000}
% was here!!!
\definecolor{dialinecolor}{rgb}{0.000000, 0.000000, 0.000000}
\pgfsetstrokecolor{dialinecolor}
\pgfsetstrokeopacity{1.000000}
\draw (0.000000\du,-4.000000\du)--(-1.000000\du,-3.000000\du);
}
\pgfsetlinewidth{0.100000\du}
\pgfsetdash{}{0pt}
\pgfsetbuttcap
{
\definecolor{diafillcolor}{rgb}{0.000000, 0.000000, 0.000000}
\pgfsetfillcolor{diafillcolor}
\pgfsetfillopacity{1.000000}
% was here!!!
\definecolor{dialinecolor}{rgb}{0.000000, 0.000000, 0.000000}
\pgfsetstrokecolor{dialinecolor}
\pgfsetstrokeopacity{1.000000}
\draw (0.000000\du,-2.000000\du)--(-1.000000\du,-1.000000\du);
}
\pgfsetlinewidth{0.100000\du}
\pgfsetdash{}{0pt}
\pgfsetbuttcap
{
\definecolor{diafillcolor}{rgb}{0.000000, 0.000000, 0.000000}
\pgfsetfillcolor{diafillcolor}
\pgfsetfillopacity{1.000000}
% was here!!!
\definecolor{dialinecolor}{rgb}{0.000000, 0.000000, 0.000000}
\pgfsetstrokecolor{dialinecolor}
\pgfsetstrokeopacity{1.000000}
\draw (0.000000\du,-9.000000\du)--(-1.000000\du,-8.000000\du);
}
\pgfsetlinewidth{0.100000\du}
\pgfsetdash{}{0pt}
\pgfsetbuttcap
{
\definecolor{diafillcolor}{rgb}{0.000000, 0.000000, 0.000000}
\pgfsetfillcolor{diafillcolor}
\pgfsetfillopacity{1.000000}
% was here!!!
\definecolor{dialinecolor}{rgb}{0.000000, 0.000000, 0.000000}
\pgfsetstrokecolor{dialinecolor}
\pgfsetstrokeopacity{1.000000}
\draw (0.000000\du,-7.000000\du)--(-1.000000\du,-6.000000\du);
}
\pgfsetlinewidth{0.100000\du}
\pgfsetdash{}{0pt}
\pgfsetbuttcap
{
\definecolor{diafillcolor}{rgb}{0.000000, 0.000000, 0.000000}
\pgfsetfillcolor{diafillcolor}
\pgfsetfillopacity{1.000000}
% was here!!!
\definecolor{dialinecolor}{rgb}{0.000000, 0.000000, 0.000000}
\pgfsetstrokecolor{dialinecolor}
\pgfsetstrokeopacity{1.000000}
\draw (0.000000\du,-5.000000\du)--(-1.000000\du,-4.000000\du);
}
\pgfsetlinewidth{0.100000\du}
\pgfsetdash{}{0pt}
\pgfsetbuttcap
{
\definecolor{diafillcolor}{rgb}{0.000000, 0.000000, 0.000000}
\pgfsetfillcolor{diafillcolor}
\pgfsetfillopacity{1.000000}
% was here!!!
\definecolor{dialinecolor}{rgb}{0.000000, 0.000000, 0.000000}
\pgfsetstrokecolor{dialinecolor}
\pgfsetstrokeopacity{1.000000}
\draw (0.000000\du,-3.000000\du)--(-1.000000\du,-2.000000\du);
}
\pgfsetlinewidth{0.100000\du}
\pgfsetdash{}{0pt}
\pgfsetbuttcap
{
\definecolor{diafillcolor}{rgb}{0.000000, 0.000000, 0.000000}
\pgfsetfillcolor{diafillcolor}
\pgfsetfillopacity{1.000000}
% was here!!!
\definecolor{dialinecolor}{rgb}{0.000000, 0.000000, 0.000000}
\pgfsetstrokecolor{dialinecolor}
\pgfsetstrokeopacity{1.000000}
\draw (0.000000\du,-1.000000\du)--(-1.000000\du,0.000000\du);
}
\pgfsetlinewidth{0.050000\du}
\pgfsetdash{}{0pt}
\pgfsetbuttcap
\definecolor{dialinecolor}{rgb}{0.000000, 0.000000, 0.000000}
\pgfsetstrokecolor{dialinecolor}
\pgfsetstrokeopacity{1.000000}
\pgfpathmoveto{\pgfpoint{17.192160\du}{-6.872710\du}}
\pgfpatharc{279}{270}{111.660170\du and 111.660170\du}
\pgfusepath{stroke}
\pgfsetlinewidth{0.050000\du}
\pgfsetdash{}{0pt}
\pgfsetbuttcap
\definecolor{dialinecolor}{rgb}{0.000000, 0.000000, 0.000000}
\pgfsetstrokecolor{dialinecolor}
\pgfsetstrokeopacity{1.000000}
\pgfpathmoveto{\pgfpoint{16.482919\du}{-0.929491\du}}
\pgfpatharc{279}{269}{94.994418\du and 94.994418\du}
\pgfusepath{stroke}
\pgfsetlinewidth{0.050000\du}
\pgfsetdash{}{0pt}
\pgfsetbuttcap
{
\definecolor{diafillcolor}{rgb}{0.000000, 0.000000, 0.000000}
\pgfsetfillcolor{diafillcolor}
\pgfsetfillopacity{1.000000}
% was here!!!
\definecolor{dialinecolor}{rgb}{0.000000, 0.000000, 0.000000}
\pgfsetstrokecolor{dialinecolor}
\pgfsetstrokeopacity{1.000000}
\draw (17.162252\du,-6.845827\du)--(16.453423\du,-0.936648\du);
}
\pgfsetlinewidth{0.100000\du}
\pgfsetdash{}{0pt}
\pgfsetbuttcap
{
\definecolor{diafillcolor}{rgb}{0.000000, 0.000000, 0.000000}
\pgfsetfillcolor{diafillcolor}
\pgfsetfillopacity{1.000000}
% was here!!!
\pgfsetarrowsend{to}
\definecolor{dialinecolor}{rgb}{0.000000, 0.000000, 0.000000}
\pgfsetstrokecolor{dialinecolor}
\pgfsetstrokeopacity{1.000000}
\draw (12.500000\du,-12.500000\du)--(12.500000\du,-8.000000\du);
}
% setfont left to latex
\definecolor{dialinecolor}{rgb}{0.000000, 0.000000, 0.000000}
\pgfsetstrokecolor{dialinecolor}
\pgfsetstrokeopacity{1.000000}
\definecolor{diafillcolor}{rgb}{0.000000, 0.000000, 0.000000}
\pgfsetfillcolor{diafillcolor}
\pgfsetfillopacity{1.000000}
\node[anchor=base west,inner sep=0pt,outer sep=0pt,color=dialinecolor] at (7.000000\du,-4.500000\du){$\Omega$};
% setfont left to latex
\definecolor{dialinecolor}{rgb}{0.000000, 0.000000, 0.000000}
\pgfsetstrokecolor{dialinecolor}
\pgfsetstrokeopacity{1.000000}
\definecolor{diafillcolor}{rgb}{0.000000, 0.000000, 0.000000}
\pgfsetfillcolor{diafillcolor}
\pgfsetfillopacity{1.000000}
\node[anchor=base west,inner sep=0pt,outer sep=0pt,color=dialinecolor] at (0.500000\du,-4.500000\du){$\Gamma_D$};
% setfont left to latex
\definecolor{dialinecolor}{rgb}{0.000000, 0.000000, 0.000000}
\pgfsetstrokecolor{dialinecolor}
\pgfsetstrokeopacity{1.000000}
\definecolor{diafillcolor}{rgb}{0.000000, 0.000000, 0.000000}
\pgfsetfillcolor{diafillcolor}
\pgfsetfillopacity{1.000000}
\node[anchor=base west,inner sep=0pt,outer sep=0pt,color=dialinecolor] at (6.081788\du,-8.609051\du){$\Gamma_N$};
% setfont left to latex
\definecolor{dialinecolor}{rgb}{0.000000, 0.000000, 0.000000}
\pgfsetstrokecolor{dialinecolor}
\pgfsetstrokeopacity{1.000000}
\definecolor{diafillcolor}{rgb}{0.000000, 0.000000, 0.000000}
\pgfsetfillcolor{diafillcolor}
\pgfsetfillopacity{1.000000}
\node[anchor=base west,inner sep=0pt,outer sep=0pt,color=dialinecolor] at (13.245364\du,-8.718101\du){$g$};
\end{tikzpicture}

	\caption{Beispiel für mögliche Nebenbedingungen}
	\end{figure}
\end{frame}

\subsubsection*{Differenzielle Formulierung}
%\begin{frame}
%	Satz von Gauss liefert für $\omega\subseteq\Omega$
%	\begin{align*}
%		0 &= \int_\omega f\dif x + \int_{\partial\omega}\sigma\cdot n\dif s \\
%		&= \int_{\omega}f\dif x+\int_{\partial\omega}\sum_j\sigma_{ij}n_je_i\dif s \\
%		&= \int_{\omega}f+\sum_j\partial_j\sigma_{ij}e_i\dif x
%	\end{align*}
%	
%	mit $n\colon\Omega\to\R^d$ die Flächennormale.
%	Wir erhalten die Gleichgewichtsbeziehung in differenzieller Form
%	\begin{align*}
%		-\diver \sigma \coloneqq -\sum_j\partial_j\sigma_{ij}e_i =f
%	\end{align*}
%\end{frame}

\begin{frame}
	Der Satz von Gauss liefert die Gleichgewichtsbeziehung in differenzieller Form
	\begin{align*}
		-\diver \sigma \coloneqq -\sum_j\partial_j\sigma_{ij}e_i =f
	\end{align*}
	Jetzt haben wir die Formulierung
	\begin{align*}
		\qquad\qquad-\diver\sigma &= f &&,\text{auf }\Omega\qquad\qquad\\
		u &= w &&,\text{auf }\Gamma_D \\
		\sigma n &= g &&,\text{auf }\Gamma_N
	\end{align*}
	Die Impulserhaltung liefert
	\begin{align*}
		\sigma_{ij}=\sigma_{ji}
	\end{align*}
\end{frame}

\subsection*{Materialgesetze}
\subsubsection*{Hooke-Tensor}
\begin{frame}
	\frametitle{Materialgesetze}
	Für ein linear-elastisches Material ist
	\begin{align*}
		\sigma_{ij}=\sum_{k,l}C_{ijkl}\e_{kl}
	\end{align*}
	mit Hooke-Tensor $C\colon\Omega\to\otimes_{i=1}^4\R^d$
\end{frame}

\subsubsection*{St. Venant-Kirchhoff Material}
\begin{frame}
	Für St. Venant-Kirchhoff-Materialen gilt
	\begin{align*}
		C_{ijkl}=\lambda\delta_{ij}\delta_{kl}+\mu(\delta_{ik}\delta_{jl}+\delta_{il}\delta_{jk})
	\end{align*}
	mit Lamé-Koeffizienten $\lambda$ und $\mu$ (Schubmodul). Es folgt
	\begin{align*}
		\sigma_{ij}=\sum_{k,l}C_{ijkl}\e_{kl}=\lambda\Tr(\e)\delta_{ij}+2\mu\e_{ij}
	\end{align*}
	Basierend auf $\lambda$, $\mu$ kann man weitere Materialparameter $K$ (Kompressionsmodul), $E$ (Elastizitätsmodul) und $\nu$ (Querkontraktion) definieren.
\end{frame}

\begin{frame}
	Für St. Venant-Kirchhoff-Materialien ergibt sich die Lamé-Differenzialgleichung
	\begin{align*}
		-\lambda\nabla\diver u-2\mu\diver\e(u)&=f &&,\text{auf }\Omega\qquad \\
		u &= w &&,\text{auf }\Gamma_D \\
		\sigma n &= g &&,\text{auf }\Gamma_N
	\end{align*}
\end{frame}


\subsection*{Variationsmethode, Energiebetrachtungen}
\begin{frame}
	\frametitle{Variationsmethode, Energiebetrachtungen}
	Wir definieren
	\begin{align*}
		a(u,v) \coloneqq\int_\Omega\sigma(u):\e(v)\dif x \coloneqq\int_\Omega \sum_{i,j}\sigma_{ij}(u)\e_{ij}(v)\dif x
	\end{align*}
	sowie
	\begin{align*}
		b(v)\coloneqq \int_\Omega f\cdot v\dif x+\int_{\Gamma_N} g\cdot v\dif x
	\end{align*}
	Dann lautet die potenzielle Energie
	\begin{align*}
		W(v)\coloneqq \frac{1}{2}a(v,v)-b(v)
	\end{align*}
\end{frame}

\begin{frame}
	Daraus erhält man eine neue Formulierung des Problems
	Finde $u\in V$, so dass
	\begin{align*}
		a(u,v) &= b(v) &&,\text{für alle }v\in V\text{ mit }v=0\text{ auf }\Gamma_D\qquad \\
		u &= w &&,\text{auf }\Gamma_D
	\end{align*}
\end{frame}

\section{Existenz und Eindeutigkeit von Lösungen}
\subsection*{Lax-Milgram-Lemma}
\begin{frame}
	\frametitle{Existenz und Eindeutigkeit von Lösungen}
	Unter welchen Vorraussetzungen ist $a$ eine stetige symmetrische bilineare Form? D.h.
	\begin{enumerate}
		\item $a$ ist bilinear und symmetrisch
		\item (Stetigkeit) es gibt ein $\alpha_1>0$, so dass für alle $v_1,v_2\in V$
		\begin{align*}
			a(v_1,v_2)\leq \alpha_1\norm{v_1}\norm{v_2}
		\end{align*}
		\item (Elliptizität) es gibt ein $\alpha_2>0$, so dass für alle $v\in V$
		\begin{align*}
			a(v,v)\geq\alpha_2\norm{v}^2
		\end{align*}
	\end{enumerate}
\end{frame}

\begin{frame}
	\begin{theorem}[Lax-Milgram Lemma, nach \cite{Cia-1997}]
		Sei $V$ ein Banach-Raum. Sei $b\in V^*$ und $a\colon V\times V\to\R$ eine stetige symmetrische elliptische bilineare Form ist.
		
		Dann hat das Problem $u\in V$ zu finden, so dass
		\begin{align*}
			a(u,v)=b(v)
		\end{align*}
		für alle $v\in V$ eine eindeutige Lösung. Dieses ist dann auch eindeutige Lösung des Problems $u\in V$ zu finden, so dass $u$ das Funktional
		\begin{align*}
			W(v)=\frac{1}{2}a(v,v)-b(v)
		\end{align*}
		minimiert.
	\end{theorem}
\end{frame}

\subsection*{Definitionen}
\begin{frame}[allowframebreaks]
	\frametitle{Ein paar Definitionen}
	$L^2(\Omega)$ bezeichnet die Menge aller Funktionen, deren Quadrat Lebesgue-integrierbar ist.
	Wir definieren $H^k(\Omega)$ für $k\in\N$ als die Menge aller $v\in L^2(\Omega)$, so dass für alle Multiindizes $\alpha$ mit $\abs{\alpha}\leq k$ die schwache Ableiung $\partial^\alpha v\in L^2(\Omega)$.
	Es wird durch
	\begin{align*}
		\inner{u,v}_{0,\Omega} = \int_\Omega uv\dif x
	\end{align*}
	Ein Skalarprodukt auf $L^2(\Omega)$ definiert. Durch
	\begin{align*}
		\inner{u,v}_{k,\Omega} = \sum_{\abs{\alpha}\leq k}\inner{\partial^\alpha u,\partial^\alpha v}_{0,\Omega}
	\end{align*}
	wird ein Scalarprodukt auf $H^k(\Omega)$ definiert. Dieses induziert die Norm $\norm{\cdot}_{k,\Omega}$ , wodurch $H^k(\Omega)$ zu einem Banachraum wird.
	
	\framebreak
	Wir definieren durch
	\begin{align*}
		\norm{v}_{-k,\Omega}\coloneqq\sup_{v\in H^k_0(\Omega)}\frac{\inner{u,v}_{0,\Omega}}{\norm{v}_{k,\Omega}}
	\end{align*}
	eine Norm auf $L^2(\Omega)$. Die Vervollständigung von $L^2(\Omega)$ bezeichnen wir mit $H^{-k}(\Omega)$. Dies ist eine Charakterisierung des Dualraums von $H^k(\Omega)$.
\end{frame}

\subsubsection*{Lemma von J.L. Lions}
\begin{frame}
	\begin{lemma}[Lemma von J.L. Lions, nach \cite{Duv-1976}]
	Sei $\Omega\subseteq\R^d$ ein Gebiet mit regulärem Rand und $v\colon\Omega\to\R$, so dass $v\in H^{-1}(\Omega)$ und $\partial_iv\in H^{-1}(\Omega)$ für alle $i$, dann folgt $v\in L^2(\Omega)$.
	\end{lemma}
\end{frame}

\subsection*{Kornsche Ungleichung ohne Randbedingungen}
\begin{frame}
	\begin{lemma}[Kornsche Ungleichung ohne Randbedingungen]
	(Kornsche Ungleichung ohne Randbedingungen, in \cite{Cia-1997} bewiesen nur für $d=3$)
	Sei $\Omega\subseteq\R^d$ ein Gebiet mit $C^1$ Rand (Lipschitz reicht wohl auch) und beschränkt, dann gibt es ein $c>0$, so dass für alle $v\in H^1(\Omega;\R^d)$ gilt
	\begin{align*}
		\norm{v}_{1,\Omega}\leq c\left(\abs{v}_{0,\Omega}^2+\abs{\e(v)}_{0,\Omega}^2\right)^{1/2}\eqqcolon c\norm{v}_K
	\end{align*}
	\end{lemma}
\end{frame}

\subsection*{Kornsche Ungleichung mit Randbedingungen}
\begin{frame}
	\begin{lemma}[Eine Charakterisierung von Starrkörperbewegungen]
		Sei $\Omega\subseteq\R^3$ offen und zusammenhängend, dann sind für $v\in H^1(\Omega;\R^3)$ äquivalent
		\begin{enumerate}
			\item $\e(v) = 0$
			\item Es gibt $a,b\in\R^3$, so dass $v(x)=a\times x+b$
		\end{enumerate}
	\end{lemma}
\end{frame}

\begin{frame}
	\begin{theorem}[Kornsche Ungleichung mit Randbedingungen]
		Sei $\Omega\subseteq\R^3$ mit $C^1$ Rand und beschränkt. Sei $\Gamma_D\subseteq\partial\Omega$ mit positivem Flächenmaß. Dann gibt es $c>0$, so dass für alle $v\in H^1(\Omega;\R^3)$ mit $v=0$ auf $\Gamma_D$
		\begin{align*}
			\norm{v}_{1,\Omega}\leq c\abs{\e(v)}_{0,\Omega}
		\end{align*}
	\end{theorem}
\end{frame}


\section{Das diskrete Problem}
\subsection*{Nodale Basis}
\begin{frame}
	\frametitle{Das diskrete Problem}
	Es sei $\cT_h$ eine reguläre Triangulierung von $\Omega$. Bezeichne $\varphi_i$ die nodale Basis in einer Dimension. Wir definieren die $d$-dimensionale nodale Basis
	\begin{align*}
		\vect{\phi_1\cdots\phi_{dn}}=\left(\varphi_1\ev_1\cdots\varphi_1\ev_d\quad\cdots\quad\varphi_n\ev_1\cdots\varphi_n\ev_d\right)
	\end{align*}
	Weiter setzen wir $\cS_h\coloneqq\Span\{\phi_i\}_i$.
\end{frame}

\begin{frame}
	\begin{figure}[h]
	\centering
	% Graphic for TeX using PGF
% Title: /mnt/12CCB7B3CCB79009/Filing/Education/University/Bonn/Courses/Bachelorarbeit/Resources/NodalBasis1.dia
% Creator: Dia v0.97+git
% CreationDate: Mon Apr  4 16:51:44 2022
% For: theo
% \usepackage{tikz}
% The following commands are not supported in PSTricks at present
% We define them conditionally, so when they are implemented,
% this pgf file will use them.
\ifx\du\undefined
  \newlength{\du}
\fi
\setlength{\du}{15\unitlength}
\begin{tikzpicture}[even odd rule]
\pgftransformxscale{0.600000}
\pgftransformyscale{-0.600000}
\definecolor{dialinecolor}{rgb}{0.000000, 0.000000, 0.000000}
\pgfsetstrokecolor{dialinecolor}
\pgfsetstrokeopacity{1.000000}
\definecolor{diafillcolor}{rgb}{1.000000, 1.000000, 1.000000}
\pgfsetfillcolor{diafillcolor}
\pgfsetfillopacity{1.000000}
\pgfsetlinewidth{0.100000\du}
\pgfsetdash{}{0pt}
\pgfsetbuttcap
{
\definecolor{diafillcolor}{rgb}{0.000000, 0.000000, 0.000000}
\pgfsetfillcolor{diafillcolor}
\pgfsetfillopacity{1.000000}
% was here!!!
\pgfsetarrowsend{to}
\definecolor{dialinecolor}{rgb}{0.000000, 0.000000, 0.000000}
\pgfsetstrokecolor{dialinecolor}
\pgfsetstrokeopacity{1.000000}
\draw (0.000000\du,0.000000\du)--(0.000000\du,-5.000000\du);
}
\pgfsetlinewidth{0.100000\du}
\pgfsetdash{}{0pt}
\pgfsetbuttcap
{
\definecolor{diafillcolor}{rgb}{0.000000, 0.000000, 0.000000}
\pgfsetfillcolor{diafillcolor}
\pgfsetfillopacity{1.000000}
% was here!!!
\pgfsetarrowsend{to}
\definecolor{dialinecolor}{rgb}{0.000000, 0.000000, 0.000000}
\pgfsetstrokecolor{dialinecolor}
\pgfsetstrokeopacity{1.000000}
\draw (0.000000\du,0.000000\du)--(5.000000\du,0.000000\du);
}
\pgfsetlinewidth{0.100000\du}
\pgfsetdash{}{0pt}
\pgfsetbuttcap
{
\definecolor{diafillcolor}{rgb}{0.000000, 0.000000, 0.000000}
\pgfsetfillcolor{diafillcolor}
\pgfsetfillopacity{1.000000}
% was here!!!
\pgfsetarrowsend{to}
\definecolor{dialinecolor}{rgb}{0.000000, 0.000000, 0.000000}
\pgfsetstrokecolor{dialinecolor}
\pgfsetstrokeopacity{1.000000}
\draw (0.000000\du,0.000000\du)--(1.748070\du,4.164270\du);
}
\pgfsetlinewidth{0.050000\du}
\pgfsetdash{}{0pt}
\pgfsetbuttcap
{
\definecolor{diafillcolor}{rgb}{0.000000, 0.000000, 0.000000}
\pgfsetfillcolor{diafillcolor}
\pgfsetfillopacity{1.000000}
% was here!!!
\definecolor{dialinecolor}{rgb}{0.000000, 0.000000, 0.000000}
\pgfsetstrokecolor{dialinecolor}
\pgfsetstrokeopacity{1.000000}
\draw (10.499989\du,2.449987\du)--(9.499989\du,5.449987\du);
}
\pgfsetlinewidth{0.050000\du}
\pgfsetdash{}{0pt}
\pgfsetbuttcap
{
\definecolor{diafillcolor}{rgb}{0.000000, 0.000000, 0.000000}
\pgfsetfillcolor{diafillcolor}
\pgfsetfillopacity{1.000000}
% was here!!!
\definecolor{dialinecolor}{rgb}{0.000000, 0.000000, 0.000000}
\pgfsetstrokecolor{dialinecolor}
\pgfsetstrokeopacity{1.000000}
\draw (10.499989\du,2.449987\du)--(7.499989\du,1.449987\du);
}
\pgfsetlinewidth{0.050000\du}
\pgfsetdash{}{0pt}
\pgfsetbuttcap
{
\definecolor{diafillcolor}{rgb}{0.000000, 0.000000, 0.000000}
\pgfsetfillcolor{diafillcolor}
\pgfsetfillopacity{1.000000}
% was here!!!
\definecolor{dialinecolor}{rgb}{0.000000, 0.000000, 0.000000}
\pgfsetstrokecolor{dialinecolor}
\pgfsetstrokeopacity{1.000000}
\draw (19.499989\du,-1.550013\du)--(13.499989\du,-0.550013\du);
}
\pgfsetlinewidth{0.050000\du}
\pgfsetdash{}{0pt}
\pgfsetbuttcap
{
\definecolor{diafillcolor}{rgb}{0.000000, 0.000000, 0.000000}
\pgfsetfillcolor{diafillcolor}
\pgfsetfillopacity{1.000000}
% was here!!!
\definecolor{dialinecolor}{rgb}{0.000000, 0.000000, 0.000000}
\pgfsetstrokecolor{dialinecolor}
\pgfsetstrokeopacity{1.000000}
\draw (16.499989\du,3.449987\du)--(13.499989\du,-0.550013\du);
}
\pgfsetlinewidth{0.050000\du}
\pgfsetdash{}{0pt}
\pgfsetbuttcap
{
\definecolor{diafillcolor}{rgb}{0.000000, 0.000000, 0.000000}
\pgfsetfillcolor{diafillcolor}
\pgfsetfillopacity{1.000000}
% was here!!!
\definecolor{dialinecolor}{rgb}{0.000000, 0.000000, 0.000000}
\pgfsetstrokecolor{dialinecolor}
\pgfsetstrokeopacity{1.000000}
\draw (10.499989\du,2.449987\du)--(13.499989\du,-0.550013\du);
}
\pgfsetlinewidth{0.050000\du}
\pgfsetdash{}{0pt}
\pgfsetbuttcap
{
\definecolor{diafillcolor}{rgb}{0.000000, 0.000000, 0.000000}
\pgfsetfillcolor{diafillcolor}
\pgfsetfillopacity{1.000000}
% was here!!!
\definecolor{dialinecolor}{rgb}{0.000000, 0.000000, 0.000000}
\pgfsetstrokecolor{dialinecolor}
\pgfsetstrokeopacity{1.000000}
\draw (8.499989\du,-3.550013\du)--(13.499989\du,-0.550013\du);
}
\pgfsetlinewidth{0.050000\du}
\pgfsetdash{}{0pt}
\pgfsetbuttcap
{
\definecolor{diafillcolor}{rgb}{0.000000, 0.000000, 0.000000}
\pgfsetfillcolor{diafillcolor}
\pgfsetfillopacity{1.000000}
% was here!!!
\definecolor{dialinecolor}{rgb}{0.000000, 0.000000, 0.000000}
\pgfsetstrokecolor{dialinecolor}
\pgfsetstrokeopacity{1.000000}
\draw (15.499989\du,-4.550013\du)--(13.499989\du,-0.550013\du);
}
\pgfsetlinewidth{0.050000\du}
\pgfsetdash{}{0pt}
\pgfsetbuttcap
{
\definecolor{diafillcolor}{rgb}{0.000000, 0.000000, 0.000000}
\pgfsetfillcolor{diafillcolor}
\pgfsetfillopacity{1.000000}
% was here!!!
\definecolor{dialinecolor}{rgb}{0.000000, 0.000000, 0.000000}
\pgfsetstrokecolor{dialinecolor}
\pgfsetstrokeopacity{1.000000}
\draw (19.499989\du,-1.550013\du)--(22.499989\du,-3.550013\du);
}
\pgfsetlinewidth{0.050000\du}
\pgfsetdash{}{0pt}
\pgfsetbuttcap
{
\definecolor{diafillcolor}{rgb}{0.000000, 0.000000, 0.000000}
\pgfsetfillcolor{diafillcolor}
\pgfsetfillopacity{1.000000}
% was here!!!
\definecolor{dialinecolor}{rgb}{0.000000, 0.000000, 0.000000}
\pgfsetstrokecolor{dialinecolor}
\pgfsetstrokeopacity{1.000000}
\draw (19.499989\du,-1.550013\du)--(22.499989\du,0.449987\du);
}
\pgfsetlinewidth{0.050000\du}
\pgfsetdash{}{0pt}
\pgfsetbuttcap
{
\definecolor{diafillcolor}{rgb}{0.000000, 0.000000, 0.000000}
\pgfsetfillcolor{diafillcolor}
\pgfsetfillopacity{1.000000}
% was here!!!
\definecolor{dialinecolor}{rgb}{0.000000, 0.000000, 0.000000}
\pgfsetstrokecolor{dialinecolor}
\pgfsetstrokeopacity{1.000000}
\draw (16.499989\du,3.449987\du)--(19.499989\du,2.449987\du);
}
\pgfsetlinewidth{0.050000\du}
\pgfsetdash{}{0pt}
\pgfsetbuttcap
{
\definecolor{diafillcolor}{rgb}{0.000000, 0.000000, 0.000000}
\pgfsetfillcolor{diafillcolor}
\pgfsetfillopacity{1.000000}
% was here!!!
\definecolor{dialinecolor}{rgb}{0.000000, 0.000000, 0.000000}
\pgfsetstrokecolor{dialinecolor}
\pgfsetstrokeopacity{1.000000}
\draw (16.499989\du,3.449987\du)--(18.499989\du,6.449987\du);
}
\pgfsetlinewidth{0.050000\du}
\pgfsetdash{}{0pt}
\pgfsetbuttcap
{
\definecolor{diafillcolor}{rgb}{0.000000, 0.000000, 0.000000}
\pgfsetfillcolor{diafillcolor}
\pgfsetfillopacity{1.000000}
% was here!!!
\definecolor{dialinecolor}{rgb}{0.000000, 0.000000, 0.000000}
\pgfsetstrokecolor{dialinecolor}
\pgfsetstrokeopacity{1.000000}
\draw (16.499989\du,3.449987\du)--(14.499989\du,6.449987\du);
}
\pgfsetlinewidth{0.050000\du}
\pgfsetdash{}{0pt}
\pgfsetbuttcap
{
\definecolor{diafillcolor}{rgb}{0.000000, 0.000000, 0.000000}
\pgfsetfillcolor{diafillcolor}
\pgfsetfillopacity{1.000000}
% was here!!!
\definecolor{dialinecolor}{rgb}{0.000000, 0.000000, 0.000000}
\pgfsetstrokecolor{dialinecolor}
\pgfsetstrokeopacity{1.000000}
\draw (8.499989\du,-3.550013\du)--(5.499989\du,-2.550013\du);
}
\pgfsetlinewidth{0.050000\du}
\pgfsetdash{}{0pt}
\pgfsetbuttcap
{
\definecolor{diafillcolor}{rgb}{0.000000, 0.000000, 0.000000}
\pgfsetfillcolor{diafillcolor}
\pgfsetfillopacity{1.000000}
% was here!!!
\definecolor{dialinecolor}{rgb}{0.000000, 0.000000, 0.000000}
\pgfsetstrokecolor{dialinecolor}
\pgfsetstrokeopacity{1.000000}
\draw (8.499989\du,-3.550013\du)--(7.499989\du,-6.550013\du);
}
\pgfsetlinewidth{0.050000\du}
\pgfsetdash{}{0pt}
\pgfsetbuttcap
{
\definecolor{diafillcolor}{rgb}{0.000000, 0.000000, 0.000000}
\pgfsetfillcolor{diafillcolor}
\pgfsetfillopacity{1.000000}
% was here!!!
\definecolor{dialinecolor}{rgb}{0.000000, 0.000000, 0.000000}
\pgfsetstrokecolor{dialinecolor}
\pgfsetstrokeopacity{1.000000}
\draw (8.499989\du,-3.550013\du)--(11.499989\du,-6.550013\du);
}
\pgfsetlinewidth{0.050000\du}
\pgfsetdash{}{0pt}
\pgfsetbuttcap
{
\definecolor{diafillcolor}{rgb}{0.000000, 0.000000, 0.000000}
\pgfsetfillcolor{diafillcolor}
\pgfsetfillopacity{1.000000}
% was here!!!
\definecolor{dialinecolor}{rgb}{0.000000, 0.000000, 0.000000}
\pgfsetstrokecolor{dialinecolor}
\pgfsetstrokeopacity{1.000000}
\draw (15.499989\du,-4.550013\du)--(12.499989\du,-6.550013\du);
}
\pgfsetlinewidth{0.050000\du}
\pgfsetdash{}{0pt}
\pgfsetbuttcap
{
\definecolor{diafillcolor}{rgb}{0.000000, 0.000000, 0.000000}
\pgfsetfillcolor{diafillcolor}
\pgfsetfillopacity{1.000000}
% was here!!!
\definecolor{dialinecolor}{rgb}{0.000000, 0.000000, 0.000000}
\pgfsetstrokecolor{dialinecolor}
\pgfsetstrokeopacity{1.000000}
\draw (15.499989\du,-4.550013\du)--(19.499989\du,-4.550013\du);
}
\pgfsetlinewidth{0.050000\du}
\pgfsetdash{}{0pt}
\pgfsetbuttcap
{
\definecolor{diafillcolor}{rgb}{0.000000, 0.000000, 0.000000}
\pgfsetfillcolor{diafillcolor}
\pgfsetfillopacity{1.000000}
% was here!!!
\definecolor{dialinecolor}{rgb}{0.000000, 0.000000, 0.000000}
\pgfsetstrokecolor{dialinecolor}
\pgfsetstrokeopacity{1.000000}
\draw (15.499989\du,-4.550013\du)--(16.499989\du,-6.550013\du);
}
\pgfsetlinewidth{0.100000\du}
\pgfsetdash{}{0pt}
\pgfsetbuttcap
{
\definecolor{diafillcolor}{rgb}{0.000000, 0.000000, 0.000000}
\pgfsetfillcolor{diafillcolor}
\pgfsetfillopacity{1.000000}
% was here!!!
\definecolor{dialinecolor}{rgb}{0.000000, 0.000000, 0.000000}
\pgfsetstrokecolor{dialinecolor}
\pgfsetstrokeopacity{1.000000}
\draw (10.499989\du,2.449987\du)--(13.499989\du,-3.550013\du);
}
\pgfsetlinewidth{0.100000\du}
\pgfsetdash{}{0pt}
\pgfsetbuttcap
{
\definecolor{diafillcolor}{rgb}{0.000000, 0.000000, 0.000000}
\pgfsetfillcolor{diafillcolor}
\pgfsetfillopacity{1.000000}
% was here!!!
\definecolor{dialinecolor}{rgb}{0.000000, 0.000000, 0.000000}
\pgfsetstrokecolor{dialinecolor}
\pgfsetstrokeopacity{1.000000}
\draw (13.499989\du,-3.550013\du)--(16.499989\du,3.449987\du);
}
\pgfsetlinewidth{0.100000\du}
\pgfsetdash{}{0pt}
\pgfsetbuttcap
{
\definecolor{diafillcolor}{rgb}{0.000000, 0.000000, 0.000000}
\pgfsetfillcolor{diafillcolor}
\pgfsetfillopacity{1.000000}
% was here!!!
\definecolor{dialinecolor}{rgb}{0.000000, 0.000000, 0.000000}
\pgfsetstrokecolor{dialinecolor}
\pgfsetstrokeopacity{1.000000}
\draw (13.499989\du,-3.550013\du)--(19.499989\du,-1.550013\du);
}
\pgfsetlinewidth{0.100000\du}
\pgfsetdash{}{0pt}
\pgfsetbuttcap
{
\definecolor{diafillcolor}{rgb}{0.000000, 0.000000, 0.000000}
\pgfsetfillcolor{diafillcolor}
\pgfsetfillopacity{1.000000}
% was here!!!
\definecolor{dialinecolor}{rgb}{0.000000, 0.000000, 0.000000}
\pgfsetstrokecolor{dialinecolor}
\pgfsetstrokeopacity{1.000000}
\draw (13.499989\du,-3.550013\du)--(15.499989\du,-4.550013\du);
}
\pgfsetlinewidth{0.100000\du}
\pgfsetdash{}{0pt}
\pgfsetbuttcap
{
\definecolor{diafillcolor}{rgb}{0.000000, 0.000000, 0.000000}
\pgfsetfillcolor{diafillcolor}
\pgfsetfillopacity{1.000000}
% was here!!!
\definecolor{dialinecolor}{rgb}{0.000000, 0.000000, 0.000000}
\pgfsetstrokecolor{dialinecolor}
\pgfsetstrokeopacity{1.000000}
\draw (13.499989\du,-3.550013\du)--(8.499989\du,-3.550013\du);
}
\pgfsetlinewidth{0.100000\du}
\pgfsetdash{}{0pt}
\pgfsetbuttcap
{
\definecolor{diafillcolor}{rgb}{0.000000, 0.000000, 0.000000}
\pgfsetfillcolor{diafillcolor}
\pgfsetfillopacity{1.000000}
% was here!!!
\definecolor{dialinecolor}{rgb}{0.000000, 0.000000, 0.000000}
\pgfsetstrokecolor{dialinecolor}
\pgfsetstrokeopacity{1.000000}
\draw (8.499989\du,-3.550013\du)--(9.999989\du,-5.050013\du);
}
\pgfsetlinewidth{0.100000\du}
\pgfsetdash{}{0pt}
\pgfsetbuttcap
{
\definecolor{diafillcolor}{rgb}{0.000000, 0.000000, 0.000000}
\pgfsetfillcolor{diafillcolor}
\pgfsetfillopacity{1.000000}
% was here!!!
\definecolor{dialinecolor}{rgb}{0.000000, 0.000000, 0.000000}
\pgfsetstrokecolor{dialinecolor}
\pgfsetstrokeopacity{1.000000}
\draw (8.499989\du,-3.550013\du)--(7.999989\du,-5.050013\du);
}
\pgfsetlinewidth{0.100000\du}
\pgfsetdash{}{0pt}
\pgfsetbuttcap
{
\definecolor{diafillcolor}{rgb}{0.000000, 0.000000, 0.000000}
\pgfsetfillcolor{diafillcolor}
\pgfsetfillopacity{1.000000}
% was here!!!
\definecolor{dialinecolor}{rgb}{0.000000, 0.000000, 0.000000}
\pgfsetstrokecolor{dialinecolor}
\pgfsetstrokeopacity{1.000000}
\draw (8.499989\du,-3.550013\du)--(6.999989\du,-3.050013\du);
}
\pgfsetlinewidth{0.100000\du}
\pgfsetdash{}{0pt}
\pgfsetbuttcap
{
\definecolor{diafillcolor}{rgb}{0.000000, 0.000000, 0.000000}
\pgfsetfillcolor{diafillcolor}
\pgfsetfillopacity{1.000000}
% was here!!!
\definecolor{dialinecolor}{rgb}{0.000000, 0.000000, 0.000000}
\pgfsetstrokecolor{dialinecolor}
\pgfsetstrokeopacity{1.000000}
\draw (10.499989\du,2.449987\du)--(8.999989\du,1.949987\du);
}
\pgfsetlinewidth{0.100000\du}
\pgfsetdash{}{0pt}
\pgfsetbuttcap
{
\definecolor{diafillcolor}{rgb}{0.000000, 0.000000, 0.000000}
\pgfsetfillcolor{diafillcolor}
\pgfsetfillopacity{1.000000}
% was here!!!
\definecolor{dialinecolor}{rgb}{0.000000, 0.000000, 0.000000}
\pgfsetstrokecolor{dialinecolor}
\pgfsetstrokeopacity{1.000000}
\draw (10.499989\du,2.449987\du)--(9.999989\du,3.949987\du);
}
\pgfsetlinewidth{0.100000\du}
\pgfsetdash{}{0pt}
\pgfsetbuttcap
{
\definecolor{diafillcolor}{rgb}{0.000000, 0.000000, 0.000000}
\pgfsetfillcolor{diafillcolor}
\pgfsetfillopacity{1.000000}
% was here!!!
\definecolor{dialinecolor}{rgb}{0.000000, 0.000000, 0.000000}
\pgfsetstrokecolor{dialinecolor}
\pgfsetstrokeopacity{1.000000}
\draw (16.499989\du,3.449987\du)--(15.499989\du,4.949987\du);
}
\pgfsetlinewidth{0.100000\du}
\pgfsetdash{}{0pt}
\pgfsetbuttcap
{
\definecolor{diafillcolor}{rgb}{0.000000, 0.000000, 0.000000}
\pgfsetfillcolor{diafillcolor}
\pgfsetfillopacity{1.000000}
% was here!!!
\definecolor{dialinecolor}{rgb}{0.000000, 0.000000, 0.000000}
\pgfsetstrokecolor{dialinecolor}
\pgfsetstrokeopacity{1.000000}
\draw (16.499989\du,3.449987\du)--(17.499989\du,4.949987\du);
}
\pgfsetlinewidth{0.100000\du}
\pgfsetdash{}{0pt}
\pgfsetbuttcap
{
\definecolor{diafillcolor}{rgb}{0.000000, 0.000000, 0.000000}
\pgfsetfillcolor{diafillcolor}
\pgfsetfillopacity{1.000000}
% was here!!!
\definecolor{dialinecolor}{rgb}{0.000000, 0.000000, 0.000000}
\pgfsetstrokecolor{dialinecolor}
\pgfsetstrokeopacity{1.000000}
\draw (16.499989\du,3.449987\du)--(17.999989\du,2.949987\du);
}
\pgfsetlinewidth{0.100000\du}
\pgfsetdash{}{0pt}
\pgfsetbuttcap
{
\definecolor{diafillcolor}{rgb}{0.000000, 0.000000, 0.000000}
\pgfsetfillcolor{diafillcolor}
\pgfsetfillopacity{1.000000}
% was here!!!
\definecolor{dialinecolor}{rgb}{0.000000, 0.000000, 0.000000}
\pgfsetstrokecolor{dialinecolor}
\pgfsetstrokeopacity{1.000000}
\draw (19.499989\du,-1.550013\du)--(20.999989\du,-0.550013\du);
}
\pgfsetlinewidth{0.100000\du}
\pgfsetdash{}{0pt}
\pgfsetbuttcap
{
\definecolor{diafillcolor}{rgb}{0.000000, 0.000000, 0.000000}
\pgfsetfillcolor{diafillcolor}
\pgfsetfillopacity{1.000000}
% was here!!!
\definecolor{dialinecolor}{rgb}{0.000000, 0.000000, 0.000000}
\pgfsetstrokecolor{dialinecolor}
\pgfsetstrokeopacity{1.000000}
\draw (19.499989\du,-1.550013\du)--(20.999989\du,-2.550013\du);
}
\pgfsetlinewidth{0.100000\du}
\pgfsetdash{}{0pt}
\pgfsetbuttcap
{
\definecolor{diafillcolor}{rgb}{0.000000, 0.000000, 0.000000}
\pgfsetfillcolor{diafillcolor}
\pgfsetfillopacity{1.000000}
% was here!!!
\definecolor{dialinecolor}{rgb}{0.000000, 0.000000, 0.000000}
\pgfsetstrokecolor{dialinecolor}
\pgfsetstrokeopacity{1.000000}
\draw (15.499989\du,-4.550013\du)--(17.499989\du,-4.550013\du);
}
\pgfsetlinewidth{0.100000\du}
\pgfsetdash{}{0pt}
\pgfsetbuttcap
{
\definecolor{diafillcolor}{rgb}{0.000000, 0.000000, 0.000000}
\pgfsetfillcolor{diafillcolor}
\pgfsetfillopacity{1.000000}
% was here!!!
\definecolor{dialinecolor}{rgb}{0.000000, 0.000000, 0.000000}
\pgfsetstrokecolor{dialinecolor}
\pgfsetstrokeopacity{1.000000}
\draw (15.499989\du,-4.550013\du)--(15.999989\du,-5.550013\du);
}
\pgfsetlinewidth{0.100000\du}
\pgfsetdash{}{0pt}
\pgfsetbuttcap
{
\definecolor{diafillcolor}{rgb}{0.000000, 0.000000, 0.000000}
\pgfsetfillcolor{diafillcolor}
\pgfsetfillopacity{1.000000}
% was here!!!
\definecolor{dialinecolor}{rgb}{0.000000, 0.000000, 0.000000}
\pgfsetstrokecolor{dialinecolor}
\pgfsetstrokeopacity{1.000000}
\draw (15.499989\du,-4.550013\du)--(13.999989\du,-5.550013\du);
}
\pgfsetlinewidth{0.100000\du}
\pgfsetdash{}{0pt}
\pgfsetbuttcap
{
\definecolor{diafillcolor}{rgb}{0.000000, 0.000000, 0.000000}
\pgfsetfillcolor{diafillcolor}
\pgfsetfillopacity{1.000000}
% was here!!!
\definecolor{dialinecolor}{rgb}{0.000000, 0.000000, 0.000000}
\pgfsetstrokecolor{dialinecolor}
\pgfsetstrokeopacity{1.000000}
\draw (8.499989\du,-3.550013\du)--(10.499989\du,2.449987\du);
}
\pgfsetlinewidth{0.100000\du}
\pgfsetdash{}{0pt}
\pgfsetbuttcap
{
\definecolor{diafillcolor}{rgb}{0.000000, 0.000000, 0.000000}
\pgfsetfillcolor{diafillcolor}
\pgfsetfillopacity{1.000000}
% was here!!!
\definecolor{dialinecolor}{rgb}{0.000000, 0.000000, 0.000000}
\pgfsetstrokecolor{dialinecolor}
\pgfsetstrokeopacity{1.000000}
\draw (10.499989\du,2.449987\du)--(16.499989\du,3.449987\du);
}
\pgfsetlinewidth{0.100000\du}
\pgfsetdash{}{0pt}
\pgfsetbuttcap
{
\definecolor{diafillcolor}{rgb}{0.000000, 0.000000, 0.000000}
\pgfsetfillcolor{diafillcolor}
\pgfsetfillopacity{1.000000}
% was here!!!
\definecolor{dialinecolor}{rgb}{0.000000, 0.000000, 0.000000}
\pgfsetstrokecolor{dialinecolor}
\pgfsetstrokeopacity{1.000000}
\draw (16.499989\du,3.449987\du)--(19.499989\du,-1.550013\du);
}
\pgfsetlinewidth{0.100000\du}
\pgfsetdash{}{0pt}
\pgfsetbuttcap
{
\definecolor{diafillcolor}{rgb}{0.000000, 0.000000, 0.000000}
\pgfsetfillcolor{diafillcolor}
\pgfsetfillopacity{1.000000}
% was here!!!
\definecolor{dialinecolor}{rgb}{0.000000, 0.000000, 0.000000}
\pgfsetstrokecolor{dialinecolor}
\pgfsetstrokeopacity{1.000000}
\draw (19.499989\du,-1.550013\du)--(15.499989\du,-4.550013\du);
}
\pgfsetlinewidth{0.100000\du}
\pgfsetdash{}{0pt}
\pgfsetbuttcap
{
\definecolor{diafillcolor}{rgb}{0.000000, 0.000000, 0.000000}
\pgfsetfillcolor{diafillcolor}
\pgfsetfillopacity{1.000000}
% was here!!!
\definecolor{dialinecolor}{rgb}{0.000000, 0.000000, 0.000000}
\pgfsetstrokecolor{dialinecolor}
\pgfsetstrokeopacity{1.000000}
\draw (15.499989\du,-4.550013\du)--(8.499989\du,-3.550013\du);
}
\pgfsetlinewidth{0.050000\du}
\pgfsetdash{{\pgflinewidth}{0.200000\du}}{0cm}
\pgfsetbuttcap
{
\definecolor{diafillcolor}{rgb}{0.000000, 0.000000, 0.000000}
\pgfsetfillcolor{diafillcolor}
\pgfsetfillopacity{1.000000}
% was here!!!
\definecolor{dialinecolor}{rgb}{0.000000, 0.000000, 0.000000}
\pgfsetstrokecolor{dialinecolor}
\pgfsetstrokeopacity{1.000000}
\draw (13.499989\du,-3.550013\du)--(13.499989\du,-0.550013\du);
}
% setfont left to latex
\definecolor{dialinecolor}{rgb}{0.000000, 0.000000, 0.000000}
\pgfsetstrokecolor{dialinecolor}
\pgfsetstrokeopacity{1.000000}
\definecolor{diafillcolor}{rgb}{0.000000, 0.000000, 0.000000}
\pgfsetfillcolor{diafillcolor}
\pgfsetfillopacity{1.000000}
\node[anchor=base west,inner sep=0pt,outer sep=0pt,color=dialinecolor] at (5.448063\du,0.514277\du){$x_1$};
% setfont left to latex
\definecolor{dialinecolor}{rgb}{0.000000, 0.000000, 0.000000}
\pgfsetstrokecolor{dialinecolor}
\pgfsetstrokeopacity{1.000000}
\definecolor{diafillcolor}{rgb}{0.000000, 0.000000, 0.000000}
\pgfsetfillcolor{diafillcolor}
\pgfsetfillopacity{1.000000}
\node[anchor=base west,inner sep=0pt,outer sep=0pt,color=dialinecolor] at (1.948070\du,5.164268\du){$x_2$};
% setfont left to latex
\definecolor{dialinecolor}{rgb}{0.000000, 0.000000, 0.000000}
\pgfsetstrokecolor{dialinecolor}
\pgfsetstrokeopacity{1.000000}
\definecolor{diafillcolor}{rgb}{0.000000, 0.000000, 0.000000}
\pgfsetfillcolor{diafillcolor}
\pgfsetfillopacity{1.000000}
\node[anchor=base west,inner sep=0pt,outer sep=0pt,color=dialinecolor] at (-0.451925\du,-5.635710\du){$\varphi_k(x)$};
% setfont left to latex
\definecolor{dialinecolor}{rgb}{0.000000, 0.000000, 0.000000}
\pgfsetstrokecolor{dialinecolor}
\pgfsetstrokeopacity{1.000000}
\definecolor{diafillcolor}{rgb}{0.000000, 0.000000, 0.000000}
\pgfsetfillcolor{diafillcolor}
\pgfsetfillopacity{1.000000}
\node[anchor=base west,inner sep=0pt,outer sep=0pt,color=dialinecolor] at (12.5\du,1.3\du){$x^{k}$};
\end{tikzpicture}

	\caption{Ein Element einer nodalen Basis in 2D}
	\end{figure}
\end{frame}

\subsection*{Diskretisiertes Problem}
\begin{frame}
	Das diskrete Problem lautet dann:
Finde $u_h\in\cS_h$, so dass
	\begin{align*}
		a(v_h,u_h) &= b(v_h) &&,\text{für alle }v_h\in\cS_h\text{ mit }v_h=0\text{ auf }\Gamma_D \\
		u_h &= w_h &&,\text{auf }\Gamma_D
	\end{align*}
\end{frame}

%\begin{frame}
%	Wir schreiben $u_h=\sum_i\hu_i\phi_i$ und $v_h=\sum_j\hv_j\phi_j$ und es folgt
%	\begin{align*}
%		a(v_h,u_h)
%		= a(\sum_{j}\hv_j\phi_j,\sum_{i}\hu_i\phi_i)
%		= \sum_{i,j}\underbrace{a(\phi_j,\phi_i)}_{\eqqcolon \hA_{ji}}\hu_i\hv_j
%		= \sum_{i,j}\hA_{ji}\hu_i\hv_j
%		= \hv^\top \hA\hu
%	\end{align*}
%	sowie
%	\begin{align*}
%		b(v_h)
%		= b(\sum_j \hv_j\phi_j)
%		= \sum \underbrace{b(\phi_j)}_{\eqqcolon \hb_j}\hv_j
%		= \hv^\top \hb
%	\end{align*}
%\end{frame}
%
%\begin{frame}
%	Damit lautet unser neues Problem: Finde $\hu=(\hu_i)_i\in\R^{nd}$, so dass für alle $\hv=(\hv_i)_i\in\R^{nd}$ gilt
%	\begin{align*}
%		\hv^\top\hA\hu=\hv^\top\hb
%	\end{align*}
%	Oder: Finde $\hu\in\R^{nd}$, so dass
%	\begin{align*}
%		\hA\hu = \hb
%	\end{align*}
%	mit Steifheits-Matrix $\hA_{ij} = a(\phi_j,\phi_i)$.
%\end{frame}

\begin{frame}
	Indem man $u_h=\sum_i\hu_i\phi_i$ setzt gelangt man zu einer neuen Formulierung des diskreten Problems: Finde $\hu\in\R^{nd}$, so dass
	\begin{align*}
		\hA\hu = \hb
	\end{align*}
	mit Steifheits-Matrix $\hA_{ij} = a(\phi_j,\phi_i)$ und Load-Vektor $\hb_i = b(\phi_i)$.
\end{frame}

%\subsection*{Implementierung des Hooke Tensors}
%\begin{frame}[allowframebreaks]
%	\frametitle{Implementierung des Hooke Tensors}
%	
%	Wir wenden die Voigt representation für $d=2$
%	\begin{align*}
%		\gamma(\e) = \vect{\e_{11} \\ \e_{22} \\ 2\e_{12}}
%	\end{align*}
%	beziehungsweise für $d=3$ an
%	\begin{align*}
%		\gamma(\e) = \vect{\e_{11} \\ \e_{22} \\ \e_{22} \\ 2\e_{12} \\ 2\e_{13} \\ 2\e_{23}}
%	\end{align*}
%	
%	\framebreak
%	Damit lässt sich die Relation 
%	\begin{align*}
%		\sigma_{ij}=\sum_{k,l}C_{ijkl}\e_{kl}=\lambda\Tr(\e)\delta_{ij}+2\mu\e_{ij}
%	\end{align*}
%	schreiben als
%	\begin{align*}
%		\vect{\sigma_{11} \\ \sigma_{22} \\ \sigma_{33} \\ \sigma_{12} \\ \sigma_{13} \\ \sigma_{23}}
%		= \underbrace{\begin{pmatrix}
%			\lambda+2\mu & \lambda & \lambda & & & \\
%			\lambda & \lambda+2\mu & \lambda & & & \\
%			\lambda & \lambda & \lambda+2\mu & & & \\
%			& & & \mu & & \\
%			& & & & \mu & \\
%			& & & & & \mu \\
%		\end{pmatrix}}_{\eqqcolon \hC}
%		\vect{\e_{11} \\ \e_{22} \\ \e_{33} \\ 2\e_{12} \\ 2\e_{13} \\ 2\e_{23}}
%	\end{align*}
%	also $\gamma(\sigma) = \hC\gamma(\e)$
%\end{frame}

\section{A Posteriori Fehlerschätzer}
\begin{frame}
	\frametitle{Residuale Fehlerschätzer}
	Gegeben sei die Lösung $u$ des Problems und $u_h$ des diskretisierten Problems. Wir definieren
	\begin{itemize}
		\item die flächenbezogenen Residuen $R_T\coloneqq f+\diver\sigma(u_h)$
		\item die kantenbezogenen Sprünge
		\begin{align*}
			R_E = \begin{cases}
				\left[\![\sigma(u_h)\cdot n\right]\!] &,\text{falls }E\subseteq\bigcup_T\partial T\setminus\partial\Omega \\
				0 &,\text{falls }E\subseteq\Gamma_D \\
				g-\sigma(u_h) &,\text{falls }E\subseteq\Gamma_N
			\end{cases}
		\end{align*}
%		mit
%		\begin{align*}
%		\left[\![\sigma(u_h)\cdot n\right]\!]
%		&\coloneqq\sigma(u_h)\cdot n_{T_1}\big\Vert_{\partial T_1}+\sigma(u_h)\cdot n_{T_2} \big\Vert_{\partial T_2}\\
%		&= \sigma(u_h)n_{T_1}\big\Vert_{\partial T_1}-\sigma(u_h)n_{T_1}\big\Vert_{\partial T_2}\cdot 
%		\end{align*}
%		wobei $\cdot\Vert_{\partial T}\colon H^1(T;\R^d)\to L^2(\partial T;\R^d)$ den Trace operator bezeichnet.
		\item einen lokalen Fehlerschätzer
		\begin{align*}
			\eta_{R,T}\coloneqq\Big(h_T^2\norm{R_T}_{0,T}^2+\frac{1}{2}\sum_{E\in\partial T}h_E\norm{R_E}_{0,E}^2\Big)^{1/2}
		\end{align*}
		\item einen globalen Fehlerschätzer
		\begin{align*}
			\eta_{R}\coloneqq\Big(\sum_{T\in\cT_h}h_T^2\norm{R_T}_{0,T}^2+\sum_{E\in\cE_h}h_E\norm{R_E}_{0,E}^2\Big)^{1/2}
		\end{align*}
	\end{itemize}
\end{frame}

\begin{frame}
	\begin{theorem}[untere Schranke des residualen Schätzers, nach \cite{Bra-2007,Ban-2003}]
		Sei $\cT_h$ eine quasiuniforme Triangulierung von $\Omega$. Dann gibt es ein $c>0$, so dass für den Fehler $e\coloneqq u-u_h$ gilt
		\begin{align*}
			\norm{e}_{1,\Omega}\leq c\eta_R
		\end{align*}
	\end{theorem}
\end{frame}

\begin{frame}
	\begin{theorem}[obere Schranke des residualen Schätzers, nach \cite{Bra-2007,Ban-2003}]
		Sei $\cT_h$ eine quasiuniforme Triangulierung von $\Omega$. Dann gibt es ein $c>0$, so dass
		\begin{align*}
			\eta_{R,T}\leq c\Big(\norm{e}_{1,\omega_T}^2+\sum_{T'\subseteq\omega_T}h_{T'}^2\norm{f-P_hf}_{0,T'}^2\Big)^{1/2}
		\end{align*}
	\end{theorem}
\end{frame}

\subsection*{Fehlerschätzer durch Mittelung}
\begin{frame}
	\frametitle{Fehlerschätzung durch Mittelung, nach \cite{Alb-2002}}
	Wir setzen im Folgenden $\tisigma_h=\sigma(\tie_h)$ und $\sigma_h=\sigma(\e_h)$.
	Man definiert eine stetige Approximation $\tisigma_h$ an $\sigma_h$, indem man an den Knoten den Wert von $\tilde{\sigma}_h$ auf das Mittel von $\sigma_h$ der angrenzenden $T\in\cT_h$ setzt und dieses dann linear interpoliert. Dies liefert dann die Fehlerschätzer
	\begin{align*}
		\eta_{M1,T}\coloneqq\norm{\tisigma_h-\sigma_h}_{0,T}
	\end{align*}
	oder alternativ
	\begin{align*}
		\eta_{M2,T}\coloneqq\int_T(\tisigma_h-\sigma_h):(\tie_h-\e_h)\dif x
	\end{align*}
%	
%	Anschauliche Bedeutung in 1d
\end{frame}

\begin{frame}
	\begin{center}
		\Large{{Danke für die Aufmerksamkeit.}}
	\end{center}
\end{frame}

\begin{frame}
	\begin{center}
		\Large{{Fragen?}}
	\end{center}
\end{frame}


\section{Quellen}
\begin{frame}[allowframebreaks]
	\frametitle{Quellen}
	\begin{thebibliography}{0000}
		
		\bibitem{Alb-2002}
		 Alberty, J., C. Carstensen, S. A. Funken, and R. Klose.
		 "Matlab Implementation of the Finite Element Method in Elasticity."
		 {\em Computing 69}, no. 3 (2002): 239-263. 
		
		\bibitem{Ban-2003}
		 Bangerth, Wolfgang, and Rolf Rannacher.
		 {\em Adaptive Finite Element Methods for Differential Equations.}
		 Basel [u.a.]: Birkhäuser, 2003. S.130f.
		
		\bibitem{Bra-2007}
		 Braess, Dietrich.
		 {\em Finite Elemente: Theorie, Schnelle Löser Und Anwendungen in Der Elastizitätstheorie.}
		 4., überarb. und erw. Aufl.
		 Berlin [u.a.]: Springer, 2007.
		
		\bibitem{Cia-1988}
		 Ciarlet, Philippe G.
		 {\em Studies in Mathematics and Its Applications. Mathematical Elasticity. 1, Three-dimensional Elasticity.}
		 Amsterdam [u.a.]: North-Holland, 1988.
		
		\bibitem{Cia-1997}
		 Ciarlet, Philippe G.
		 {\em Studies in Mathematics and Its Applications. Mathematical Elasticity. 2, Theory of Plates.}
		 Amsterdam [u.a.]: North-Holland, 1997. 
		
		\bibitem{Duv-1976}
		 Lions, Jacques Louis, and Georges Duvaut.
		 {\em Inequalities in Mechanics and Physics.}
		 Berlin, Heidelberg: Springer, 1976. 
		
		\bibitem{Kik-1988}
		 Kikuchi, Noboru, and John Tinsley Oden.
		 {\em Contact Problems in Elasticity: A Study of Variational Inequalities and Finite Element Methods.}
		 Philadelphia: SIAM, 1988. 
		
		\bibitem{Lif-1959}
		 Lifshitz, Evgenii Mikhailovich, and Lev Davidovich Landau.
		 {\em Course of Theoretical Physics.}
		 Pergamon, 1959.
		
		\bibitem{Nei-2004}
		 Neittaanmäki, Pekka, and Sergey R. Repin.
		 {\em Reliable Methods for Computer Simulation: Error Control and Posteriori Estimates}.
		 Oxford: Elsevier Science \& Technology, 2004.

	\end{thebibliography}
\end{frame}

\frame[plain]

\end{document}
