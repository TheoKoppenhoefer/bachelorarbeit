\documentclass[11pt, a4paper, german]{article}

\usepackage[german]{babel}
\usepackage{BA_Titelseite}


%Namen des Verfassers der Arbeit
\author{X Y}
%Geburtsdatum des Verfassers
\geburtsdatum{1. April 1900}
%Gebortsort des Verfassers
\geburtsort{Bonn}
%Datum der Abgabe der Arbeit
\date{\today}

%Name des Betreuers
% z.B.: Prof. Dr. Peter Koepke
\betreuer{Betreuer: Prof. Dr. X Y}
%Name des Zweitgutachters
\zweitgutachter{Zweitgutachter: Prof. Dr. X Y}
%Name des Instituts an dem der Betreuer der Arbeit tätig ist.
%z.B.: Mathematisches Institut
\institut{Institut XYZ}
%\institut{Mathematisches Institut}
%\institut{Institut f\"ur Angewandte Mathematik}
%\institut{Institut f\"ur Numerische Simulation}
%\institut{Forschungsinstitut f\"ur Diskrete Mathematik}
%Titel der Bachelorarbeit
\title{This is only an example}
%Do not change!
\ausarbeitungstyp{Bachelorarbeit Mathematik}

\usepackage{amsmath}
\usepackage{amssymb}
\usepackage{amsthm}
\theoremstyle{plain}
\newtheorem{theorem}{Satz}
\newtheorem{proposition}[theorem]{Proposition}
\newtheorem{lemma}[theorem]{Lemma}
\newtheorem{corollary}[theorem]{Korollar}
\theoremstyle{remark}
\newtheorem{remark}[theorem]{Bemerkung}
\theoremstyle{definition}
\newtheorem{definition}[theorem]{Definition}
\numberwithin{equation}{section}
\numberwithin{theorem}{section}

\usepackage{mathtools}
\DeclarePairedDelimiter\abs{\lvert}{\rvert}
\DeclarePairedDelimiter\norm{\lVert}{\rVert}

\usepackage[T1]{fontenc}
\usepackage[utf8]{inputenc}
\usepackage[hidelinks]{hyperref}
\usepackage[style=alphabetic]{biblatex}
\addbibresource{Literatur.bib}

\begin{document}

\maketitle

\section*{Zusammenfassung}
Wir führen die Benutzung einiger \LaTeX-Befehle und -Pakete anhand von Beispielen vor.
Die Verwendung der hier empfohlenen Lösungen ist, mit Ausnahme des Titelblatts, nicht verpflichtend, sollte Ihre Arbeit aber erheblich vereinfachen.

Dieses Dokument spiegelt den Stand von 2021 wider; wenn Sie es im Jahr 2030 oder später noch lesen, sollten Sie sich vermutlich nach Alternativen für hier vorgestellte Lösungen umschauen.

\begin{flushright}
---Pavel Zorin-Kranich
\end{flushright}

\clearpage
\tableofcontents

\clearpage
\section{Einrichtung Ihres \TeX-Systems}
\subsection{Installation}
Sie benötigen eine aktuelle Version von \TeX Live.
Alle gängigen Linux-Distri\-bu\-ti\-onen stellen dafür Pakete bereit.
Für andere Betriebssysteme ist \TeX Live unter
\url{https://www.tug.org/texlive/}
erhältlich.
Bei der Installation von \TeX Live lassen sich verschiedene Bestandteile auswählen; wenn Sie unsicher sind, was Sie brauchen, können Sie alles installieren.

\subsection{Dokumentation}
Die Distribution \TeX Live beinhaltet Dokumentation für alle darin enthaltenen Pakete.
Eine kurze Einführung in \LaTeX\ sehen Sie, wenn Sie in der Kommandozeile
\begin{verbatim}
$ texdoc lshort-german
\end{verbatim}
eingeben.
Die Dokumentation für ein Paket können Sie sich mit
\begin{verbatim}
$ texdoc PAKETNAME
\end{verbatim}
anzeigen lassen, also z.B.\ \verb|texdoc tikz|.

Wenn Sie ausgefallene mathematische Symbole suchen (die in der o.g.\ Kurzbeschreibung nicht auftauchen), werden Sie hier fündig:
\begin{verbatim}
$ texdoc comprehensive
\end{verbatim}
Es gibt viele Online-Ressourcen zu \LaTeX.
Es kann aber schwierig sein, aktuelle Informationen von veralteten zu unterscheiden.
Die aktivste, und deshalb aktuellste, Ressource ist zur Zeit \url{https://tex.stackexchange.com/}

Es gibt auch Bücher über \LaTeX, wie z.B.\ \cite{More_Math_into_Latex}.

\subsection{Editor}
Wenn Sie noch keinen Lieblingseditor haben, empfehle ich aktuell \emph{Visual Studio Code} mit den Erweiterungen \emph{LaTeX Workshop} (IDE-Funktionen) und \emph{LTeX} (Rechtschreibprüfung).
Spezialisierte \LaTeX-Editoren bieten dem gegenüber keine echten Vorteile, und diesen Editor können Sie auch für viele andere Zwecke einsetzen.

Richten Sie ihren Editor so ein, dass er \verb|pdflatex| und \verb|biber| (und nicht \verb|latex|+\verb|dvipdf| bzw.\ \verb|bibtex|) benutzt.
Qualitäts-Editoren sollten automatisch erkennen, was Sie benötigen.
Wenn Sie keinen Editor mit integriertem Viewer verwenden, sollten Sie noch \verb|pdfsync| manuell einrichten.

\clearpage
\section{Paketempfehlungen}
\label{sec:pakete}
Ich stelle hier kurz die (nach meiner Meinung) für Mathematik wichtigsten Pakete vor.

\subsection{\texttt{amsmath}}
Stellt nummerierte Umgebungen für Gleichungen bereit, wie z.B.
\begin{equation}
\label{eq:1}
F = ma.
\end{equation}
Wenn Sie auf das Display \eqref{eq:1} später verweisen wollen, müssen Sie ihm mit dem Befehl \verb|\label| einen Namen zuweisen, auf den Sie dann mit dem Befehl \verb|\eqref| verweisen können.
Ähnliches (mit \verb|\ref| statt \verb|\eqref|) ist für Kapitel, wie z.B.\ Kapitel~\ref{sec:pakete}, und Sätze, wie z.B.\ Satz~\ref{thm:Cauchy-Schwarz}, möglich.

\subsection{\texttt{amssymb}}
Definiert wichtige Symbole, wie z.B. $\mathbb{R}$ mit dem Befehl \verb|\mathbb{R}|.

\subsection{\texttt{amsthm}}
Stellt den Befehl \verb|\newtheorem| bereit, der benutzt wird um Umgebungen für Sätze, Definitionen, und Ähnliches zu definieren.
Hier einige Beispiele.

\begin{definition}
Ein \emph{metrischer Raum} besteht aus einer Menge $M$ und einer Abbildung $d : M \times M \to \mathbb{R}_{\geq 0}$, genannt \emph{Metrik}, sodass für alle $x,y,z\in M$ gilt
\begin{enumerate}
\item $x=y \iff d(x,y)=0$,
\item $d(x,y)=d(y,x)$ (Symmetrie),
\item $d(x,z) \leq d(x,y)+d(y,z)$ (Subadditivität oder Dreiecksungleichung).
\end{enumerate}
\end{definition}

\begin{theorem}[Cauchy--Schwarz-Ungleichung]
\label{thm:Cauchy-Schwarz}
Sei $H$ ein Hilbertraum und $v,w \in H$.
Dann gilt
\[
\abs{ \langle v,w \rangle }
\leq
\norm{v} \norm{w}.
\]
\end{theorem}
\begin{proof}
Wenn der Beweis mit einer Gleichung endet, können Sie \verb|\qedhere| verwenden, um das Symbol für Beweisende auf der gleichen Zeile anzuzeigen.
\end{proof}
\begin{remark}
Diese Bemerkung ist nicht nötig um den Rest des Textes zu verstehen.
\end{remark}

\begin{proof}[Beweis von Satz~\ref{thm:Cauchy-Schwarz}]
Wenn zwischen der Aussage eines Satzes und dem Beweis noch etwas steht, sollte man explizit sagen, was man gerade beweisen will.
\end{proof}

\subsection{\texttt{fontenc}}
\TeX\ ist ein sehr altes System und benutzt standardmäßig Rasterschriftarten.
Dieses Paket mit der Option \verb|T1| ersetzt sie durch Vektorschriftarten.

\subsection{\texttt{inputenc}}
\TeX\ ist ein sehr altes System und akzeptiert standardmäßig nur ASCII-Eingabe.
Dieses Paket mit der Option \verb|utf8| erlaubt Unicode-Eingabe, bietet also insbesondere Unterstützung für Umlaute und andere Akzente.

\subsection{\texttt{hyperref}}
Erzeugt klickbare Hyperlinks innerhalb des Dokuments bei Benutzung von \verb|\ref| und  \verb|\eqref| und ins Internet mit \verb|\url| und \verb|\href|.

\subsection{\texttt{biblatex}}
Formatiert die Bibliographie aus Daten, die in einer \verb|.bib|-Datei vorliegen.
Die Datei muss mit \verb|\addbibresource| eingebunden werden.
Die Bibliographie wird mit dem Befehl \verb|\printbibliography| gesetzt.
Es ist ein Zwischenschritt mit dem Programm \verb|biber| notwendig, der von Qualitäts-Editoren automatisch ausgeführt wird; Details entnehmen Sie der Paket-Dokumentation.

Bibliographische Daten in dem passenden Format finden Sie in \emph{Math\-Sci\-Net} (\url{https://mathscinet.ams.org/}) oder \emph{zbMATH} (\url{https://zbmath.org/}).
Wenn Sie eine Veröffentlichung, die Sie zitieren wollen, dort nicht finden, wie z.B.\ \cite{arxiv:1507.02515}, können Sie den Eintrag von Hand ausfüllen, oder Sie benutzen einen \verb|.bib|-Manager, wie z.B.\ \emph{KBibTeX} (dieser hat den Vorteil einer eingebauten MathSciNet-Suche).

Wenn Sie aus einem Buch zitieren, benennen Sie immer den relevanten Abschnitt, also z.B.\ \cite[Satz III.2.2]{MR1232192} oder zumindest \cite[p.\ 107]{MR1232192} (die Seitenzahl-Variante ist schlechter, da eher von der Ausgabe abhängig), und nicht einfach \cite{MR1232192}.
In diesem Beispiel ersparen Sie \emph{jedem} Leser Ihrer Arbeit den Aufwand, in einem 700-seitigen Buch etwas Bestimmtes zu suchen.

\clearpage
\section{Optionale Pakete}
Die hier vorgestellten Pakete decken Aufgaben ab, die sich nicht in jeder Arbeit stellen, aber häufig genug vorkommen dürften, um sie an dieser Stelle erwähnenswert zu machen.

\subsection{\texttt{mathtools}}
Vereinfacht häufige Formatierungsaufgaben, wie z.B.\ Einrückung von Gleichungen in der \verb|align|-Umgebung mit dem Befehl \verb|\MoveEqLeft| oder hübsche Formatierung von Klammern, Beträgen, Normen, inneren Produkten, etc.\ mit dem Befehl \verb|\DeclarePairedDelimiter|.

\subsection{\texttt{tikz}}
Das umfangreiche Paket zur Erstellung von zweidimensionalen Grafiken aller Art.

\subsection{\texttt{tikz-cd}}
Eine auf \verb|tikz| aufbauende Lösung zur Erstellung von kommutativen Diagrammen.
Komfortabler und flexibler als alle früheren Lösungen.

Mit \emph{Quiver} (\url{https://q.uiver.app/}) existiert eine WYSIWYG-Ober\-fläche zur Erstellung von \verb|tikz-cd|-Diagrammen.

\subsection{\texttt{graphicx}}
Einbindung externer Grafiken.
Empfehlenswerte Formate sind PDF (für Vektorgrafiken), PNG (für Rastergrafiken), oder JPG (für Fotos).

\subsection{\texttt{booktabs}}
Schönere Tabellen.

\clearpage
\addcontentsline{toc}{section}{Literatur}
\printbibliography

\end{document}
