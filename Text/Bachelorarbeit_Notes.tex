
\documentclass{scrartcl}
% alternative article or report


%%%%% PACKAGES

% small tweaks and nicer typography
\usepackage{microtype}

% changes language to German
% gives proper date, and correct hyphenation
\usepackage[ngerman]{babel}

% basic math stuff
\usepackage{mathtools}
\usepackage{amssymb}
\usepackage{amsthm}
%\usepackage{tikz-cd}

% tikz
\usepackage{tikz}
\usetikzlibrary{positioning}
\usetikzlibrary{babel}
\tikzset{>=stealth}

\newcommand{\tikzmark}[3][]{\tikz[remember picture,baseline] \node [anchor=base,#1](#2) {$#3$};}

% title page
\usepackage{BA_Titelseite}

%%%%% CONFIGURATION

% prevents automatic line breaks inside of equations
% since it looks bad
\binoppenalty = \maxdimen
\relpenalty   = \maxdimen

% theorem-like environments
\newcounter{everything}
\newtheorem{corollary}[everything]{Korollar}
\newtheorem{lemma}[everything]{Lemma}
\newtheorem{proposition}[everything]{Proposition}
\newtheorem{theorem}[everything]{Satz}


%%%%% CUSTOM COMMANDS

% real numbers via \R
% complex numbers via \C
% general field via \K
\def\C{\mathbb{C}}
\def\R{\mathbb{R}}
\def\K{\mathbb{K}}
\def\Q{\mathbb{Q}}
\def\Z{\mathbb{Z}}
\def\N{\mathbb{N}}
\def\H{\mathbb{H}}
\def\e{\varepsilon}
\def\ev{e}

\newcommand{\cA}{\mathcal{A}}
\newcommand{\cB}{\mathcal{B}}
\newcommand{\cC}{\mathcal{C}}
\newcommand{\cD}{\mathcal{D}}
\newcommand{\cE}{\mathcal{E}}
\newcommand{\cF}{\mathcal{F}}
\newcommand{\cG}{\mathcal{G}}
\newcommand{\cH}{\mathcal{H}}
\newcommand{\cI}{\mathcal{I}}
\newcommand{\cJ}{\mathcal{J}}
\newcommand{\cK}{\mathcal{K}}
\newcommand{\cL}{\mathcal{L}}
\newcommand{\cM}{\mathcal{M}}
\newcommand{\cN}{\mathcal{N}}
\newcommand{\cO}{\mathcal{O}}
\newcommand{\cP}{\mathcal{P}}
\newcommand{\cQ}{\mathcal{Q}}
\newcommand{\cR}{\mathcal{R}}
\newcommand{\cS}{\mathcal{S}}
\newcommand{\cT}{\mathcal{T}}
\newcommand{\cU}{\mathcal{U}}
\newcommand{\cV}{\mathcal{V}}
\newcommand{\cW}{\mathcal{W}}
\newcommand{\cX}{\mathcal{X}}
\newcommand{\cY}{\mathcal{Y}}
\newcommand{\cZ}{\mathcal{Z}}

\newcommand{\bA}{\mathbb{A}}
\newcommand{\bB}{\mathbb{B}}
\newcommand{\bC}{\mathbb{C}}
\newcommand{\bD}{\mathbb{D}}
\newcommand{\bE}{\mathbb{E}}
\newcommand{\bF}{\mathbb{F}}
\newcommand{\bG}{\mathbb{G}}
\newcommand{\bH}{\mathbb{H}}
\newcommand{\bI}{\mathbb{I}}
\newcommand{\bJ}{\mathbb{J}}
\newcommand{\bK}{\mathbb{K}}
\newcommand{\bL}{\mathbb{L}}
\newcommand{\bM}{\mathbb{M}}
\newcommand{\bN}{\mathbb{N}}
\newcommand{\bO}{\mathbb{O}}
\newcommand{\bP}{\mathbb{P}}
\newcommand{\bQ}{\mathbb{Q}}
\newcommand{\bR}{\mathbb{R}}
\newcommand{\bS}{\mathbb{S}}
\newcommand{\bT}{\mathbb{T}}
\newcommand{\bU}{\mathbb{U}}
\newcommand{\bV}{\mathbb{V}}
\newcommand{\bW}{\mathbb{W}}
\newcommand{\bX}{\mathbb{X}}
\newcommand{\bY}{\mathbb{Y}}
\newcommand{\bZ}{\mathbb{Z}}

\newcommand{\hu}{\hat{u}}
\newcommand{\hv}{\hat{v}}
\newcommand{\hA}{\hat{A}}
\newcommand{\hC}{\hat{C}}
\newcommand{\hR}{\hat{R}}
\newcommand{\hV}{\hat{V}}
\newcommand{\hw}{\hat{w}}
\newcommand{\hb}{\hat{b}}


\newcommand{\tiS}{\tilde{S}}
\newcommand{\tiu}{\tilde{u}}
\newcommand{\tih}{\tilde{h}}
\newcommand{\tie}{\tilde{\varepsilon}}
\newcommand{\tisigma}{\tilde{\sigma}}

\newcommand{\dif}[1]{\,\mathrm{d} #1}
\newcommand{\norm}[1]{\lVert #1 \rVert}
\newcommand{\bnorm}[1]{\left\lVert #1\right\rVert}
\newcommand{\vii}[2]{\ensuremath{\begin{bmatrix}#1 \\ #2 \end{bmatrix}}}
\newcommand{\mii}[4]{\ensuremath{\begin{bmatrix}#1&#2 \\ #3&#4 \end{bmatrix}}}
\newcommand{\mc}[1]{\mathcal{#1}}


%%%%%%%%%%    Math operators    %%%%%%%%%%%%%%%%%%%%%%%%%%%

\DeclareMathOperator{\Id}{Id}             % identity morphism
% \DeclareMathOperator{\ker}{ker}           % kernel
\DeclareMathOperator{\rg}{rg}             % image
\DeclareMathOperator{\defekt}{def}             % defect
\DeclareMathOperator{\im}{im}             % image
\DeclareMathOperator{\Hom}{Hom}           % homomorphisms
\DeclareMathOperator{\End}{End}           % endomorphisms
\DeclareMathOperator{\Span}{Span}         % linear span
\DeclareMathOperator{\grad}{\nabla}         % gradient
\DeclareMathOperator{\Tr}{Tr}       	  % trace
\DeclareMathOperator{\diver}{Div}			% divergence
\DeclareMathOperator{\supp}{supp}			% support
\DeclareMathOperator{\dist}{dist}			% distance
\DeclareMathOperator{\inter}{int}			% interiour

% inner product (scalar product) via \inner{v, w}
% norm via \norm{x}
% absolute value via \abs{x}
% use the star-version for automatic scaling
\DeclarePairedDelimiter{\abs}{|}{|}
\DeclarePairedDelimiter{\inner}{\langle}{\rangle}

% \vect{ x // y // z } for a column vector with entries x, y, z
% similarly for larger vectors
% in this code:  1 = number of arguments
%               #1 = first argument
\newcommand{\vect}[1]{\begin{bmatrix} #1 \end{bmatrix}}


% \conj{z} for complex conjugation
\newcommand{\conj}{\overline}



%%%%% TITLE PAGE %%%%%%%%%%%%%%%%%%%%%%%%%%

%Namen des Verfassers der Arbeit
\author{Theo Koppenhöfer}
%Geburtsdatum des Verfassers
\geburtsdatum{9. November 2000}
%Gebortsort des Verfassers
\geburtsort{Heidelberg}
%Datum der Abgabe der Arbeit
\date{\today}

%Name des Betreuers
% z.B.: Prof. Dr. Peter Koepke
\betreuer{Betreuer: Prof. Dr. Joscha Gedicke}
%Name des Zweitgutachters
\zweitgutachter{Zweitgutachter: Prof. Dr. X Y}
%Name des Instituts an dem der Betreuer der Arbeit tätig ist.
%z.B.: Mathematisches Institut
%\institut{Mathematisches Institut}
%\institut{Institut f\"ur Angewandte Mathematik}
\institut{Institut f\"ur Numerische Simulation}
%\institut{Forschungsinstitut f\"ur Diskrete Mathematik}
%Titel der Bachelorarbeit
\title{Adaptive Finite Elemente für Lineare Elastizität}
%Do not change!
\ausarbeitungstyp{Bachelorarbeit Mathematik}


%%%%% The content starts here %%%%%%%%%%%%%


\begin{document}

\maketitle

\tableofcontents

\newpage

Zunächst beginnen wir mit einer Liste an Bezeichnungen.
Zur Notation: Im folgenden werden bei Summen der Übersichtlichkeit halber die Grenzen weggelassen. Die Menge der Indizes, über die sich die Summe erstreckt wird als maximal angenommen.


\section{Einleitung}
\subsubsection*{Verzerrung, Geometrische Betrachtungen}

Da der Körper sich im Kräftegleichgewicht befindet, verschwindet auch das gesamte Drehmoment in einem geeigneten Gebiet $\omega\subseteq\Omega$ für $d=3$ und es folgt wieder mit dem Satz von Gauss
\begin{align*}
	0 &= \int_\omega \left(x\times f\right)_i\dif x + \int_{\partial\omega}\left(x\times\sigma n\right)_i\dif x \\
	&=\int_\omega \left(x\times f\right)_i\dif x + \int_{\partial\omega}\sum_{k,j,l}\epsilon_{ijk}x_j\sigma_{kl}n_l\dif x \\ \\
	&\tikzmark{gaussdrehimp}{=}\int_\omega \left(x\times f\right)_i\dif x + \int_{\omega}\sum_{j,k,l}\partial_l\epsilon_{ijk}x_j\sigma_{kl}n_l\dif x \\
	&=\int_\omega \left(x\times f\right)_i\dif x + \int_{\omega}\sum_{j,k,l}\left(\epsilon_{ijk}x_j\partial_l\sigma_{kl}+\epsilon_{ijk}\delta_{lj}\sigma_{kl}\right)\dif x \\
	&=\int_\omega \left(x\times f\right)_i\dif x + \int_{\omega}\left(x\times \diver\sigma \right)_i\dif x+\int_\omega\sum_{j,k}\epsilon_{ijk}\sigma_{kj}\dif x \\
	&= \int_\omega\sum_{j,k}\epsilon_{ijk}\sigma_{kj}\dif x
\end{align*}
\begin{tikzpicture}[remember picture, overlay, node distance = 0.6cm]
	\node[,text width=5cm] (gaussdrehimpdescr) [above right= of gaussdrehimp]{Satz von Gauss};
	\draw[,->,thick] (gaussdrehimpdescr) to [in=90,out=180] (gaussdrehimp);
\end{tikzpicture}


Hierbei bezeichnet $\epsilon_{ijk}$ den $\epsilon$-Tensor, der dadurch definiert ist, dass er in $ijk$ total antisymmetrisch ist und mit $\epsilon_{123}=1$ normiert ist. Da dies für alle $\omega\subseteq\Omega$ regulär genug gilt, folgt
\begin{align*}
	0=\sum_{j,k}\epsilon_{ijk}\sigma_{kj}
\end{align*}
Also
\begin{align*}
	0 &= \sum_{j,k}\epsilon_{1jk}\sigma_{kj} = \sigma_{32}-\sigma_{23} \\
	0 &= \sum_{j,k}\epsilon_{2jk}\sigma_{kj} = \sigma_{13}-\sigma_{31} \\
	0 &= \sum_{j,k}\epsilon_{3jk}\sigma_{kj} = \sigma_{21}-\sigma_{12} \\
\end{align*}
und damit $\sigma_{ij}=\sigma_{ji}$, also ist $\sigma$ symmetrisch.

TODO:
Verzerrungsdichte (strain density)
Nansons Formel:
\begin{align*}
	Mu\times Mv=(\det M)M^{-T}(u\times v)
\end{align*}

\subsubsection*{Stress}

TODO: Konsistente Notation, int f auf andere Seite der Gleichung.

TODO: Man kann auch gleitende Randbedingungen haben, mit $M$ und $\Gamma_N$ und $\Gamma_D$ nicht notwendigerweise disjunkt.

TODO: muss nicht noch eine Annahme in $i$ getroffen werden?


Für ein Cauchy-Material (Cauchy elastic material, check name) gilt das materialabhängige Gesetz
\begin{align*}
	t_{ij}(x) = \sigma(x,\partial v)
\end{align*}
(this makes notationally absolutely no sense)


\subsubsection*{Materialgesetze}
\begin{align*}
	f
	&= -\sum_{i,j}\partial_j\sigma_{ij}e_i \\
	&= -\sum_{i,j}\partial_j\left(\lambda\Tr(\e)\delta_{ij}+2\mu\e_{ij}\right)e_i \\
	&= -\sum_{i,j,k}\partial_j\left(\lambda\e_{kk}\delta_{ij}+2\mu\e_{ij}\right)e_i \\
	&= -\sum_{i,j,k}\left(\lambda(\partial_j\e_{kk})\delta_{ij}+2\mu\partial_j\e_{ij}\right)e_i \\
	&= -\sum_{i,j,k}\left(\lambda(\partial_i\partial_ku)+2\mu\partial_j\e_{ij}\right)e_i \\
	&= -\lambda\grad\diver u-2\mu \diver\e(u)
\end{align*}
Was uns zur Lamé Differenzialgleichung führt
\begin{align*}
	-\lambda\grad\diver u-2\mu\diver\e(u) &= f &&\text{ auf }\Omega \\
	u&= w &&\text{ auf }\Gamma_D \\
	\sigma(u)\, n &= g &&\text{ auf }\Gamma_N \\
\end{align*}
TODO: Wie folgen die Symmetriebeziehungen genau?

\subsubsection*{St. Venant-Kirchhoff-Material}


TODO: (es folgt in Lip eine argumentation mit Energie und minimum, woraus folgt, dass K>0 und mu>0).
Es gilt in der physik $K\geq0$ und $\mu\geq0$, woraus folgt


Beziehungsweise erhalten wir die DGL (nach \cite{Alb-2002})
\begin{align*}
	(\lambda+\mu)(\grad\diver u)^\top + \mu\Delta u &= f \\
	M\cdot u&= w \text{ auf }\Gamma_D\\
	\sigma(u)\cdot n &= g \\
\end{align*}


TODO: Im folgenden Fokus weniger auf $K$ und mehr auf $E$, $\nu$, $\lambda$ und $\mu$ -> versuche $E$ aus Experimenten zu entfernen

Wegen
\begin{align*}
	\Tr(\e) &= \Tr\left(\frac{1}{2}\left(\nabla u+\nabla u^\top\right)\right) \\
	&= \frac{1}{2}\left(\Tr\left(\nabla u\right)+\Tr\left(\nabla u^\top\right)\right) \\
	&= \Tr\left(\nabla u\right) \\
\end{align*}
können wir $\Tr(\e)$ als Volumenänderung der Deformation interpretieren (TODO: Das ist nicht offensichtlich). Falls $\e$ spurfrei ist, also $\Tr(\e)=0$, hat die von $\e$ hervorgebrachte Deformation keine Volumenänderung zur Folge und wir bezeichnen sie als reine Scherung.
Wir können die lineare Verzerrung in einen reinen Scheranteil (engl. pure shear) und einen reinen Kompressionsanteil (pure compression) (nach \cite{Lif-1959}) zerlegen
\begin{align*}
	\e=(\e-\frac{1}{d}\Tr(\e)\Id)+\frac{1}{d}\Tr(\e)\Id
\end{align*}
Es folgt für den Cauchy Tensor
\begin{align*}
	\sigma &= \lambda\Tr(\e)\Id+2\mu\e \\
	&= (\lambda+\frac{2\mu}{d})\Tr(\e)\Id+2\mu(\e-\frac{1}{d}\Tr(\e)\Id) \\
	&= K\Tr(\e)\Id+2\mu(\e-\frac{1}{d}\Tr(\e)\Id)
\end{align*}
Wobei wir das Kompressionsmdul $K$ ('Modulus of (hydrstatic) compression / rigidity under compression') durch 
\begin{align*}
	K = \lambda+\frac{2\mu}{d}
\end{align*}
definiert haben. Ist $\e$ ein pure compression, so ist
\begin{align*}
	\sigma = K\Tr(\e)\Id
\end{align*}
Komprimieren wir das Material in der $x_1$-Richtung, sagen wir mit $\e_{ij}=-\delta_{i1}\delta_{j1}$, so müssen die induzierten Oberflächenkräfte in diesselbe Richtung zeigen, also muss auch $\sigma_{11}<0$. Damit folgt aber für ein physikalisches Material $K>0$. Analog (oder wie in Lip mit der Energie) kann man argumentieren, dass $\mu>0$. Hiermit folgt
\begin{align*}
	\sum_{i,j}\sigma_{ij}\e_{ij}\geq0
\end{align*}

In \cite{Lif-1959}, S.13 wird die Bedeutung von $\nu$ und $E$ diskutiert.

Anwendung von $\Tr$ liefert (vgl. mit Braess)
\begin{align*}
	&\Tr(\sigma)=d\cdot\lambda\Tr(\e)+2\mu\Tr(\e)=(d\cdot\lambda+2\mu)\Tr(\e) \\
	&\implies \Tr(\e) = \frac{1}{d\cdot\lambda+2\mu}\Tr(\sigma)
\end{align*}
und wir erhalten
\begin{align*}
	\e
	&=\frac{1}{2\mu}\left(\sigma-\lambda\Tr(\e)\Id\right) \\
	&=\frac{1}{2\mu}\left(\sigma-\frac{\lambda}{d\cdot\lambda+2\mu}\Tr(\sigma)\Id\right) \\
	&= \frac{1}{2\mu}\sigma-\frac{\lambda}{2\mu(d\cdot\lambda+2\mu)}\Tr(\sigma)\Id \\
	&= \frac{1+\nu}{E}\sigma-\frac{\nu}{E}\Tr(\sigma)\Id
\end{align*}
wobei wir das Elastizitätsmodul $E$ (Youngs modulus) durch
\begin{align*}
	\frac{1}{E}=\frac{\lambda+\mu}{\mu(d\cdot\lambda+2\mu)}
\end{align*}
und die Querkontraktion $\nu\in[-1,\frac{1}{2}]$ (nach \cite{Lif-1959}, TODO: stimmt das?) (Possons ration) durch 
\begin{align*}
	\nu = \frac{\lambda}{2(\lambda+\mu)}	
\end{align*}
definieren.
Üben wir Druck an einem Ende eines Stabes in $x_1$-Richtung (dieses Beispiel ist aus Lifschitz und wahrscheinlich auch die Motivation für diese Definition) aus, also $\sigma_{ij} = \delta_{1i}\delta_{1j}$, so haben wir
\begin{align*}
	\e_{ij} = \frac{1+\nu}{E}\delta_{i1}\delta_{j1}-\frac{\nu}{E}\Tr(\sigma)\delta_{ij}
	= \frac{1+\nu}{E}\delta_{i1}\delta_{j1}-\frac{\nu}{E}\delta_{ij}
\end{align*}
Also hängt die Verformung in $x_1$ Richtung
\begin{align*}
	\partial_1u_1 = \e_{11}=\frac{1}{E}
\end{align*}
reziprok von $E$ ab und die Verformung in $x_i$-Richtung mit $i\neq1$
\begin{align*}
	\partial_iu_i = \e_{ii}=-\frac{\nu}{E}=-\nu\partial_1u_1
\end{align*}
über den Parameter $\nu$ von der Verformung in $x_1$-Richtung ab.

TODO: Füge ein Bild hinzu, gehe auf Beispiel ein



Man kann auch die Kräfte betrachten, die durch eine Volumenänderung verursacht werden.
Das folgende Beispiel ist aus Lifschitz.
Nun betrachten wir einen Körper, der hydrostatische Kompression erfährt. Der Körper erfährt also von allen Seiten eine Kraft mit derselben Stärke $p$. Der Stresstensor ist also gegeben durch
\begin{align*}
	\sigma^K _{ij} = p\delta_{ij}
\end{align*}
Wir wollen nun einen Ausdruck für $p$ in Abhängigkeit von $u$ finden. 
für eine Konstante $c$. 
Aus der Physik (Compressibility,Bulk modulus,Boyles law TODO: Welches ist es?) wissen wir, dass $p$ annähernd proportional zur Volumenänderung durch die Verschiebung $u$ ist. Die Verformung ändert das Volumen um den Faktor $\det(\nabla\chi)=\det(\Id+\nabla u)\approx 1+\Tr(\nabla u)$ für eine kleines $\nabla u$. Die Verschiebung $u$ verändert das Volumen also um ungefähr
\begin{align*}
	\Tr\left(\nabla u\right)
	&= \frac{1}{2}\left(\Tr\left(\nabla u\right)+\Tr\left(\nabla u^\top\right)\right) \\
	&= \Tr\left(\frac{1}{2}\left(\nabla u+\nabla u^\top\right)\right) \\
	&= \Tr(\e(u))
\end{align*}
Damit ergibt sich
\begin{align*}
	\sigma^K _{ij} \approx \lambda\Tr(\e(u))\delta_{ij}
\end{align*}
für eine Konstante $\lambda\in\R$.

\subsubsection*{Energiebetrachtungen}

TODO: Einheitliche Notation $\sigma:\e$ oder $\e:\sigma$

TODO: Energiebetrachtungsproof

Wir betrachten die Arbeit, die durch eine kleine Änderung der Verschiebung $u\to u+\delta u$ ($\norm{\delta u}$ klein) von den internen Kräften verrichtet wird. Gemäß der Formel $\text{Arbeit}=\text{Kraft}\cdot\text{Weg}$ erhält man für die Energie-/Arbeitsdichte $w$
\begin{align*}
	\delta w = f\cdot \delta v
	= \sum_{i,j} \partial_i\sigma_{ij}\delta v_j
\end{align*}
und nach Integration und Anwendung von Gauss
\begin{align*}
	\int_\omega w\dif x
	&= \int_\omega\sum_{i,j} \partial_i\sigma_{ij}\delta u_j\dif x \\
	&= \underbrace{\int_{\partial\omega}\sum_{i,j} \sigma_{ij}\delta u_jn_i\dif x}_{\to0, \text{für }\omega\text{ groß genug }}-\int_\omega\sum_{i,j} \sigma_{ij}\partial_i\delta u_j\dif x \\
	&= -\frac{1}{2}\int_\omega\sum_{i,j}\sigma_{ij}\left(\partial_i\delta u_j+\partial_j\delta u_i\right)\dif x \\
	&= -\frac{1}{2}\int_\omega\sum_{i,j}\sigma_{ij}\e_{ij}\dif x
\end{align*}
Dies motiviert
\begin{align*}
	\delta w=-\frac{1}{2}\sum_{ij}\sigma_{ij}\delta \e_{ij}
\end{align*}
zu setzen. Nach (\cite{Lif-1959}, S.12), haben wir $\partial_{\e_{ij}}w =\sigma_{ij}$ und deshalb $w=\frac{1}{2}\sum_{ij}\sigma_{ij}\e_{ij}$ (Argumentation gilt auch für nichtlinearen Fall).


(Das folgdende ist etwas komisch) Es gibt 2 unabhängige Scalare 2. Ordnung, die aus $\e$ hervorgehen, nämlich $\sum_i\e_{ii}^2$ und $\sum_{ij}\e_{ij}^2$
Nun Taylorn wir $w$ in $0$ bis zur 2. Ordnung und erhalten
\begin{align*}
	w=w_0+\frac{1}{2}\lambda\sum_i\e_{ii}^2+\mu\sum_{ij}\e_{ij}^2
\end{align*}
wobei die Terme erster Ordnung verschwinden, da $0=\sigma_{ij}|_{u=0}=\partial_{\e_{ij}}w\vert_{\e=0}$
Bei pure shear (\cite{Lif-1959}, S.10) ist $\Tr(\e)=0$ und bei hydrostatic compression $\e_{ij}=c\delta_{ij}$. Jede Deformation lässt sich durch
\begin{align*}
	\e_{ij}=(\e_{ij}-\frac{1}{d}\delta_{ij}\Tr(\e))+\frac{1}{3}\delta_{ij}\Tr(\e)
\end{align*}
in pure shear und hydrostatic compression zerlegen. Einsetzen in $w$ liefert
\begin{align*}
	w = \sum_{i,j,k}\mu(\e_{ij}-\frac{1}{3}\e_{kk}\delta_{ij})^2+\frac{1}{2}K\e_{kk}^2
\end{align*}
Außerdem erhält man
\begin{align*}
	\sigma_{ij}=K\Tr(\e)\delta_{ij}+2\mu(\e_{ij}-\frac{1}{d}\Tr(\e)\delta_{ij})
\end{align*}
Sowie für die Volumenänderung $dK\Tr(\e)=\Tr(\sigma)$. In \cite{Lif-1959}, S.13 wird die Bedeutung von $\nu$ und $E$ diskutiert.

Außerdem definieren wir (stored energy function)
\begin{align*}
	w(\e)\coloneqq \frac{1}{2}\e\colon C\e\coloneqq \frac{1}{2}\sum_{i,j,k,l}C_{ijkl}\e_{ij}\e_{kl}
\end{align*}
und nehmen an, dass $b\in V^*$ gegeben ist, dass uns die Arbeit liefert, welche von einer Verschiebung $v\in V$ auf unser Objekt ausgeübt wird. Damit können wir die potentielle Energie einer Verschiebung $v\in V$ definieren
\begin{align*}
	W(v)=\int_\Omega w(\e(v))\dif x- b(v)
\end{align*}


\begin{proposition}
	Es gilt
	\begin{align*}
		\partial_vW(u)=a(u,v)-\ell(v)
	\end{align*}
\end{proposition}
\begin{proof}
	Man rechnet
	\begin{align*}
		&\frac{1}{\theta}\left(W(u+\theta v) -W(u)\right) \\
		&= \frac{1}{\theta}\left(\frac{1}{2}a(u+\theta v,u+\theta v)-\ell(u+\theta v)-\frac{1}{2}a(u,u)-\ell(u)\right) \\
		&= \frac{1}{\theta}\left(\theta a(u,v)+\frac{1}{2}\theta^2a(v,v)-\theta \ell(v)\right) \\
		&= a(u,v)-\ell(v)+\frac{1}{2}\theta a(v,v) \\
		&\xrightarrow{\theta\to0} a(u,v)-\ell(v)
	\end{align*}
\end{proof}

Für St. Vernant-Kirchhoff Materialien ergibt sich die Energiedichte zu
\begin{align*}
	w = \frac{\lambda}{2}(\Tr(E))^2+\mu\Tr(\nabla u^\top\nabla u)
\end{align*}

\section{Existenz (und Eindeutigkeit?)}

\subsubsection*{Lax-Milgram lemma}

\subsubsection*{Grundlegende Bezeichnungen}


TODO: Definiere eine Domain

Wir definieren $H^k$($k\in\Z$) 

%Wir bezeichnen die Menge aller stetigen Abbildungen $f\colon V\to W$ mit $C(V,W)$ (Was sind zulässige V und W).
Gegeben sei ein Banachraum $V$. Wir definieren den Dualraum als $V^*\coloneqq C(V,\R)$.

TODO: definiere mehr dualitätsgedöns


%Sei $V$,$W$ normierte Räume und $T\colon V\to W$ eine stetige Lineare Abbildung. Es wird durch die Beziehung
%\begin{align*}
%	T^*(w^*) = w^*\circ T
%\end{align*}
%eine duale Abbildung $T^*\colon W^*\to V^*$ definiert. Nach (Conti) ist sie linear und stetig. Falls es sich bei $V$ und $W$ um Hilbert-Räume handelt, schreiben wir (genauer!) $T^\dagger$.
%
%TODO: definiere mehr dualitätsgedöns

\subsection*{Kornsche Ungleichungen}

\subsubsection*{Lemma von J.L. Lions}

\subsubsection*{Lemma von J.L. Lions}


Wir definieren den Raum
\begin{align*}
	X(\Omega)\coloneqq\{v\in H^{-1}(\Omega), \text{so dass für alle }i\colon\partial_iv\in H^{-1}(\Omega)\}
\end{align*}
Dies ist ein Hilbert Raum mit der Norm
\begin{align*}
	\norm{v}_X\coloneqq\Big(\sum_{\abs{\alpha}\leq1}\norm{\partial^\alpha v}_{-1,\Omega}^2\Big)^{1/2}
\end{align*}

\begin{lemma}[Eine erste Charakterisierung der Ableitung in $H^{-1}$]
	Sei $\Omega\subseteq\R^d$ ein Lipschitz-Gebiet (welche Vorraussetzungen genau). Dann gilt
	\begin{align*}
		L^2(\Omega)\subseteq X(\Omega)
	\end{align*}
\end{lemma}
\begin{proof}
	Sei $v\in L^2(\Omega)$. Wir definieren $\partial_i v\in \cD(\Omega)$
	\begin{align*}
		\inner{\partial_i v,\varphi} = -\inner{f,\partial_i\varphi}_{0,\Omega}
	\end{align*}
	für $\varphi\in\cD(\Omega)$. Weiter sehen wir
	\begin{align*}
		\abs{\inner{v,\partial_i\varphi}_{0,\Omega}}
		\leq \norm{v}_{0,\Omega}\norm{\nabla\varphi}_{0,\Omega}
		\leq \norm{v}_{0,\Omega}\norm{\varphi}_{1,\Omega}
	\end{align*}
	also $\norm{\partial_i v}_{-1,\Omega}\leq \norm{v}$, womit Hahn Banach eine Erweiterung von $\partial_iv$ liefert (genauer).
\end{proof}

\begin{proposition}[Dichtheit von $\cC^\infty(\Omega)$ in $X(\Omega)$]
Es ist $\cC^\infty(\Omega)$ dicht in $X(\Omega)$ für den Halbraum $\R^{d-1}\times\R_{>0}$.
\end{proposition}
\begin{proof}
	Wir setzen
	\begin{align*}
		Y(\Omega) =\{v\colon\R^d\to\R\mid \partial_dv\in H^{-1}(\R_{>0};L^2(\R^{d-1}))\}
	\end{align*}
	nun Behaupten wir, dass $Y(\Omega)$ in $X(\Omega)$ dicht ist. Sei dazu $\eta_j\in \cD(\R^{d-1})$ eine Standart Dirac-Schar. Sei $v\in X(\Omega)$. Setze $v_j\coloneqq v*\eta_j$ mit $*$ die Konvolution (sollte irgendwo definiert werden). Dann gilt $v_j\to v$ in $X(\Omega)$ (da die Norm auf $X$ stärker ist, als auf $L^2(\Omega)$) und $v_j\in Y(\Omega)$.
	
	Wir zeigen nun, dass $\cD(\overline{\Omega})$ (why overline) dicht in $X(\Omega)$ ist. Da $Y(\Omega)$ dicht in $X(\Omega)$ ist, reicht es zu zeigen, dass $\cD(\overline{\Omega})$ dicht in $Y(\Omega)$ ist.
	Angenommen nicht. Dann gibt es $u\in Y(\Omega)\setminus\overline{\cD}(\Omega)$. Da $\overline{\cD}(\Omega)$ abgeschlossen (bzgl. $\norm{\cdot}_{H^{-1}(\R_{>0};L^2(\R^{d-1}))}$) ist und $(Y(\Omega),\norm{\cdot}_{-1,\Omega})$ ein normierter Raum, liefert Hahn Banach (Conti cor. 4.22) ein $F\in Y(\Omega)^*\setminus\{0\}$, so dass $F\vert_{\overline{\cD(\Omega)}} = \{0\}$. Folglich ist $F$ von der Form (Begründung?)
	\begin{align*}
		F(v) =\int_0^\infty\inner{f_1,v}_{0,\R^{d-1}}+\inner{f_2,\partial_dv}_{0,\R^{d-1}}\dif x
	\end{align*}
	für $f_1,f_2\in H_0^1(\R_{>0};L^2(\R^{d-1}))$.
	Nun gilt für alle $v\in \cD(\overline{\Omega})$, dass
	\begin{align*}
		0&=F(v) \\
		&=\int_0^\infty\inner{f_1,v}_{0,\R^{d-1}}+\inner{f_2,\partial_dv}_{0,\R^{d-1}}\dif x \\
		&=\int_\R\inner{\tilde{f}_1,v}_{0,\R^{d-1}}+\inner{\tilde{f}_2,\partial_dv}_{0,\R^{d-1}}\dif x \\
		&=\int_\R\inner{\tilde{f}_1,v}_{0,\R^{d-1}}-\inner{\partial_d\tilde{f}_2,v}_{0,\R^{d-1}}\dif x \\
	\end{align*}
	Also folgt (aus Dichtheit und Norm-Argumenten), dass
	\begin{align*}
		\tilde{f}_1-\partial_d\tilde{f}_2 = 0
	\end{align*}
	Folglich ist $\partial_d\tilde{f}_2\in H^1(\R,L^2(\R^{d-1}))$, also $f_2\in H_0^2(\R_{>0};L^2(\R^{d-1})$. Aber dann folgt für alle $y\in Y(\Omega)$ (Warum?), dass
	\begin{align*}
		\int_{\R_{>0}}\inner{f_2,\partial_d v}\dif x_d = -\int_{\R_{>0}}\inner{\partial_df_2,v}\dif x_d
	\end{align*}
	und folglich $F=0$ ein Widerspruch.
\end{proof}


\begin{lemma}[Lemma von J.L. Lions]\label{le:LemmaVonLions}
Statement und Beweis aus \cite{Duv-1976}
Sei $\Omega\subseteq\R^d$ ein Gebiet mit regulärem Rand und $v\colon\Omega\to\R$, so dass $v\in H^{-1}(\Omega)$ und $\partial_iv\in H^{-1}(\Omega)$ für alle $i$, dann folgt $v\in L^2(\Omega)$.
\end{lemma}
\begin{proof}
	Wir behaupten, dass $X(\Omega)=L^2(\Omega)$ und zeigen dies in mehreren Schritten
	\begin{enumerate}
	\item Die Behauptung stimmt für $\Omega=\R^d$. Dazu bezeichne $\cF$ die Fourier-transformation. Dann folgt
	\begin{align*}
		(1+\abs{\xi}^2)^{-1/2}\cF(v)\in L^2(\R^d), \qquad
		(1+\abs{\xi}^2)^{-1/2}\xi_i\cF(v)\in L^2(\R^d)
	\end{align*}
	und deshalb weiter
	\begin{align*}
		\int_{\R^d}(1+\abs{\xi}^2)^{-1}(1+\sum_i\xi_i^2)\abs{\cF(v)}^2\dif\xi
	\end{align*}
	also ist $v\in L^2(\R^d)=L^2(\Omega)$
	
	\item Die Behauptung stimmt für einen Halbraum $\Omega=\R^{d-1}\times\R_{>0}$.
	Wir definieren $P\colon \cD(\Omega)\subseteq X(\Omega)\to\\cD(\Omega)\subseteq X(\R^d)$
	Wir setzen (anderer Buchstabe?) für $v\in \cD(\Omega)$
	
	\begin{align*}
		Pv(x) =\begin{cases}
			v(x) 							&,\text{falls }x_d>0 \\
			a_1 v(x',-x_d)+a_2v(x',-2x_d) 	&,\text{sonst}
		\end{cases}
	\end{align*}
	wobei $a_1+a_2=1$ und $a_1+a_2/2=-1$ (Warum?). Wir behaupten nun, dass $P$ auf $\cD(\Omega)$ (why is $\Omega$ closed?) stetig ist bezüglich $\norm{\cdot}_X$. Falls dies war ist, so erhalten wir eine Erweiterung (wie genau) von $P$ auf $\Omega$.
	Wegen der Dichte von $\cD(\Omega)$ nach der Proposition oben (wie genau) folgt $P\vert_\Omega(v)=v$ für $v\in X(\Omega)$. Dann folgt für $v\in X(\Omega)$, dass $Pv\in X(\R^d)$ und mit dem vorhergehenden, dass $Pv\in L^2(\R^d)$. Deshalb folgt, dass
	\begin{align*}
		v = P\vert_\Omega(v)\in L^2(\Omega)
	\end{align*}
	
	Es verbleibt die Stetigkeit von $P$ bezüglich $\norm{\cdot}_X$ / der Topologien zu zeigen.
	Dazu reicht es, die Stetigkeit von $P$ bezüglich der Topologie $H^{-1}(\Omega)\to H^{-1}(\Omega)$ auf $\cD(\Omega)$ und die Stetigkeit von $\partial_iP$ bezüglich der Topologie $H^{-1}(\Omega)\to H^{-1}(\Omega)$ auf $\cD(\Omega)$ zu zeigen. Es gilt für $i\in\{1,\dots,d-1\}$
	\begin{align*}
		\partial_iPv(x) &= \begin{cases}
			\partial_i v &,\text{falls }x_d>0 \\
			a_1\partial_iv(x',-x_d)+a_2\partial_iv(x',-2x_d) &,\text{falls }x_d<0
		\end{cases} \\
		&= P(\partial_iv(x))
	\end{align*}	
	sowie
	\begin{align*}
		\partial_dPv(x) &= \begin{cases}
			\partial_dv &,\text{falls }x_d>0 \\
			-a_1\partial_dv(x',-x_d)-2a_2\partial_dv(x',-2x_3) &,\text{falls }x_3<0
		\end{cases} \\
		&\eqqcolon Q(\partial_dv(x))
	\end{align*}
	Die Behauptung folgt also, wenn wir die Stetigkeit von $P\colon\cD(\Omega)\to X(\R^d)$ mit Topologien $\cD(\Omega)\subseteq H^{-1}(\Omega)$ und $X(\R^d)\subseteq H^{-1}(\R^d)$ und $Q\colon\cD(\Omega)\dots$ zeigen. Dazu berechnet man für $w\in H^1(\R^d)$
	\begin{align*}
		&\inner{Pv,w}_{0,\R^d} \\
		&= \int_{\R^{d-1}\times\R_{>0}}vw\dif x + \int_{\R^{d-1}\times\R_{<0}}a_1v(x',-x_d)w(x)+a_2v(x',-2x_d)w(x)\dif x \\
		&= \int_{\R^{d-1}\times\R_{>0}}vw\dif x +\int_{\R^{d-1}\times\R_{>0}}v(x',x_d)(- a_1)w(x',-x_d)-\frac{a_2}{2}v(x',x_d)w(x',-\frac{x_d}{2})\dif x \\
		&= \int_{\R^{d-1}\times\R_{>0}}v(x)\left(w(x) + a_1w(x',-x_d)+\frac{a_2}{2}w(x',-\frac{x_d}{2})\right)\dif x \\
		&= \inner{v,P^\dagger w}_{0,\Omega}
	\end{align*}
	Dann ist	
	\begin{align*}
		P^\dagger w(x',0) = w(x',0) + a_1w(x',0)+\frac{a_2}{2}w(x',0) = 0
	\end{align*}
	wenn wir $a_1+a_2/2=-1$ wählen und folglich $P^\dagger w\in H^1_0(\Omega)$. Wir sehen, dass die zu $P$ transponierte Abbildung $P^\dagger\colon H^1(\R^d)\to H^1_0(\Omega)$ stetig ist. Analog schlussfolgern wir für $Q$.
	
	
	
	
	\item Die Behauptung stimmt generell.
	Da $\Omega$ eine beschränkte Lipschitzberandete Menge ist, gibt es eine endliche Überdeckung $U_i$ von $\Omega$ und $\psi_i$, sodass $\Omega$ auf $U_i$ der strikte Hypograph (oder Epigraph) von $\psi_i$ ist für $i\geq1$ und $U_0\Subset\Omega$. Ferner gibt es dann eine Zerlegung der Eins bezuüglich $U_i$, also $\theta_i\in \cD(U_i)$ mit $0\leq\theta_i$, so dass $1=\sum_i\theta_i$ auf $\overline{\Omega}$. Indem wir $\theta_0$ mit $0$ auf $\R^d\setminus U_0$ erweitern, ist $\theta_0v\in X(\R^d)$. (The rest of the argumentation requires $C^1$ boundary, but is very incomplete).
	\end{enumerate}
\end{proof}

\subsubsection*{Kornsche Ungleichung ohne Randbedingungen}

\begin{lemma}[Globale Approximation, Conti Thrm. 7.6]
	Sei $\Omega\subseteq\R^d$ beschränkt mit Lipschitz-Rand und $p\in[1,\infty)$, $v\in W^{1,p}(\Omega)$. Dann gibt es eine Folge $v_i\in C_c^\infty(\R^d)$ mit $v_i\to v$ in $W^{1,p}(\Omega)$
\end{lemma}

\begin{lemma}[Kettenregel, Conti Lemma 2.39 - reicht hier nicht aus]
	Sei $\Omega\subseteq\R^d$ offen, $p\in[1,\infty)$, $v\in W^{1,p}(\Omega;\R^m)$
\end{lemma}

\begin{lemma}[Erweiterungen, Conti Thrm. 7.8]
	Sei $\Omega\subseteq\R^d$ beschränkt mit Lipschitz-Rand und $p\in[1,\infty)$. Dann existiert eine beschränkte, lineare Erweiterung $E\colon W^{1,p}(\Omega)\to W^{1,p}(\R^d)$ mit $Ev\vert_\Omega = v$ und $\norm{\partial_iEv+\partial_jEw}_{2,\R^d}^2 \leq c\norm{\partial_iv+\partial_jw}_{2,\Omega}^2$.
\end{lemma}
\begin{proof}
	Wir nehmen zunächst an, dass $v\in C_c^\infty(\R^d)$ ist. Seien $\theta_i$ und $B_i\coloneqq B_{r_i}$ wie in dem Lemma oben zur Partition der Eins. Setze $i>0$ fest. Sei $\psi$, wie in (Conti, 7.1). Man sieht  (genauer) mit geometrischen Überlegungen, dass
	\begin{align*}
		0<(1+M^2)^{-1/2}(\psi(x')-x_d)\leq \dist(x,\Omega)\leq \psi(x')-x_d
	\end{align*}
	Da $\Omega$ Lipschitz ist, gibt es nach Stein (angeblich) ein $f\in C^\infty$, so dass
	\begin{align*}
		0 < 2(\psi(x')-x_d)\leq f(x)\leq c_i(\psi(x')-x_d)
	\end{align*}
	und für Multiindizes $\alpha$
	\begin{align*}
		\abs{\partial^\alpha f(x)}\leq c_\alpha f(x)^{1-\alpha}
	\end{align*}
	
	
	Wir definieren $\Phi_t\colon\R^d\to\R^d$ für $t\in[1,2]$ durch
	\begin{align*}
		\Phi_t(x)\coloneqq\begin{cases}
			x &,\text{falls }x_d\leq\psi(x') \\
			(x',x_d+tf(x)) &,\text{sonst}
		\end{cases}
	\end{align*}
	Sowie ein Gewicht $w\in c^0([1,2])$, so dass
	\begin{align*}
		\int_1^2w(t)\dif t &= 1 \\
		\int_1^2tw(t)\dif t &= 0
	\end{align*}
	Nun definieren wir eine Erweiterung
	\begin{align*}
		\tilde{u}_j(x)\coloneqq\begin{cases}
			\int_1^2w(t)\left(u_j(\Phi_t(x))+t(\partial_jf(x))u_d(\Phi_t(x))\right)\dif t &,\text{falls }x\text{ im Epigraphen von }\psi\text{ ist.} \\
			u(x) &,\text{sonst}
		\end{cases}
	\end{align*}
	In Nitsche wird jetzt Stetigkeit gezeigt (nicht Lipschitz-Stetigkeit).	
	
	Wir wollen die Lipschitzstetigkeit von $\Phi$ zeigen. Seien dazu $x,y\in\R^d$. Falls beide auf im Epigraphen oder beide im Supergraphen von $\Psi$ sind, sind wir fertig. Andernfalls nehmen wir $x_n<\psi(x')$ und $\psi(y')<y_n$ an. Sei $z$ der zu $x$ nächste Punkt auf der Strecke $[x,y]$ mit $z_n=\psi(z)$. Dann gehört $z$ sowohl zum Epigraphen, als auch zum Supergraphen von $\psi$ und wir erhalten
	\begin{align*}
		\abs{\Phi(x)-\Phi(y)}\leq\abs{\Phi(x)-\Phi(z)}+\abs{\Phi(z)-\Phi(y)}\leq L\abs{x-z}+L\abs{z-y} =\abs{x-y}
	\end{align*}
	und die Lipschitzstetigkeit folgt. Da $B_i$ beschränkt ist, (und noch etwas) folgt damit $\Phi\in W^{1,\infty}(B_i)$. Nun setzen wir 
	\begin{align*}
		\tilde{v}(x)\coloneqq\theta_i(x)v(\Phi(x))
	\end{align*}
	Wir erhalten (warum genau?)
	\begin{align*}
		\e_{jk} = \int_1^2w(t)\left(\e_{jk}(u\circ \Phi_t)+t(\partial_jf)\e_{kd}(u\circ \Phi_t)+t(\partial_kf)\e_{jd}(u\circ \Phi_t)+t^2(\partial_jf)(\partial_kf)\e_{dd}(u\circ \Phi_t)+t(\partial_{j}\partial_kf)u_d\circ\Phi_t\right)\dif t
	\end{align*}
	Wegen Taylor gilt
	\begin{align*}
		u_d\circ\Phi_t = u_d\circ\Phi_1+f\int_1^t(\partial_du)\circ\Phi_s\dif s
	\end{align*}
	Dies führt dann zu (warum?)
	\begin{align*}
		\int_1^2tw(t)u_d\circ \Phi_t\dif t
		= \int_1^2tw(t)\left(u_d\circ\Phi_1+f\int_1^t(\partial_du)\circ\Phi_s\dif s\right)\dif t \\
		= \int_1^2tw(t)u_d\circ\Phi_1+f\int_1^2tw(t)\int_1^t(\partial_du)\circ\Phi_s\dif s\dif t \\
		= f\int_1^2tw(t)\int_1^t(\partial_du)\circ\Phi_s\dif s\dif t\\
		= f\int_1^2\e_{dd}(u\circ\Phi_s)\int_s^2tw(t)\dif t\dif s
	\end{align*}
	
	Sei $x'$ fest. Nun wählen wir eine Transformation, so dass $\psi(x')=0$ (Wo wird das verwendet?). Wir erhalten
	\begin{align*}
		\int_1^2\abs{\e_{jk}(u(\Phi_t(x)))}\dif t
		&= \int_1^2\abs{\e_{jk}(u(x_d+tf(x_d)))}\dif t
		&= \int_{x_d+f(x_d)}^{x_d+2f(x_d)}f^{-1}(x_d)\abs{\e_{jk}(u(s))}\dif s
	\end{align*}
	und weiter
	\begin{align*}
		f^{-1}(x_d)\leq\frac{1}{2}\abs{x_d}^{-1} \\
		x_d+f(x_d)\geq\abs{x_d} \\
		x_d+2f(x_d)\leq 2c_i\abs{x_d}
	\end{align*}
	Wir erhalten mit $c=2c_i$, dass
	\begin{align*}
		2\int_1^2\abs{\e_{jk}(u(\Phi_t(x)))}\dif t\leq \frac{1}{\abs{x_d}}\int_{\abs{x_d}}^{c\abs{x_d}}\abs{\e_{jk}(u(s))}\dif s
	\end{align*}
	und weiter mit der Schwarzschen Ungleichung
	\begin{align*}
		4\int_1^2\abs{\e_{jk}(u(\Phi_t(x)))}^2\dif t\leq \frac{2}{3}c^{3/2}\abs{x_d}^{-1/2}\int_{\abs{x_d}}^{c\abs{x_d}}s^{-1/2}\abs{\e_{jk}(u(s))}^2\dif s
	\end{align*}
	Es folgt
	\begin{align*}
		\int_{-\infty}^{\psi(x')}\int_1^2\abs{\e_{jk}(u(\Phi_t(x)))}^2\dif t
		&\leq c\int_{-\infty}^{\psi(x')}s^{-1/2}\abs{\e_{jk}(u(s))}^2\int_{c^{-1}s}^st^{-1/2}\dif t\dif s \\
		&\leq c\int_{\psi(x')}^{\infty}\abs{\e_{jk}(u(s))}^2\dif s \\
	\end{align*}
	
	
	
	
	Nach der Kettenregel (zitieren) folgt $\tilde{v}\in W^{1,p}(B_i)$ mit (genauer)
	\begin{align*}
		\nabla \tilde{v} = (\nabla v\circ\Phi)\nabla\Phi
	\end{align*}
	Folglich ist
	\begin{align*}
		\abs{\nabla \tilde{v}}(x)\leq \abs{\nabla\Phi}_{0,\infty,B_i}\abs{\nabla v(\Phi(x))}\leq c_i\abs{\nabla v}(\Phi(x))
	\end{align*}
	Wir mit dem Transformationssatz (Frage: Was passiert mit dem $\theta_i$?, Recap Transformationssatz)
	\begin{align*}
		\int_{B_i}\abs{\nabla \tilde{v}}^p+\abs{\tilde{v}}^p \dif x
		&\leq \int_{B_i}c_i\abs{\nabla v(\Phi(x))}^p+\abs{v(\Phi(x))}^p\dif x \\
		&= \int_{B_i\cap\Omega}\left(c_i\abs{\nabla v(x)}^p+\abs{v(x)}^p\right)\abs{\det(\nabla\Phi(x))}\dif x \\
		&\leq C_i\int_{B_i\cap\Omega}\abs{\nabla v}^p+\abs{v}^p\dif x
	\end{align*}
	
	dies alles gilt analog für ein $w\in C_c^\infty(\R^d)$.
	Wir berechnen nun konkret für $x_d\leq\psi(x')$ oder $l>d$
	\begin{align*}
		\partial_j \Phi_l(x)
		= \partial_jx_l = \delta_{jl}
	\end{align*}
	Damit ergibt sich
	\begin{align*}
		\abs{\partial_j\tilde{v}(x)+\partial_k\tilde{w}(x)}
		&= \abs{\sum_{l}\partial_lv(\Phi(x))\partial_j\Phi_l(x)+\sum_l\partial_lw(\Phi(x))\partial_k\Phi_l(x)} \\
		&= \abs{(\partial_jv)(\Phi(x))+(\partial_kw)(\Phi(x))}
	\end{align*}
	Falls dagegen $x_d>\psi(x')$, so folgt
	\begin{align*}
		\partial_j\Phi_d(x)
		= \partial_j(2\psi(x')-x_d)
	\end{align*}
	Damit ergibt sich
	\begin{align*}
		&\abs{\partial_j\tilde{v}(x)+\partial_k\tilde{w}(x)} \\
		&= \abs{\sum_{l}\partial_lv(\Phi(x))\partial_j\Phi_l(x)+\sum_l\partial_lw(\Phi(x))\partial_k\Phi_l(x)} \\
		&= \abs{(\partial_jv)(\Phi(x))+(\partial_dv)(\Phi(x))\partial_j(2\psi(x')-x_d)+(\partial_kw)(\Phi(x))+(\partial_dw)(\Phi(x))\partial_k(2\psi(x')-x_d)} \\
		&\leq \abs{(\partial_jv)(\Phi(x))+(\partial_kw)(\Phi(x))}+\abs{(\partial_dv)(\Phi(x))\partial_j(2\psi(x')-x_d)+(\partial_dw)(\Phi(x))\partial_k(2\psi(x')-x_d)} \\
		&\leq \abs{(\partial_jv)(\Phi(x))+(\partial_kw)(\Phi(x))}+(2\abs{\psi}_{W^{1,\infty}(\Omega)}+1)\abs{(\partial_dv)(\Phi(x))+(\partial_dw)(\Phi(x))}
	\end{align*}
	Also erhalten wir insgesamt mit dem Transformationssatz
	\begin{align*}
		\int_{B_i}\abs{\partial_j\tilde{v}(x)+\partial_k\tilde{w}(x)}^p\dif x 
		&\leq \int_{B_i}\abs{\partial_jv(x)+\partial_kw(x)}^p+c_i\dif x 
	\end{align*}
	
	Indem wir über $i$ summieren (ausführen) erhalten wir einen beschränkten Erweiterungsoperator $\tilde{E}\colon W^{1,p}(\Omega)\to W^{1,p}(\tilde{\Omega})$, wobei $\tilde{\Omega}\coloneqq\bigcup_iB_i$, so dass die behaupteten Eigenschaften erfüllt sind. Nun wählen wir eine Abschneidefunktion $\tilde{\theta}\in C_c^\infty(\Omega;[0,1])$ mit $\theta=1$ auf $\Omega$ und setzen
	\begin{align*}
		(Ev)(x)\coloneqq\tilde{\theta}(x)(\tilde{E}v)(x)
	\end{align*}
	Dann ist $Ev\in W_0^{1,p}(\tilde{\Omega}\subseteq W^{1,p}(\R^d)$ und nach der Produktregel
	\begin{align*}
		\nabla(Ev)=\tilde{\theta}\nabla(\tilde{E}v)+(\tilde{E}v)\otimes \nabla \tilde{\theta}
	\end{align*}
	also folgt
	\begin{align*}
		\norm{\nabla E u}_{p}\leq \norm{\nabla \tilde{E}v}_{p}+\norm{\tilde{E}v}_p\norm{\nabla\tilde{\theta}}_{\infty}
	\end{align*}
	womit die Behauptung folgt (wirklich?).
\end{proof}

\newpage

Wir setzen
\begin{align*}
	K(\Omega;\R^d)\coloneqq\{v\in L^2(\Omega;\R^d)\colon \e(v)\in L^2(\Omega;\R^d) \text{ im distributionellen Sinne}\}
\end{align*}
Dabei verstehen wir unter $\e_{ij}(v)\in L^2(\Omega)$ im distributionellen Sinne, dass für alle $\varphi\in\cD(\Omega)$
\begin{align*}
	\inner{\e_{ij}(v),\varphi}_{0,\Omega}= -\frac{1}{2}\left(\inner{v_i,\partial_j\varphi}_{0,\Omega}+\inner{v_j,\partial_i\varphi}_{0,\Omega}\right)
\end{align*}
Wir definieren nun für $v\in K(\Omega;\R^d)$ die Norm
\begin{align*}
	\norm{v}_K\coloneqq \left(\abs{v}_{0,\Omega}^2+\abs{\e(v)}_{0,\Omega}^2\right)^{1/2}
\end{align*}
$K(\Omega;\R^d)$ wird versehen mit der Norm $\norm{\cdot}_K$ zu einem Hilbert Raum.	
Wir behaupten, dass
\begin{proposition}
	Es gilt für $\Omega\subseteq\R^d$ offen, zusammenhängend und mit $C^1$ Rand, dass
	\begin{align*}
		K(\Omega;\R^d) =H^1(\Omega,\R^d)
	\end{align*}
\end{proposition}
\begin{proof}
	Es folgt aus $v\in K(\Omega;\R^d)$, dass $\e_{ij}(v)\in L^2(\Omega)$ und folglich $\partial_k\e_{ij}(v)\in H^{-1}(\Omega)$. Nun erhält man
	\begin{align*}
		\partial_j\partial_kv_i = \partial_j\e_{ik}(v)+\partial_k\e_{ij}(v)-\partial_i\e_{jk}(v)\in H^{-1}(\Omega)
	\end{align*}
	Es folgt dann mit Lemma \ref{le:LemmaVonLions} (Lemma von Lions), dass folglich $\partial_kv_i\in L^2(\Omega)$, also ist $v\in H^1(\Omega,\R^d)$ und es folgt $K(\Omega;\R^d)\subseteq H^1(\Omega,\R^d)$.
\end{proof}

Wir zitieren nun ein Korollar aus Werner (Funktionalanalysis, 2011) einen Korollar vom Satz der offenen Abbildungen (Banach-Steinhaus)
\begin{proposition}[Korollar IV.3.5, Werner]
	Sind $\norm{\cdot}_1$ und $\norm{\cdot}_2$ zwei Normen auf dem Vektorraum $V$, die beide $V$ zu einem Banachraum machen und gibt es ein $c>0$, so dass für alle $v\in V$
	\begin{align*}
		\norm{v}_1\leq c\norm{v}_2
	\end{align*}
	So sind $\norm{\cdot}_1$ und $\norm{\cdot}_2$ äquivalent.
\end{proposition}

Da sowohl $\norm{\cdot}_K$, als auch $\norm{\cdot}_{1,\Omega}$ $K(\Omega;\R^d)=H^1(\Omega;\R^d)$ zu einem Banachraum machen und gilt
\begin{align*}
	\norm{v}_K = \left(\abs{v}_{0,\Omega}^2 + \abs{\e(v)}_{0,\Omega}^2\right)^{1/2} \leq \abs{v}_{0,\Omega}+\frac{1}{2}\left(\abs{\nabla v}_{0,\Omega}+\abs{\nabla v^\top}_{0,\Omega}\right) = \norm{v}_{1,\Omega}
\end{align*}
folgt aus der vorausgehenden Proposition, dass $\norm{\cdot}_K$ und $\norm{\cdot}_{1,\Omega}$ äquivalent sind, also insbesondere

\begin{lemma}[Kornsche Ungleichung ohne Randbedingungen]
	(nach \cite{Cia-1997} bewiesen für d=3)
	Sei $\Omega\subseteq\R^d$ ein Gebiet, dann gibt es ein $c_K>0$, so dass für alle $v\in H^1(\Omega;\R^d)$ gilt
	\begin{align*}
		\norm{v}_{1,\Omega}\leq c_K\norm{v}_K
	\end{align*}
\end{lemma}


Ferner ist $\abs{\e(\cdot)}_{0,\Omega}$ convex und Gateaux-differenzierbar, also schwach lower semi-continuous (nach \cite{Kik-1988})

\subsubsection*{Erweiterung für Lipschitz-Mengen}
Unser Ziel ist es, diese Aussage auf mehr Gebiete $\Omega$ zu erweitern.
Dazu definieren wir für eine Funktion $\psi\colon\R^{d-1}\to\R^d$ den Epigraphen durch
\begin{align*}
	\epi(\psi)\coloneqq\{(y',y_n)\colon y_n>\psi(y')\}
\end{align*}
Weiter bezeichnen wir mit $H^k_S(\Omega;\R^d)$ die Menge aller $v\in H^k(\Omega;\R^d)$, so dass $v=0$ auf der Randmenge $S\subseteq\Gamma$.
Mit diesen Vorraussetzungen erhalten wir das folgende Lemma

\begin{lemma}[Erweiterung von Epigraphen]\label{le:ErweiterungVonEpigraphen}
	Gegeben sei ein Ball $B=B_r(0)\subseteq\R^d$, eine Lipschitz-stetige Funktion $\psi\colon\R^{d-1}\to\R$ mit beschränkter Lipschitzkonstante $\Lip(\psi)\leq M$ Diese definieren ein Gebiet $\Omega\coloneqq B\cap\epi(\psi)$. Wir definieren einen Radius $\tir\coloneqq\sqrt{r^2+(2M+1)r}$ und einen Ball $\tiB\coloneqq B_{\tir}(0)$. Dann existiert ein $c_E>0$, welches von $d$ und $M$ abhängt, eine stetige, lineare Erweiterung $E\colon H^1_{\partial B}(\Omega;\R^d)\to H^1_{\partial \tiB}(B';\R^d)$ mit $Ev\vert_\Omega = v$ auf $B$ und
	\begin{align}
		\abs{\e(Ev)}_{0,\tiB}^2 \leq c_E\abs{\e(v)}_{0,\Omega}^2 \label{eq:ErweiterungEpigraphenUngleichung1}
	\end{align}
\end{lemma}
\begin{proof}
	(siehe auch Conti, Thrm 7.8) Wir folgen im Wesentlichen \cite{Nit-1981}. Im Folgenden bezeichnen wir mit $\tiOmega\coloneqq\epi(\psi)\cap\tiB$ und $\tiOmega^\complement\coloneqq \tiB\setminus\tiOmega$ das Komplement von $\tiOmega$ bezüglich $\tiB$.
	 Man sieht  (genauer) mit geometrischen Überlegungen, dass
	\begin{align*}
		0<(1+M^2)^{-1/2}(\psi(x')-x_d)\leq \dist(x,\Omega)\leq \psi(x')-x_d
	\end{align*}
	Da $\epi(\psi)$ Lipschitz ist, gibt es nach Stein (angeblich) ein $f\in C^\infty(\tiOmega^\complement;\R_{>0})$ (nachschauen), so dass
	\begin{align}
		0 < 2(\psi(x')-x_d)\leq f(x)\leq \frac{c_f}{2}(\psi(x')-x_d) \label{eq:erweiterungEpigraphfUngleichung}
	\end{align}
	(TODO: wo wird der Faktor 1/2) verwendet?) und für Multiindizes $\alpha$ gilt
	\begin{align*}
		\abs{\partial^\alpha f(x)}\leq c_{\abs{\alpha}} f(x)^{1-\abs{\alpha}}
	\end{align*}
	
	\begin{figure}[h]
	\centering
	\input{Drawings/ErweiterungNullrand2.pdf_tex}
	\caption{Visualisierung der Erweiterung}
	\end{figure}
	
	Wir definieren eine Spiegelung $\Phi_t\colon\tiOmega^\complement\to\epi(\psi)$ für $t\in[1,2]$ durch
	\begin{align*}
		\Phi_t(x)\coloneqq\vect{x'\\x_d+tf(x)}
	\end{align*}
	wobei $x'\coloneqq\vect{x_1&\dots& x_{d-1}}^\top$. 
	für $x\in\tiOmega^\complement$ folgt mit \eqref{eq:erweiterungEpigraphfUngleichung} 
	\begin{align*}
		 \psi(x')-x_d\leq 2(\psi(x')-x_d)\leq f(x)
	\end{align*}
	und folglich
	\begin{align*}
		\psi(x')\leq f(x)+x_d\leq x_d+tf(x)
	\end{align*}
	Also ist tatsächlich $\Phi_t(x)\in\epi(\psi)$.
	Nun definieren wir ein Gewicht $w\in C^0([1,2])$, so dass
	\begin{align*}
		\int_1^2w(t)\dif t = 1
		\qquad\text{und}\qquad\int_1^2tw(t)\dif t = 0
	\end{align*}
	Schließlich definieren wir eine Erweiterung für $v\in H^1_{\partial B}(\Omega;\R^d)$
	\begin{align*}
		(Ev_j(x))_j\coloneqq\begin{cases}
			v(x) &,\text{falls }x\in\epi(\psi)\\
			\int_1^2w(t)\big(v_j(\Phi_t(x))+t(\partial_jf(x))v_d(\Phi_t(x))\big)\dif t &,\text{sonst}
		\end{cases}
	\end{align*}
	wobei wir stillschweigend $v_j=0$ auf $\epi(\psi)\setminus B$ erweitert haben.
	
	Wir zeigen zunächst, dass $Ev\in H^1(\tiB;\R^d)$. Es geht aus der Konstruktion hervor, dass $Ev\big\vert_{\tiOmega}\in H^1(\tiOmega;\R^d)$ und $Ev\big\vert_{\tiOmega^\complement}\in H^1(\tiOmega^\complement;\R^d)$. Es verbleibt also die Stetigkeit an der Naht $\graph(\psi)=\{(x',\psi(x'))\colon x'\in\R^d\}$ zu zeigen.
	Sei dazu $v\in C^0(\Omega;\R^d)$ und $x^k=(x',x_d^k)\in\tiOmega^\complement$ eine Folge, die gegen $x^*=(x',\psi(x'))\in \graph(\psi)$ konvergiert. Dann gilt $0\leq f(x^k)\leq\frac{c_f}{2}(\psi(x')-x_d^k)\to0$ für $k\to\infty$ und es folgt $\Phi_t(x^k)\to x^*$. Damit ergibt sich aufgrund der Wahl von $w$
	\begin{align*}
		\abs{\int_1^2w(t)t(\partial_jf(x^k))v_d(\Phi_t(x))\dif t}
		&=\abs{(\partial_jf(x^k))}\abs{\int_1^2w(t)tv_d(\underbrace{\Phi_t(x^k)}_{\to x^*})\dif t} \\
		&\leq c_1 \abs{\int_1^2w(t)tv_d(\underbrace{\Phi_t(x)}_{\to x^*}))\dif t} \\
		&\xrightarrow{k\to\infty}c_1 \abs{\int_1^2w(t)tv_d(x^*)\dif t}
		=0
	\end{align*}
	Damit ergibt sich wieder aufgrund der Wahl von $w$
	\begin{align*}
		\lim_{k\to\infty}Ev(x^k)&=\lim_{k\to\infty}\int_1^2w(t)\big(v_j(\underbrace{\Phi_t(x^k)}_{\to x^*})+t(\partial_jf(x^k))v_d(\Phi_t(x^k))\big)\dif t \\
		&=\int_1^2w(t)v_j(x^*)\dif t
		= v_j(x^*)
	\end{align*}
	Wie erwünscht.
	
	Wir zeigen nun $Ev\in H^1_{\partial B}(\tiB;\R^d)$, also dass $Ev(x)=0$ für alle $x\in\partial \tiB$.
	Für den Fall, dass $x\in\tiOmega\setminus\Omega$, ist $Ev(x) = v(x) = 0$ nach Konstruktion.
	Falls $\abs{x'}>r$ ist ebenfalls nach Konstruktion $Ev(x)=0$.
	Falls nun $x\in\tiOmega^\complement$, $\abs{x'}\leq r$ und $x_d\leq0$, so folgt
	\begin{align*}
		\tir^2 = \abs{x}^2 = \abs{x'}^2+x_d^2\leq r^2+x_d^2
	\end{align*}
	und wegen $x_d\leq0$ ist 
	\begin{align*}
		x_d\leq -\left(\tir^2-r^2\right)^{1/2}= -2Mr-r
	\end{align*}
	Damit folgt
	\begin{align*}
		x_d\leq -2Mr-r\leq -2\Lip(\psi)\abs{x'}-r\leq -2\psi(x')-r
	\end{align*}
	und folglich
	\begin{align*}
		r-x_d\leq -2(\psi(x')-x_d)\leq f(x)
	\end{align*}
	Also
	\begin{align*}
		r\leq x_d+f(x) \leq x_d+tf(x)
	\end{align*}
	und es ist also $\Phi_t(x)\in\epi(\psi)\setminus B$, also $v_j(\Phi_t(x))=0$ und somit auch $Ev(x) = 0$. Dies war auch die Motivation für die Wahl von $\tir$.
	
	Die Linearität des Operators ist klar, ebenso die Stetigkeit, die aus der Beschränktheit folgt. Es verbleibt also die Eigenschaft \eqref{eq:ErweiterungEpigraphenUngleichung1} zu zeigen.
	
	Nun gilt mit der Kettenregel (siehe z.B. Conti lemma 2.39) für $i>d$
	\begin{align*}
		\partial_i(v_j\circ\Phi_t)
		&= \nabla v_j\big\vert_{\Phi_t} \partial_i\Phi_t \\
		&= \partial_iv_j\big\vert_{\Phi_t}+\partial_dv_j\big\vert_{\Phi_t}\partial_i(x_d+tf(x)) \\
		&= \partial_iv_j\big\vert_{\Phi_t}+t\partial_if\partial_dv_j\big\vert_{\Phi_t}
	\end{align*}
	und für $i=d$
	\begin{align*}
		\partial_d(v_j\circ\Phi_t)
		&= \nabla v_j\big\vert_{\Phi_t} \partial_d\Phi_t \\
		&= \partial_dv_j\big\vert_{\Phi_t}\partial_d(x_d+tf(x)) \\
		&= \partial_dv_j\big\vert_{\Phi_t}+t\partial_df\partial_dv_j\big\vert_{\Phi_t}
	\end{align*}
	Also in jedem Fall
	\begin{align*}
		\partial_i(v_j\circ\Phi_t) = \partial_iv_j\big\vert_{\Phi_t}+t\partial_if\partial_dv_j\big\vert_{\Phi_t}
	\end{align*}
	Wir erhalten auf $\tiOmega^\complement$
	\begin{align*}
		2\e_{ij}(Ev)
		&= \partial_i\int_1^2w(t)\big(v_j\circ\Phi_t+t(\partial_jf)v_d\circ\Phi_t\big)\dif t \\
		&\quad+\partial_j\int_1^2w(t)\big(v_i\circ\Phi_t+t(\partial_if)v_d\circ\Phi_t\big)\dif t \\
		&= \int_1^2w(t)\big(\partial_i(v_j\circ\Phi_t)+t(\partial_i\partial_jf)v_d\circ\Phi_t+t(\partial_jf)\partial_i(v_d\circ\Phi_t)\big)\dif t \\
		&\quad+\int_1^2w(t)\big(\partial_j(v_i\circ\Phi_t)+t(\partial_j\partial_if)v_d\circ\Phi_t+t(\partial_if)\partial_j(v_d\circ\Phi_t(x))\big)\dif t \\
		%
		&\begin{aligned}=\int_1^2w(t)\Big(\quad&\partial_iv_j\big\vert_{\Phi_t}+\partial_if\partial_dv_j\big\vert_{\Phi_t}+t(\partial_i\partial_jf)v_d\circ\Phi_t & \\
		+&t(\partial_jf)\partial_iv_d\big\vert_{\Phi_t}+t^2(\partial_jf)^2\partial_dv_d\big\vert_{\Phi_t} & \\
		+&\partial_jv_i\big\vert_{\Phi_t}+t\partial_jf\partial_dv_i\big\vert_{\Phi_t}+t(\partial_j\partial_if)v_d\circ\Phi_t & \\
		+&t(\partial_if)\partial_jv_d\big\vert_{\Phi_t}+t^2(\partial_if)(\partial_jf)\partial_dv_d\big\vert_{\Phi_t}&\Big)\dif t
		\end{aligned} \\
		%
		&\begin{aligned}=\int_1^2w(t)\Big(\quad&\partial_iv_j\big\vert_{\Phi_t}+\partial_jv_i\big\vert_{\Phi_t}+t(\partial_if)\partial_jv_d\big\vert_{\Phi_t}+t\partial_if\partial_dv_j\big\vert_{\Phi_t} & \\
		+&2t(\partial_i\partial_jf)v_d\circ\Phi_t+t(\partial_jf)\partial_iv_d\big\vert_{\Phi_t}+t\partial_jf\partial_dv_i\big\vert_{\Phi_t} & \\
		+&t^2(\partial_jf)^2\partial_dv_d\big\vert_{\Phi_t}+t^2(\partial_if)(\partial_jf)\partial_dv_d\big\vert_{\Phi_t}&\Big)\dif t \\
		\end{aligned}\\
		%
		&\begin{aligned}=\int_1^2w(t)\Big(\quad&2\e_{ij}(v)\big\vert_{\Phi_t}+2t(\partial_if)\e_{jd}(v)\big\vert_{\Phi_t}+2t(\partial_jf)\e_{id}(v)\big\vert_{\Phi_t} & \\
		+&2t^2(\partial_if)(\partial_jf)\e_{dd}(v)\big\vert_{\Phi_t}+2t(\partial_{i}\partial_jf)v_d\circ\Phi_t&\Big)\dif t
		\end{aligned}
	\end{align*}
	Zunächst möchten wir uns um den letzten Term im Integral kümmern. Wir erhalten mit dem Hauptsatz der Integralrechnung (Vorraussetzungen prüfen) und der Kettenregel auf $\tiOmega^\complement$
	\begin{align*}
		v_d\circ\Phi_t
		&= v_d\circ\Phi_1+\int_1^t\partial_sv_d\circ\Phi_s\dif s \\
		&= v_d\circ\Phi_1+\int_1^t\nabla v_d\big\vert_{\Phi_s}\partial_s\Phi_s\dif s \\
		&= v_d\circ\Phi_1+f\int_1^t\partial_dv_d\big\vert_{\Phi_s}\dif s
	\end{align*}
	Womit dann folgt, dass
	\begin{align*}
		\int_1^2tw(t)v_d\circ \Phi_t\dif t
		&= \int_1^2tw(t)\left(v_d\circ\Phi_1+f\int_1^t\partial_dv_d\big\vert_{\Phi_s}\dif s\right)\dif t \\
		&= \int_1^2tw(t)v_d\circ\Phi_1\dif t+f\int_1^2tw(t)\int_1^t\partial_dv_d\big\vert_{\Phi_s}\dif s\dif t \\
		&= f\int_1^2tw(t)\int_1^t\partial_dv_d\big\vert_{\Phi_s}\dif s\dif t\\
		&= f\int_1^2\e_{dd}(v)\big\vert_{\Phi_s}\int_s^2tw(t)\dif t\dif s
	\end{align*}
	Wir erhalten damit nun die Abschätzung 
	\begin{align*}
		&\abs{\e_{ij}(Ev)} \\
		&\begin{aligned}\leq \norm{w}_{L^\infty([1,2])}\int_1^2\Big|&\quad\e_{ij}(v)\big\vert_{\Phi_t}+t(\partial_if)\e_{jd}(v)\big\vert_{\Phi_t}+t(\partial_jf)\e_{id}(v)\big\vert_{\Phi_t}& \\
		&+t^2(\partial_if)(\partial_jf)\e_{dd}(v)\big\vert_{\Phi_t}+&\Big|\dif t
		\end{aligned} \\
		&\quad+\abs*{(\partial_{i}\partial_jf)\int_1^2w(t)tv_d\circ\Phi_t\dif t} \\
		&\begin{aligned}\leq c_w\Big(&\quad\int_1^2\abs{\e_{ij}(v)\big\vert_{\Phi_t}}\dif t+\norm{t(\partial_if)}_{L^\infty([1,2])}\int_1^2\abs{\e_{jd}(v)\big\vert_{\Phi_t}}\dif t & \\
		&+\norm{t(\partial_jf)}_{L^\infty([1,2])}\int_1^2\abs{\e_{id}(v)\big\vert_{\Phi_t}}\dif t & \\
		&+\norm{t^2(\partial_if)(\partial_jf)}_{L^\infty([1,2])}\int_1^2\abs{\e_{dd}(v)\big\vert_{\Phi_t}}\dif t&\Big )
		\end{aligned} \\
		&\quad+\abs*{(\partial_{i}\partial_jf)}\abs{f}\abs*{ \int_1^2\e_{dd}(v)\big\vert_{\Phi_s}\int_s^2tw(t)\dif t\dif s} \\
		&\leq c_w\int_1^2\abs{\e_{ij}(v)\big\vert_{\Phi_t}}\dif t
		+2c_wc_1\int_1^2\abs{\e_{jd}(v)\big\vert_{\Phi_t}}\dif t
		+ 2c_wc_1\int_1^2\abs{\e_{id}(v)\big\vert_{\Phi_t}}\dif t \\
		&\quad+ 4c_wc_1^2\int_1^2\abs{\e_{dd}(v)\big\vert_{\Phi_t}}\dif t
		+ c_2\int_1^2\abs{\e_{dd}(v)\big\vert_{\Phi_t}}\dif t \\
		&\leq c_3\max_{k,l}\int_1^2\abs{\e_{kl}(v)\big\vert_{\Phi_t}}\dif t
	\end{align*}
	mit einem $c_3$ nur von $w$ und $f$ abhängig.
	
	Nun kümmern wir uns um die restlichen Terme. Sei $x'$ fest. Wir erhalten auf $\tiOmega^\complement$ mit einer Transformation
	\begin{align*}
		\int_1^2\abs{\e_{kl}(v)\big\vert_{\Phi_t}}\dif t
		&= \int_1^2\abs{\e_{kl}(v)(x',x_d+tf(x'))}\dif t \\
		&= \int_{x_d+f(x_d)}^{x_d+2f(x_d)}\frac{1}{f(x_d)}\abs{\e_{kl}(v)(x',s)}\dif s
	\end{align*}
	Zusammen mit
	\begin{align*}
		\frac{1}{f(x_d)}&\leq\frac{1}{2(\psi(x')-x_d)} = \frac{1}{2D} \\
		x_d+f(x_d)&\geq\psi(x')-x_d = D \\
		x_d+2f(x_d)&\leq c_f(\psi(x')-x_d) = c_fD
	\end{align*}
	wobei wir $D\coloneqq\psi(x')-x_d$ gesetzt haben. Damit erhalten wir
	\begin{align*}
		\int_1^2\abs{\e_{kl}(v)\big\vert_{\Phi_t}}\dif t
		&= \int_{x_d+f(x_d)}^{x_d+2f(x_d)}\frac{1}{f(x_d)}\abs{\e_{kl}(v)(x',s)}\dif s \\		
		&\leq \frac{1}{2D}\int_{D}^{c_fD}\abs{\e_{kl}(v)(x',s)}\dif s \\
	\end{align*}
	Es folgt weiter mit Cauchy-Schwarz, dass
	\begin{align*}
		\left(\int_1^2\abs{\e_{kl}(v)\big\vert_{\Phi_t}}\dif t\right)^2
		&= \left(\frac{1}{2D}\int_{D}^{c_fD}\abs{\e_{kl}(v)(x',s)}\dif s\right)^2 \\
		&= \frac{1}{4D^2}\left(\int_{D}^{c_fD}\abs{s^{1/4}s^{-1/4}\e_{kl}(v)(x',s)}\dif s\right)^2 \\
		&\leq \frac{1}{4D^2}\int_{D}^{c_fD}s^{1/2}\dif s \int_{D}^{c_fD}s^{-1/2}\abs{\e_{kl}(v)(x',s)}^2\dif s \\
		&\leq \frac{1}{4D^2}\int_{0}^{c_fD}s^{1/2}\dif s \int_{D}^{c_fD}s^{-1/2}\abs{\e_{kl}(v)(x',s)}^2\dif s \\
		&= \frac{1}{4D^2}\frac{2}{3}s^{3/2}\Big\vert_{s=0}^{c_fD}\int_{D}^{c_fD}s^{-1/2}\abs{\e_{kl}(v)(x',s)}^2\dif s \\
		&= \frac{1}{4D^2}\frac{2}{3}c^{3/2}D^{3/2}\int_{D}^{c_fD}s^{-1/2}\abs{\e_{kl}(v)(x',s)}^2\dif s \\
		&= \frac{c^{3/2}}{6}D^{-1/2}\int_{D}^{c_fD}s^{-1/2}\abs{\e_{kl}(v)(x',s)}^2\dif s
	\end{align*}
	Es folgt
	\begin{align*}
		&\int_{-\infty}^{\psi(x')}\left(\int_1^2\abs{\e_{kl}(v)\big\vert_{\Phi_s}}\dif t\right)^2\dif x_d \\
		&\leq c_3\int_{-\infty}^{\psi(x')}D^{-1/2}\int_{D}^{c_fD}s^{-1/2}\abs{\e_{kl}(v)(x',s)}^2\dif s\dif x_d \\
		&\leq c\int_{\psi(x')}^{\infty}s^{-1/2}\abs{\e_{kl}(v)(x',s)}^2\int_{c^{-1}s}^sD^{-1/2}\dif x_d\dif s \\
		&\leq c\int_{\psi(x')}^{\infty}s^{-1/2}\abs{\e_{kl}(v)(x',s)}^2\int_{0}^sD^{-1/2}\dif x_d\dif s \\
		&= c\int_{\psi(x')}^{\infty}s^{-1/2}\abs{\e_{kl}(v)(x',s)}^22x_d^{1/2}\Big\vert_{x_d=0}^s\dif s \\
		&= 2c\int_{\psi(x')}^{\infty}s^{-1/2}\abs{\e_{kl}(v)(x',s)}^2s^{1/2}\dif s \\
		&= 2c\int_{\psi(x')}^{\infty}\abs{\e_{kl}(v)(x',s)}^2\dif s
	\end{align*}
	und weiter
	\begin{align*}
		\int_{\tiOmega^\complement}\left(\int_1^2\abs{\e_{kl}(v)\big\vert_{\Phi_t}}\dif t\right)^2\dif x
		&= \int_{K}\int_{-\infty}^{\Psi(x')}\left(\int_1^2\abs{\e_{kl}(v)\big\vert_{\Phi_t}}\dif t\right)^2\dif x_d\dif x' \\
		&\leq c\int_{K}\int_{\psi(x')}^{\infty}\abs{\e_{kl}(v)(x',s)}^2\dif s\dif x' \\
		&= c\int_{\Omega}\abs{\e_{kl}(v)}^2\dif x \\
		&= c\abs{\e_{kl}(v)}_{0,\Omega}^2
	\end{align*}
	Es ergibt sich also insgesamt
	\begin{align*}
		\abs{\e(Ev)}_{0,\R^d}^2-\abs{\e(v)}_{0,\Omega}^2
		&= \abs{\e(Ev)}_{0,\tiOmega^\complement}^2 \\
		&= \int_{\tiOmega^\complement}\sum_{i,j}\abs{\e_{ij}(Ev)}^2\dif x \\
		&\leq  \int_{\tiOmega^\complement}\sum_{i,j}\left(c_3\max_{k,l}\int_1^2\abs{\e_{kl}(v)\big\vert_{\Phi_t}}\dif t\right)^2\dif x \\
		&=  c_3d^2\max_{k,l}\int_{\tiOmega^\complement}\left(\int_1^2\abs{\e_{kl}(v)\big\vert_{\Phi_t}}\dif t\right)^2\dif x \\
		&\leq  cc_3d^2\max_{k,l}c\abs{\e_{kl}(v)}_{0,\Omega}^2 \\
		&\leq  c\abs{\e(v)}_{0,\Omega}^2 \\
	\end{align*}
	womit dann die Behauptung folgt.
\end{proof}

\subsubsection*{Lipschitz-Mengen}

Wir bezeichnen eine beschränkte offene Menge $\Omega\subseteq\R^d$ als Lipschitz, falls es für jeden Randpunkt $x\in\partial\Omega$ einen Radius $r>0$, eine Lipschitz-stetige Funktion $\psi\colon\R^{d-1}\to\R$ und eine affine Isometrie $A\colon\R^d\to\R^d$ gibt, so dass
\begin{align*}
	B_r(x)\cap\Omega = B_r(x)\cap A\epi(\psi)
\end{align*}

\begin{figure}[h]
\centering
\hspace*{-3cm}
% Graphic for TeX using PGF
% Title: /mnt/12CCB7B3CCB79009/Filing/Education/University/Bonn/Courses/Bachelorarbeit/Resources/LipschitzDefinition2.dia
% Creator: Dia v0.97+git
% CreationDate: Wed May 18 14:50:45 2022
% For: theo
% \usepackage{tikz}
% The following commands are not supported in PSTricks at present
% We define them conditionally, so when they are implemented,
% this pgf file will use them.
\ifx\du\undefined
  \newlength{\du}
\fi
\setlength{\du}{15\unitlength}
\begin{tikzpicture}[even odd rule]
\pgftransformxscale{0.4}
\pgftransformyscale{-0.4}
\definecolor{dialinecolor}{rgb}{0.000000, 0.000000, 0.000000}
\pgfsetstrokecolor{dialinecolor}
\pgfsetstrokeopacity{1.000000}
\definecolor{diafillcolor}{rgb}{1.000000, 1.000000, 1.000000}
\pgfsetfillcolor{diafillcolor}
\pgfsetfillopacity{1.000000}
\pgfsetlinewidth{0.050000\du}
\pgfsetdash{}{0pt}
\pgfsetbuttcap
{
\definecolor{diafillcolor}{rgb}{0.000000, 0.000000, 0.000000}
\pgfsetfillcolor{diafillcolor}
\pgfsetfillopacity{1.000000}
% was here!!!
\definecolor{dialinecolor}{rgb}{0.000000, 0.000000, 0.000000}
\pgfsetstrokecolor{dialinecolor}
\pgfsetstrokeopacity{1.000000}
\draw (-10.000000\du,-3.000000\du)--(-10.000000\du,0.000000\du);
}
\pgfsetlinewidth{0.050000\du}
\pgfsetdash{}{0pt}
\pgfsetbuttcap
{
\definecolor{diafillcolor}{rgb}{0.000000, 0.000000, 0.000000}
\pgfsetfillcolor{diafillcolor}
\pgfsetfillopacity{1.000000}
% was here!!!
\definecolor{dialinecolor}{rgb}{0.000000, 0.000000, 0.000000}
\pgfsetstrokecolor{dialinecolor}
\pgfsetstrokeopacity{1.000000}
\draw (-10.000000\du,0.000000\du)--(-2.000000\du,0.000000\du);
}
\pgfsetlinewidth{0.050000\du}
\pgfsetdash{}{0pt}
\pgfsetmiterjoin
\pgfsetbuttcap
{
\definecolor{diafillcolor}{rgb}{0.000000, 0.000000, 0.000000}
\pgfsetfillcolor{diafillcolor}
\pgfsetfillopacity{1.000000}
% was here!!!
\definecolor{dialinecolor}{rgb}{0.000000, 0.000000, 0.000000}
\pgfsetstrokecolor{dialinecolor}
\pgfsetstrokeopacity{1.000000}
\pgfpathmoveto{\pgfpoint{-2.000000\du}{-1.000000\du}}
\pgfpathcurveto{\pgfpoint{-19.000000\du}{-19.000000\du}}{\pgfpoint{19.000000\du}{-19.000000\du}}{\pgfpoint{2.000000\du}{-1.000000\du}}
\pgfusepath{stroke}
}
\pgfsetlinewidth{0.050000\du}
\pgfsetdash{}{0pt}
\pgfsetmiterjoin
\pgfsetbuttcap
{
\definecolor{diafillcolor}{rgb}{0.000000, 0.000000, 0.000000}
\pgfsetfillcolor{diafillcolor}
\pgfsetfillopacity{1.000000}
% was here!!!
\definecolor{dialinecolor}{rgb}{0.000000, 0.000000, 0.000000}
\pgfsetstrokecolor{dialinecolor}
\pgfsetstrokeopacity{1.000000}
\pgfpathmoveto{\pgfpoint{-4.400000\du}{-2.200000\du}}
\pgfpathcurveto{\pgfpoint{-26.000000\du}{-20.000000\du}}{\pgfpoint{26.000000\du}{-20.000000\du}}{\pgfpoint{4.400000\du}{-2.200000\du}}
\pgfusepath{stroke}
}
\pgfsetlinewidth{0.050000\du}
\pgfsetdash{}{0pt}
\pgfsetmiterjoin
\pgfsetbuttcap
{
\definecolor{diafillcolor}{rgb}{0.000000, 0.000000, 0.000000}
\pgfsetfillcolor{diafillcolor}
\pgfsetfillopacity{1.000000}
% was here!!!
\definecolor{dialinecolor}{rgb}{0.000000, 0.000000, 0.000000}
\pgfsetstrokecolor{dialinecolor}
\pgfsetstrokeopacity{1.000000}
\pgfpathmoveto{\pgfpoint{-4.400000\du}{-2.200000\du}}
\pgfpathcurveto{\pgfpoint{-3.200000\du}{-1.000000\du}}{\pgfpoint{-8.500000\du}{-1.500000\du}}{\pgfpoint{-10.000000\du}{-3.000000\du}}
\pgfusepath{stroke}
}
\pgfsetlinewidth{0.050000\du}
\pgfsetdash{}{0pt}
\pgfsetbuttcap
{
\definecolor{diafillcolor}{rgb}{0.000000, 0.000000, 0.000000}
\pgfsetfillcolor{diafillcolor}
\pgfsetfillopacity{1.000000}
% was here!!!
\definecolor{dialinecolor}{rgb}{0.000000, 0.000000, 0.000000}
\pgfsetstrokecolor{dialinecolor}
\pgfsetstrokeopacity{1.000000}
\draw (2.000000\du,0.000000\du)--(10.000000\du,0.000000\du);
}
\pgfsetlinewidth{0.050000\du}
\pgfsetdash{}{0pt}
\pgfsetbuttcap
{
\definecolor{diafillcolor}{rgb}{0.000000, 0.000000, 0.000000}
\pgfsetfillcolor{diafillcolor}
\pgfsetfillopacity{1.000000}
% was here!!!
\definecolor{dialinecolor}{rgb}{0.000000, 0.000000, 0.000000}
\pgfsetstrokecolor{dialinecolor}
\pgfsetstrokeopacity{1.000000}
\draw (10.000000\du,0.000000\du)--(10.000000\du,-3.000000\du);
}
\pgfsetlinewidth{0.050000\du}
\pgfsetdash{}{0pt}
\pgfsetmiterjoin
\pgfsetbuttcap
{
\definecolor{diafillcolor}{rgb}{0.000000, 0.000000, 0.000000}
\pgfsetfillcolor{diafillcolor}
\pgfsetfillopacity{1.000000}
% was here!!!
\definecolor{dialinecolor}{rgb}{0.000000, 0.000000, 0.000000}
\pgfsetstrokecolor{dialinecolor}
\pgfsetstrokeopacity{1.000000}
\pgfpathmoveto{\pgfpoint{4.400000\du}{-2.200000\du}}
\pgfpathcurveto{\pgfpoint{3.400000\du}{-1.000000\du}}{\pgfpoint{9.000000\du}{-2.000000\du}}{\pgfpoint{10.000000\du}{-3.000000\du}}
\pgfusepath{stroke}
}
\pgfsetlinewidth{0.050000\du}
\pgfsetdash{}{0pt}
\pgfsetbuttcap
{
\definecolor{diafillcolor}{rgb}{0.000000, 0.000000, 0.000000}
\pgfsetfillcolor{diafillcolor}
\pgfsetfillopacity{1.000000}
% was here!!!
\definecolor{dialinecolor}{rgb}{0.000000, 0.000000, 0.000000}
\pgfsetstrokecolor{dialinecolor}
\pgfsetstrokeopacity{1.000000}
\draw (-2.000000\du,-1.000000\du)--(-2.000000\du,0.000000\du);
}
\pgfsetlinewidth{0.050000\du}
\pgfsetdash{}{0pt}
\pgfsetbuttcap
{
\definecolor{diafillcolor}{rgb}{0.000000, 0.000000, 0.000000}
\pgfsetfillcolor{diafillcolor}
\pgfsetfillopacity{1.000000}
% was here!!!
\definecolor{dialinecolor}{rgb}{0.000000, 0.000000, 0.000000}
\pgfsetstrokecolor{dialinecolor}
\pgfsetstrokeopacity{1.000000}
\draw (2.000000\du,-1.000000\du)--(2.000000\du,0.000000\du);
}
\pgfsetlinewidth{0.050000\du}
\pgfsetdash{}{0pt}
\definecolor{dialinecolor}{rgb}{0.000000, 0.000000, 0.000000}
\pgfsetstrokecolor{dialinecolor}
\pgfsetstrokeopacity{1.000000}
\pgfpathellipse{\pgfpoint{10.000000\du}{-3.000000\du}}{\pgfpoint{2.500000\du}{0\du}}{\pgfpoint{0\du}{2.500000\du}}
\pgfusepath{stroke}
\pgfsetlinewidth{0.050000\du}
\pgfsetdash{}{0pt}
\pgfsetbuttcap
{
\definecolor{diafillcolor}{rgb}{0.000000, 0.000000, 0.000000}
\pgfsetfillcolor{diafillcolor}
\pgfsetfillopacity{1.000000}
% was here!!!
\definecolor{dialinecolor}{rgb}{0.000000, 0.000000, 0.000000}
\pgfsetstrokecolor{dialinecolor}
\pgfsetstrokeopacity{1.000000}
\draw (9.991661\du,-1.104192\du)--(9.461355\du,-0.547371\du);
}
\pgfsetlinewidth{0.050000\du}
\pgfsetdash{}{0pt}
\pgfsetbuttcap
{
\definecolor{diafillcolor}{rgb}{0.000000, 0.000000, 0.000000}
\pgfsetfillcolor{diafillcolor}
\pgfsetfillopacity{1.000000}
% was here!!!
\definecolor{dialinecolor}{rgb}{0.000000, 0.000000, 0.000000}
\pgfsetstrokecolor{dialinecolor}
\pgfsetstrokeopacity{1.000000}
\draw (10.000499\du,-1.828944\du)--(8.913372\du,-0.768332\du);
}
\pgfsetlinewidth{0.050000\du}
\pgfsetdash{}{0pt}
\pgfsetbuttcap
{
\definecolor{diafillcolor}{rgb}{0.000000, 0.000000, 0.000000}
\pgfsetfillcolor{diafillcolor}
\pgfsetfillopacity{1.000000}
% was here!!!
\definecolor{dialinecolor}{rgb}{0.000000, 0.000000, 0.000000}
\pgfsetstrokecolor{dialinecolor}
\pgfsetstrokeopacity{1.000000}
\draw (10.000499\du,-2.606726\du)--(8.356551\du,-1.086515\du);
}
\pgfsetlinewidth{0.050000\du}
\pgfsetdash{}{0pt}
\pgfsetbuttcap
{
\definecolor{diafillcolor}{rgb}{0.000000, 0.000000, 0.000000}
\pgfsetfillcolor{diafillcolor}
\pgfsetfillopacity{1.000000}
% was here!!!
\definecolor{dialinecolor}{rgb}{0.000000, 0.000000, 0.000000}
\pgfsetstrokecolor{dialinecolor}
\pgfsetstrokeopacity{1.000000}
\draw (8.984079\du,-2.438796\du)--(7.967660\du,-1.572629\du);
}
\pgfsetlinewidth{0.050000\du}
\pgfsetdash{}{0pt}
\pgfsetbuttcap
{
\definecolor{diafillcolor}{rgb}{0.000000, 0.000000, 0.000000}
\pgfsetfillcolor{diafillcolor}
\pgfsetfillopacity{1.000000}
% was here!!!
\pgfsetarrowsend{latex}
\definecolor{dialinecolor}{rgb}{0.000000, 0.000000, 0.000000}
\pgfsetstrokecolor{dialinecolor}
\pgfsetstrokeopacity{1.000000}
\draw (20.000000\du,-3.000000\du)--(35.000000\du,-3.000000\du);
}
\pgfsetlinewidth{0.050000\du}
\pgfsetdash{}{0pt}
\pgfsetbuttcap
{
\definecolor{diafillcolor}{rgb}{0.000000, 0.000000, 0.000000}
\pgfsetfillcolor{diafillcolor}
\pgfsetfillopacity{1.000000}
% was here!!!
\pgfsetarrowsend{latex}
\definecolor{dialinecolor}{rgb}{0.000000, 0.000000, 0.000000}
\pgfsetstrokecolor{dialinecolor}
\pgfsetstrokeopacity{1.000000}
\draw (20.000000\du,-3.000000\du)--(20.000000\du,-15.000000\du);
}
\pgfsetlinewidth{0.050000\du}
\pgfsetdash{}{0pt}
\pgfsetbuttcap
{
\definecolor{diafillcolor}{rgb}{0.000000, 0.000000, 0.000000}
\pgfsetfillcolor{diafillcolor}
\pgfsetfillopacity{1.000000}
% was here!!!
\definecolor{dialinecolor}{rgb}{0.000000, 0.000000, 0.000000}
\pgfsetstrokecolor{dialinecolor}
\pgfsetstrokeopacity{1.000000}
\draw (27.000000\du,-7.000000\du)--(32.000000\du,-13.000000\du);
}
\pgfsetlinewidth{0.050000\du}
\pgfsetdash{}{0pt}
\pgfsetbuttcap
{
\definecolor{diafillcolor}{rgb}{0.000000, 0.000000, 0.000000}
\pgfsetfillcolor{diafillcolor}
\pgfsetfillopacity{1.000000}
% was here!!!
\definecolor{dialinecolor}{rgb}{0.000000, 0.000000, 0.000000}
\pgfsetstrokecolor{dialinecolor}
\pgfsetstrokeopacity{1.000000}
\draw (32.000000\du,-13.000000\du)--(36.000000\du,-13.000000\du);
}
\pgfsetlinewidth{0.050000\du}
\pgfsetdash{}{0pt}
\pgfsetmiterjoin
\pgfsetbuttcap
{
\definecolor{diafillcolor}{rgb}{0.000000, 0.000000, 0.000000}
\pgfsetfillcolor{diafillcolor}
\pgfsetfillopacity{1.000000}
% was here!!!
\definecolor{dialinecolor}{rgb}{0.000000, 0.000000, 0.000000}
\pgfsetstrokecolor{dialinecolor}
\pgfsetstrokeopacity{1.000000}
\pgfpathmoveto{\pgfpoint{19.000000\du}{-11.000000\du}}
\pgfpathcurveto{\pgfpoint{20.000000\du}{-12.000000\du}}{\pgfpoint{25.042830\du}{-11.470666\du}}{\pgfpoint{27.000000\du}{-7.000000\du}}
\pgfusepath{stroke}
}
\pgfsetlinewidth{0.050000\du}
\pgfsetdash{}{0pt}
\definecolor{dialinecolor}{rgb}{0.000000, 0.000000, 0.000000}
\pgfsetstrokecolor{dialinecolor}
\pgfsetstrokeopacity{1.000000}
\pgfpathellipse{\pgfpoint{26.991696\du}{-7.014620\du}}{\pgfpoint{2.500000\du}{0\du}}{\pgfpoint{0\du}{2.500000\du}}
\pgfusepath{stroke}
\pgfsetlinewidth{0.050000\du}
\pgfsetdash{}{0pt}
\pgfsetbuttcap
{
\definecolor{diafillcolor}{rgb}{0.000000, 0.000000, 0.000000}
\pgfsetfillcolor{diafillcolor}
\pgfsetfillopacity{1.000000}
% was here!!!
\definecolor{dialinecolor}{rgb}{0.000000, 0.000000, 0.000000}
\pgfsetstrokecolor{dialinecolor}
\pgfsetstrokeopacity{1.000000}
\draw (27.828732\du,-8.001710\du)--(27.834982\du,-9.370382\du);
}
\pgfsetlinewidth{0.050000\du}
\pgfsetdash{}{0pt}
\pgfsetbuttcap
{
\definecolor{diafillcolor}{rgb}{0.000000, 0.000000, 0.000000}
\pgfsetfillcolor{diafillcolor}
\pgfsetfillopacity{1.000000}
% was here!!!
\definecolor{dialinecolor}{rgb}{0.000000, 0.000000, 0.000000}
\pgfsetstrokecolor{dialinecolor}
\pgfsetstrokeopacity{1.000000}
\draw (28.366202\du,-8.601676\du)--(28.366202\du,-9.107897\du);
}
\pgfsetlinewidth{0.050000\du}
\pgfsetdash{}{0pt}
\pgfsetbuttcap
{
\definecolor{diafillcolor}{rgb}{0.000000, 0.000000, 0.000000}
\pgfsetfillcolor{diafillcolor}
\pgfsetfillopacity{1.000000}
% was here!!!
\definecolor{dialinecolor}{rgb}{0.000000, 0.000000, 0.000000}
\pgfsetstrokecolor{dialinecolor}
\pgfsetstrokeopacity{1.000000}
\draw (27.328761\du,-7.432993\du)--(27.316262\du,-9.514124\du);
}
\pgfsetlinewidth{0.050000\du}
\pgfsetdash{}{0pt}
\pgfsetbuttcap
{
\definecolor{diafillcolor}{rgb}{0.000000, 0.000000, 0.000000}
\pgfsetfillcolor{diafillcolor}
\pgfsetfillopacity{1.000000}
% was here!!!
\definecolor{dialinecolor}{rgb}{0.000000, 0.000000, 0.000000}
\pgfsetstrokecolor{dialinecolor}
\pgfsetstrokeopacity{1.000000}
\draw (26.779548\du,-7.493632\du)--(26.761872\du,-9.482262\du);
}
\pgfsetlinewidth{0.050000\du}
\pgfsetdash{}{0pt}
\pgfsetbuttcap
{
\definecolor{diafillcolor}{rgb}{0.000000, 0.000000, 0.000000}
\pgfsetfillcolor{diafillcolor}
\pgfsetfillopacity{1.000000}
% was here!!!
\definecolor{dialinecolor}{rgb}{0.000000, 0.000000, 0.000000}
\pgfsetstrokecolor{dialinecolor}
\pgfsetstrokeopacity{1.000000}
\draw (26.266924\du,-8.306760\du)--(26.266924\du,-9.429232\du);
}
\pgfsetlinewidth{0.050000\du}
\pgfsetdash{}{0pt}
\pgfsetbuttcap
{
\definecolor{diafillcolor}{rgb}{0.000000, 0.000000, 0.000000}
\pgfsetfillcolor{diafillcolor}
\pgfsetfillopacity{1.000000}
% was here!!!
\definecolor{dialinecolor}{rgb}{0.000000, 0.000000, 0.000000}
\pgfsetstrokecolor{dialinecolor}
\pgfsetstrokeopacity{1.000000}
\draw (25.816168\du,-9.234788\du)--(25.825006\du,-8.898930\du);
}
% setfont left to latex
\definecolor{dialinecolor}{rgb}{0.000000, 0.000000, 0.000000}
\pgfsetstrokecolor{dialinecolor}
\pgfsetstrokeopacity{1.000000}
\definecolor{diafillcolor}{rgb}{0.000000, 0.000000, 0.000000}
\pgfsetfillcolor{diafillcolor}
\pgfsetfillopacity{1.000000}
\node[anchor=base west,inner sep=0pt,outer sep=0pt,color=dialinecolor] at (4.594568\du,-17.940543\du){$\Omega$};
\pgfsetlinewidth{0.050000\du}
\pgfsetdash{}{0pt}
\pgfsetmiterjoin
\pgfsetbuttcap
{
\definecolor{diafillcolor}{rgb}{0.000000, 0.000000, 0.000000}
\pgfsetfillcolor{diafillcolor}
\pgfsetfillopacity{1.000000}
% was here!!!
\definecolor{dialinecolor}{rgb}{0.000000, 0.000000, 0.000000}
\pgfsetstrokecolor{dialinecolor}
\pgfsetstrokeopacity{1.000000}
\pgfpathmoveto{\pgfpoint{4.000000\du}{-14.200000\du}}
\pgfpathcurveto{\pgfpoint{3.800000\du}{-15.600000\du}}{\pgfpoint{5.000000\du}{-16.200000\du}}{\pgfpoint{5.000000\du}{-17.200000\du}}
\pgfusepath{stroke}
}
\pgfsetlinewidth{0.050000\du}
\pgfsetdash{}{0pt}
\pgfsetmiterjoin
\pgfsetbuttcap
{
\definecolor{diafillcolor}{rgb}{0.000000, 0.000000, 0.000000}
\pgfsetfillcolor{diafillcolor}
\pgfsetfillopacity{1.000000}
% was here!!!
\pgfsetarrowsend{to}
\definecolor{dialinecolor}{rgb}{0.000000, 0.000000, 0.000000}
\pgfsetstrokecolor{dialinecolor}
\pgfsetstrokeopacity{1.000000}
\pgfpathmoveto{\pgfpoint{13.000000\du}{-1.000000\du}}
\pgfpathcurveto{\pgfpoint{16.000000\du}{0.000000\du}}{\pgfpoint{20.000000\du}{1.000000\du}}{\pgfpoint{24.000000\du}{-4.000000\du}}
\pgfusepath{stroke}
}
% setfont left to latex
\definecolor{dialinecolor}{rgb}{0.000000, 0.000000, 0.000000}
\pgfsetstrokecolor{dialinecolor}
\pgfsetstrokeopacity{1.000000}
\definecolor{diafillcolor}{rgb}{0.000000, 0.000000, 0.000000}
\pgfsetfillcolor{diafillcolor}
\pgfsetfillopacity{1.000000}
\node[anchor=base west,inner sep=0pt,outer sep=0pt,color=dialinecolor] at (16.589886\du,-1.518826\du){$A^{-1}$};
% setfont left to latex
\definecolor{dialinecolor}{rgb}{0.000000, 0.000000, 0.000000}
\pgfsetstrokecolor{dialinecolor}
\pgfsetstrokeopacity{1.000000}
\definecolor{diafillcolor}{rgb}{0.000000, 0.000000, 0.000000}
\pgfsetfillcolor{diafillcolor}
\pgfsetfillopacity{1.000000}
\node[anchor=base west,inner sep=0pt,outer sep=0pt,color=dialinecolor] at (11.060406\du,-6.394281\du){$B_r(x)$};
% setfont left to latex
\definecolor{dialinecolor}{rgb}{0.000000, 0.000000, 0.000000}
\pgfsetstrokecolor{dialinecolor}
\pgfsetstrokeopacity{1.000000}
\definecolor{diafillcolor}{rgb}{0.000000, 0.000000, 0.000000}
\pgfsetfillcolor{diafillcolor}
\pgfsetfillopacity{1.000000}
\node[anchor=base west,inner sep=0pt,outer sep=0pt,color=dialinecolor] at (30.654024\du,-6.940543\du){$B_r(x)$};
% setfont left to latex
\definecolor{dialinecolor}{rgb}{0.000000, 0.000000, 0.000000}
\pgfsetstrokecolor{dialinecolor}
\pgfsetstrokeopacity{1.000000}
\definecolor{diafillcolor}{rgb}{0.000000, 0.000000, 0.000000}
\pgfsetfillcolor{diafillcolor}
\pgfsetfillopacity{1.000000}
\node[anchor=base west,inner sep=0pt,outer sep=0pt,color=dialinecolor] at (36.151163\du,-3.183615\du){$x'$};
% setfont left to latex
\definecolor{dialinecolor}{rgb}{0.000000, 0.000000, 0.000000}
\pgfsetstrokecolor{dialinecolor}
\pgfsetstrokeopacity{1.000000}
\definecolor{diafillcolor}{rgb}{0.000000, 0.000000, 0.000000}
\pgfsetfillcolor{diafillcolor}
\pgfsetfillopacity{1.000000}
\node[anchor=base west,inner sep=0pt,outer sep=0pt,color=dialinecolor] at (19.345976\du,-15.821630\du){$\psi(x')$};
% setfont left to latex
\definecolor{dialinecolor}{rgb}{0.000000, 0.000000, 0.000000}
\pgfsetstrokecolor{dialinecolor}
\pgfsetstrokeopacity{1.000000}
\definecolor{diafillcolor}{rgb}{0.000000, 0.000000, 0.000000}
\pgfsetfillcolor{diafillcolor}
\pgfsetfillopacity{1.000000}
\node[anchor=base west,inner sep=0pt,outer sep=0pt,color=dialinecolor] at (25.000000\du,-12.000000\du){$\epi(\psi)$};
\end{tikzpicture}

\caption{Visualisierung eines Lipschitz-Gebiets}
\end{figure}


\begin{lemma}[Zerlegung der Eins]
Sei $B_0\coloneqq \tiB_0\coloneqq\Omega\subseteq\R^d$ Lipschitz. Dann gibt es
\begin{itemize}
	\item
	Randpunkte $x^{1},\dots,x^{N}\in\partial\Omega$
	\item
	eine Konstante $M>0$
	\item
	Bälle $B_i=B_{r_i}(x^{i})$ und $\tiB_i\coloneqq B_{\tir_i}(x^i)$ mit Radius $\tir_i\coloneqq $
	\item
	Lipschitz stetige Funktionen $\psi_i\colon \R^{d-1}\to\R$ mit $\Lip(\psi_i)\leq M$ und affine Isometrien $A_i\colon \R^d\to\R^d$
	\item
	Abschneidefunktionen $\theta_i\in C_c^\infty(B_i)$ (muss hier $\geq0$ gefordert werden?) und $\theta_0\in C_c^\infty(\Omega)$
\end{itemize}
so dass
\begin{itemize}
	\item $B_i\cap\Omega = B_i\cap A_i\epi(\psi_i)$
	\item $\partial\Omega\subseteq\bigcup_i\tiB_i$
	\item $\supp\theta_i\subseteq \tiB_i$
	\item $\sum_i\theta_i=1$ auf $\overline{\Omega}$
\end{itemize}
\end{lemma}
\begin{proof}
	Siehe Theorem 7.4 in \cite{Con-2021}
	TODO: Das reicht nicht aus
\end{proof}
\begin{figure}[h]
\centering
\hspace*{-2cm}
% Graphic for TeX using PGF
% Title: /mnt/12CCB7B3CCB79009/Filing/Education/University/Bonn/Courses/Bachelorarbeit/Resources/PartitionOfUnity2.dia
% Creator: Dia v0.97+git
% CreationDate: Wed May 18 15:01:20 2022
% For: theo
% \usepackage{tikz}
% The following commands are not supported in PSTricks at present
% We define them conditionally, so when they are implemented,
% this pgf file will use them.
\ifx\du\undefined
  \newlength{\du}
\fi
\setlength{\du}{15\unitlength}
\begin{tikzpicture}[even odd rule]
\pgftransformxscale{0.5}
\pgftransformyscale{-0.5}
\definecolor{dialinecolor}{rgb}{0.000000, 0.000000, 0.000000}
\pgfsetstrokecolor{dialinecolor}
\pgfsetstrokeopacity{1.000000}
\definecolor{diafillcolor}{rgb}{1.000000, 1.000000, 1.000000}
\pgfsetfillcolor{diafillcolor}
\pgfsetfillopacity{1.000000}
\pgfsetlinewidth{0.050000\du}
\pgfsetdash{}{0pt}
\pgfsetbuttcap
{
\definecolor{diafillcolor}{rgb}{0.000000, 0.000000, 0.000000}
\pgfsetfillcolor{diafillcolor}
\pgfsetfillopacity{1.000000}
% was here!!!
\definecolor{dialinecolor}{rgb}{0.000000, 0.000000, 0.000000}
\pgfsetstrokecolor{dialinecolor}
\pgfsetstrokeopacity{1.000000}
\draw (5.390355\du,18.169596\du)--(5.390355\du,21.169596\du);
}
\pgfsetlinewidth{0.050000\du}
\pgfsetdash{}{0pt}
\pgfsetbuttcap
{
\definecolor{diafillcolor}{rgb}{0.000000, 0.000000, 0.000000}
\pgfsetfillcolor{diafillcolor}
\pgfsetfillopacity{1.000000}
% was here!!!
\definecolor{dialinecolor}{rgb}{0.000000, 0.000000, 0.000000}
\pgfsetstrokecolor{dialinecolor}
\pgfsetstrokeopacity{1.000000}
\draw (5.390355\du,21.169596\du)--(13.390355\du,21.169596\du);
}
\pgfsetlinewidth{0.050000\du}
\pgfsetdash{}{0pt}
\pgfsetmiterjoin
\pgfsetbuttcap
{
\definecolor{diafillcolor}{rgb}{0.000000, 0.000000, 0.000000}
\pgfsetfillcolor{diafillcolor}
\pgfsetfillopacity{1.000000}
% was here!!!
\definecolor{dialinecolor}{rgb}{0.000000, 0.000000, 0.000000}
\pgfsetstrokecolor{dialinecolor}
\pgfsetstrokeopacity{1.000000}
\pgfpathmoveto{\pgfpoint{13.390355\du}{20.169596\du}}
\pgfpathcurveto{\pgfpoint{-3.609645\du}{2.169596\du}}{\pgfpoint{34.390355\du}{2.169596\du}}{\pgfpoint{17.390355\du}{20.169596\du}}
\pgfusepath{stroke}
}
\pgfsetlinewidth{0.050000\du}
\pgfsetdash{}{0pt}
\pgfsetmiterjoin
\pgfsetbuttcap
{
\definecolor{diafillcolor}{rgb}{0.000000, 0.000000, 0.000000}
\pgfsetfillcolor{diafillcolor}
\pgfsetfillopacity{1.000000}
% was here!!!
\definecolor{dialinecolor}{rgb}{0.000000, 0.000000, 0.000000}
\pgfsetstrokecolor{dialinecolor}
\pgfsetstrokeopacity{1.000000}
\pgfpathmoveto{\pgfpoint{10.990355\du}{18.969596\du}}
\pgfpathcurveto{\pgfpoint{-10.609645\du}{1.169596\du}}{\pgfpoint{41.390355\du}{1.169596\du}}{\pgfpoint{19.790355\du}{18.969596\du}}
\pgfusepath{stroke}
}
\pgfsetlinewidth{0.050000\du}
\pgfsetdash{}{0pt}
\pgfsetmiterjoin
\pgfsetbuttcap
{
\definecolor{diafillcolor}{rgb}{0.000000, 0.000000, 0.000000}
\pgfsetfillcolor{diafillcolor}
\pgfsetfillopacity{1.000000}
% was here!!!
\definecolor{dialinecolor}{rgb}{0.000000, 0.000000, 0.000000}
\pgfsetstrokecolor{dialinecolor}
\pgfsetstrokeopacity{1.000000}
\pgfpathmoveto{\pgfpoint{10.990355\du}{18.969596\du}}
\pgfpathcurveto{\pgfpoint{12.190355\du}{20.169596\du}}{\pgfpoint{6.890355\du}{19.669596\du}}{\pgfpoint{5.390355\du}{18.169596\du}}
\pgfusepath{stroke}
}
\pgfsetlinewidth{0.050000\du}
\pgfsetdash{}{0pt}
\pgfsetbuttcap
{
\definecolor{diafillcolor}{rgb}{0.000000, 0.000000, 0.000000}
\pgfsetfillcolor{diafillcolor}
\pgfsetfillopacity{1.000000}
% was here!!!
\definecolor{dialinecolor}{rgb}{0.000000, 0.000000, 0.000000}
\pgfsetstrokecolor{dialinecolor}
\pgfsetstrokeopacity{1.000000}
\draw (17.390355\du,21.169596\du)--(25.390355\du,21.169596\du);
}
\pgfsetlinewidth{0.050000\du}
\pgfsetdash{}{0pt}
\pgfsetbuttcap
{
\definecolor{diafillcolor}{rgb}{0.000000, 0.000000, 0.000000}
\pgfsetfillcolor{diafillcolor}
\pgfsetfillopacity{1.000000}
% was here!!!
\definecolor{dialinecolor}{rgb}{0.000000, 0.000000, 0.000000}
\pgfsetstrokecolor{dialinecolor}
\pgfsetstrokeopacity{1.000000}
\draw (25.390355\du,21.169596\du)--(25.390355\du,18.169596\du);
}
\pgfsetlinewidth{0.050000\du}
\pgfsetdash{}{0pt}
\pgfsetmiterjoin
\pgfsetbuttcap
{
\definecolor{diafillcolor}{rgb}{0.000000, 0.000000, 0.000000}
\pgfsetfillcolor{diafillcolor}
\pgfsetfillopacity{1.000000}
% was here!!!
\definecolor{dialinecolor}{rgb}{0.000000, 0.000000, 0.000000}
\pgfsetstrokecolor{dialinecolor}
\pgfsetstrokeopacity{1.000000}
\pgfpathmoveto{\pgfpoint{19.790355\du}{18.969596\du}}
\pgfpathcurveto{\pgfpoint{18.790355\du}{20.169596\du}}{\pgfpoint{24.390355\du}{19.169596\du}}{\pgfpoint{25.390355\du}{18.169596\du}}
\pgfusepath{stroke}
}
\pgfsetlinewidth{0.050000\du}
\pgfsetdash{}{0pt}
\pgfsetbuttcap
{
\definecolor{diafillcolor}{rgb}{0.000000, 0.000000, 0.000000}
\pgfsetfillcolor{diafillcolor}
\pgfsetfillopacity{1.000000}
% was here!!!
\definecolor{dialinecolor}{rgb}{0.000000, 0.000000, 0.000000}
\pgfsetstrokecolor{dialinecolor}
\pgfsetstrokeopacity{1.000000}
\draw (13.390355\du,20.169596\du)--(13.390355\du,21.169596\du);
}
\pgfsetlinewidth{0.050000\du}
\pgfsetdash{}{0pt}
\pgfsetbuttcap
{
\definecolor{diafillcolor}{rgb}{0.000000, 0.000000, 0.000000}
\pgfsetfillcolor{diafillcolor}
\pgfsetfillopacity{1.000000}
% was here!!!
\definecolor{dialinecolor}{rgb}{0.000000, 0.000000, 0.000000}
\pgfsetstrokecolor{dialinecolor}
\pgfsetstrokeopacity{1.000000}
\draw (17.390355\du,20.169596\du)--(17.390355\du,21.169596\du);
}
% setfont left to latex
\definecolor{dialinecolor}{rgb}{0.000000, 0.000000, 0.000000}
\pgfsetstrokecolor{dialinecolor}
\pgfsetstrokeopacity{1.000000}
\definecolor{diafillcolor}{rgb}{0.000000, 0.000000, 0.000000}
\pgfsetfillcolor{diafillcolor}
\pgfsetfillopacity{1.000000}
\node[anchor=base west,inner sep=0pt,outer sep=0pt,color=dialinecolor] at (19.984923\du,3.229053\du){$\Omega$};
\pgfsetlinewidth{0.050000\du}
\pgfsetdash{}{0pt}
\pgfsetmiterjoin
\pgfsetbuttcap
{
\definecolor{diafillcolor}{rgb}{0.000000, 0.000000, 0.000000}
\pgfsetfillcolor{diafillcolor}
\pgfsetfillopacity{1.000000}
% was here!!!
\definecolor{dialinecolor}{rgb}{0.000000, 0.000000, 0.000000}
\pgfsetstrokecolor{dialinecolor}
\pgfsetstrokeopacity{1.000000}
\pgfpathmoveto{\pgfpoint{19.390355\du}{6.969596\du}}
\pgfpathcurveto{\pgfpoint{19.190355\du}{5.569596\du}}{\pgfpoint{20.390355\du}{4.969596\du}}{\pgfpoint{20.390355\du}{3.969596\du}}
\pgfusepath{stroke}
}
\pgfsetlinewidth{0.050000\du}
\pgfsetdash{}{0pt}
\definecolor{dialinecolor}{rgb}{0.000000, 0.000000, 0.000000}
\pgfsetstrokecolor{dialinecolor}
\pgfsetstrokeopacity{1.000000}
\pgfpathellipse{\pgfpoint{25.000000\du}{10.000000\du}}{\pgfpoint{2.000000\du}{0\du}}{\pgfpoint{0\du}{2.000000\du}}
\pgfusepath{stroke}
\pgfsetlinewidth{0.050000\du}
\pgfsetdash{}{0pt}
\definecolor{dialinecolor}{rgb}{0.000000, 0.000000, 0.000000}
\pgfsetstrokecolor{dialinecolor}
\pgfsetstrokeopacity{1.000000}
\pgfpathellipse{\pgfpoint{24.868191\du}{12.772906\du}}{\pgfpoint{2.000000\du}{0\du}}{\pgfpoint{0\du}{2.000000\du}}
\pgfusepath{stroke}
\pgfsetlinewidth{0.050000\du}
\pgfsetdash{}{0pt}
\definecolor{dialinecolor}{rgb}{0.000000, 0.000000, 0.000000}
\pgfsetstrokecolor{dialinecolor}
\pgfsetstrokeopacity{1.000000}
\pgfpathellipse{\pgfpoint{23.720693\du}{14.877902\du}}{\pgfpoint{2.000000\du}{0\du}}{\pgfpoint{0\du}{2.000000\du}}
\pgfusepath{stroke}
\pgfsetlinewidth{0.050000\du}
\pgfsetdash{}{0pt}
\pgfsetbuttcap
{
\definecolor{diafillcolor}{rgb}{0.000000, 0.000000, 0.000000}
\pgfsetfillcolor{diafillcolor}
\pgfsetfillopacity{1.000000}
% was here!!!
\definecolor{dialinecolor}{rgb}{0.000000, 0.000000, 0.000000}
\pgfsetstrokecolor{dialinecolor}
\pgfsetstrokeopacity{1.000000}
\draw (24.339681\du,8.768027\du)--(23.151920\du,9.230518\du);
}
\pgfsetlinewidth{0.050000\du}
\pgfsetdash{}{0pt}
\pgfsetbuttcap
{
\definecolor{diafillcolor}{rgb}{0.000000, 0.000000, 0.000000}
\pgfsetfillcolor{diafillcolor}
\pgfsetfillopacity{1.000000}
% was here!!!
\definecolor{dialinecolor}{rgb}{0.000000, 0.000000, 0.000000}
\pgfsetstrokecolor{dialinecolor}
\pgfsetstrokeopacity{1.000000}
\draw (24.686549\du,9.241029\du)--(23.004763\du,9.924254\du);
}
\pgfsetlinewidth{0.050000\du}
\pgfsetdash{}{0pt}
\pgfsetbuttcap
{
\definecolor{diafillcolor}{rgb}{0.000000, 0.000000, 0.000000}
\pgfsetfillcolor{diafillcolor}
\pgfsetfillopacity{1.000000}
% was here!!!
\definecolor{dialinecolor}{rgb}{0.000000, 0.000000, 0.000000}
\pgfsetstrokecolor{dialinecolor}
\pgfsetstrokeopacity{1.000000}
\draw (24.949328\du,9.798120\du)--(23.078342\du,10.502368\du);
}
\pgfsetlinewidth{0.050000\du}
\pgfsetdash{}{0pt}
\pgfsetbuttcap
{
\definecolor{diafillcolor}{rgb}{0.000000, 0.000000, 0.000000}
\pgfsetfillcolor{diafillcolor}
\pgfsetfillopacity{1.000000}
% was here!!!
\definecolor{dialinecolor}{rgb}{0.000000, 0.000000, 0.000000}
\pgfsetstrokecolor{dialinecolor}
\pgfsetstrokeopacity{1.000000}
\draw (25.096484\du,10.355212\du)--(23.299076\du,11.101504\du);
}
\pgfsetlinewidth{0.050000\du}
\pgfsetdash{}{0pt}
\pgfsetbuttcap
{
\definecolor{diafillcolor}{rgb}{0.000000, 0.000000, 0.000000}
\pgfsetfillcolor{diafillcolor}
\pgfsetfillopacity{1.000000}
% was here!!!
\definecolor{dialinecolor}{rgb}{0.000000, 0.000000, 0.000000}
\pgfsetstrokecolor{dialinecolor}
\pgfsetstrokeopacity{1.000000}
\draw (25.159551\du,10.912303\du)--(23.709011\du,11.542973\du);
}
\pgfsetlinewidth{0.050000\du}
\pgfsetdash{}{0pt}
\pgfsetbuttcap
{
\definecolor{diafillcolor}{rgb}{0.000000, 0.000000, 0.000000}
\pgfsetfillcolor{diafillcolor}
\pgfsetfillopacity{1.000000}
% was here!!!
\definecolor{dialinecolor}{rgb}{0.000000, 0.000000, 0.000000}
\pgfsetstrokecolor{dialinecolor}
\pgfsetstrokeopacity{1.000000}
\draw (25.117506\du,11.521950\du)--(24.245080\du,11.855680\du);
}
\pgfsetlinewidth{0.050000\du}
\pgfsetdash{}{0pt}
\pgfsetbuttcap
{
\definecolor{diafillcolor}{rgb}{0.000000, 0.000000, 0.000000}
\pgfsetfillcolor{diafillcolor}
\pgfsetfillopacity{1.000000}
% was here!!!
\definecolor{dialinecolor}{rgb}{0.000000, 0.000000, 0.000000}
\pgfsetstrokecolor{dialinecolor}
\pgfsetstrokeopacity{1.000000}
\draw (23.025786\du,13.558487\du)--(24.255591\du,14.042001\du);
}
\pgfsetlinewidth{0.050000\du}
\pgfsetdash{}{0pt}
\pgfsetbuttcap
{
\definecolor{diafillcolor}{rgb}{0.000000, 0.000000, 0.000000}
\pgfsetfillcolor{diafillcolor}
\pgfsetfillopacity{1.000000}
% was here!!!
\definecolor{dialinecolor}{rgb}{0.000000, 0.000000, 0.000000}
\pgfsetstrokecolor{dialinecolor}
\pgfsetstrokeopacity{1.000000}
\draw (22.857607\du,12.854240\du)--(24.570926\du,13.547976\du);
}
\pgfsetlinewidth{0.050000\du}
\pgfsetdash{}{0pt}
\pgfsetbuttcap
{
\definecolor{diafillcolor}{rgb}{0.000000, 0.000000, 0.000000}
\pgfsetfillcolor{diafillcolor}
\pgfsetfillopacity{1.000000}
% was here!!!
\definecolor{dialinecolor}{rgb}{0.000000, 0.000000, 0.000000}
\pgfsetstrokecolor{dialinecolor}
\pgfsetstrokeopacity{1.000000}
\draw (22.931185\du,12.349704\du)--(24.749616\du,12.990885\du);
}
\pgfsetlinewidth{0.050000\du}
\pgfsetdash{}{0pt}
\pgfsetbuttcap
{
\definecolor{diafillcolor}{rgb}{0.000000, 0.000000, 0.000000}
\pgfsetfillcolor{diafillcolor}
\pgfsetfillopacity{1.000000}
% was here!!!
\definecolor{dialinecolor}{rgb}{0.000000, 0.000000, 0.000000}
\pgfsetstrokecolor{dialinecolor}
\pgfsetstrokeopacity{1.000000}
\draw (23.120386\du,11.813635\du)--(24.959839\du,12.475838\du);
}
\pgfsetlinewidth{0.050000\du}
\pgfsetdash{}{0pt}
\pgfsetbuttcap
{
\definecolor{diafillcolor}{rgb}{0.000000, 0.000000, 0.000000}
\pgfsetfillcolor{diafillcolor}
\pgfsetfillopacity{1.000000}
% was here!!!
\definecolor{dialinecolor}{rgb}{0.000000, 0.000000, 0.000000}
\pgfsetstrokecolor{dialinecolor}
\pgfsetstrokeopacity{1.000000}
\draw (23.383165\du,11.382677\du)--(25.075462\du,11.981813\du);
}
\pgfsetlinewidth{0.050000\du}
\pgfsetdash{}{0pt}
\pgfsetbuttcap
{
\definecolor{diafillcolor}{rgb}{0.000000, 0.000000, 0.000000}
\pgfsetfillcolor{diafillcolor}
\pgfsetfillopacity{1.000000}
% was here!!!
\definecolor{dialinecolor}{rgb}{0.000000, 0.000000, 0.000000}
\pgfsetstrokecolor{dialinecolor}
\pgfsetstrokeopacity{1.000000}
\draw (23.950768\du,11.004276\du)--(25.117506\du,11.361655\du);
}
\pgfsetlinewidth{0.050000\du}
\pgfsetdash{}{0pt}
\pgfsetbuttcap
{
\definecolor{diafillcolor}{rgb}{0.000000, 0.000000, 0.000000}
\pgfsetfillcolor{diafillcolor}
\pgfsetfillopacity{1.000000}
% was here!!!
\definecolor{dialinecolor}{rgb}{0.000000, 0.000000, 0.000000}
\pgfsetstrokecolor{dialinecolor}
\pgfsetstrokeopacity{1.000000}
\draw (21.774958\du,15.374290\du)--(22.773518\du,16.152116\du);
}
\pgfsetlinewidth{0.050000\du}
\pgfsetdash{}{0pt}
\pgfsetbuttcap
{
\definecolor{diafillcolor}{rgb}{0.000000, 0.000000, 0.000000}
\pgfsetfillcolor{diafillcolor}
\pgfsetfillopacity{1.000000}
% was here!!!
\definecolor{dialinecolor}{rgb}{0.000000, 0.000000, 0.000000}
\pgfsetstrokecolor{dialinecolor}
\pgfsetstrokeopacity{1.000000}
\draw (23.151920\du,15.679114\du)--(21.764447\du,14.627998\du);
}
\pgfsetlinewidth{0.050000\du}
\pgfsetdash{}{0pt}
\pgfsetbuttcap
{
\definecolor{diafillcolor}{rgb}{0.000000, 0.000000, 0.000000}
\pgfsetfillcolor{diafillcolor}
\pgfsetfillopacity{1.000000}
% was here!!!
\definecolor{dialinecolor}{rgb}{0.000000, 0.000000, 0.000000}
\pgfsetstrokecolor{dialinecolor}
\pgfsetstrokeopacity{1.000000}
\draw (23.519810\du,15.143044\du)--(21.953648\du,13.976306\du);
}
\pgfsetlinewidth{0.050000\du}
\pgfsetdash{}{0pt}
\pgfsetbuttcap
{
\definecolor{diafillcolor}{rgb}{0.000000, 0.000000, 0.000000}
\pgfsetfillcolor{diafillcolor}
\pgfsetfillopacity{1.000000}
% was here!!!
\definecolor{dialinecolor}{rgb}{0.000000, 0.000000, 0.000000}
\pgfsetstrokecolor{dialinecolor}
\pgfsetstrokeopacity{1.000000}
\draw (22.247960\du,13.492793\du)--(23.877190\du,14.712087\du);
}
\pgfsetlinewidth{0.050000\du}
\pgfsetdash{}{0pt}
\pgfsetbuttcap
{
\definecolor{diafillcolor}{rgb}{0.000000, 0.000000, 0.000000}
\pgfsetfillcolor{diafillcolor}
\pgfsetfillopacity{1.000000}
% was here!!!
\definecolor{dialinecolor}{rgb}{0.000000, 0.000000, 0.000000}
\pgfsetstrokecolor{dialinecolor}
\pgfsetstrokeopacity{1.000000}
\draw (22.794540\du,13.103880\du)--(24.213547\du,14.123462\du);
}
\pgfsetlinewidth{0.050000\du}
\pgfsetdash{}{0pt}
\pgfsetbuttcap
{
\definecolor{diafillcolor}{rgb}{0.000000, 0.000000, 0.000000}
\pgfsetfillcolor{diafillcolor}
\pgfsetfillopacity{1.000000}
% was here!!!
\definecolor{dialinecolor}{rgb}{0.000000, 0.000000, 0.000000}
\pgfsetstrokecolor{dialinecolor}
\pgfsetstrokeopacity{1.000000}
\draw (23.488277\du,12.904168\du)--(24.570926\du,13.597904\du);
}
% setfont left to latex
\definecolor{dialinecolor}{rgb}{0.000000, 0.000000, 0.000000}
\pgfsetstrokecolor{dialinecolor}
\pgfsetstrokeopacity{1.000000}
\definecolor{diafillcolor}{rgb}{0.000000, 0.000000, 0.000000}
\pgfsetfillcolor{diafillcolor}
\pgfsetfillopacity{1.000000}
\node[anchor=base west,inner sep=0pt,outer sep=0pt,color=dialinecolor] at (27.692128\du,9.327699\du){$B_{1}\left(x^{1}\right)$};
% setfont left to latex
\definecolor{dialinecolor}{rgb}{0.000000, 0.000000, 0.000000}
\pgfsetstrokecolor{dialinecolor}
\pgfsetstrokeopacity{1.000000}
\definecolor{diafillcolor}{rgb}{0.000000, 0.000000, 0.000000}
\pgfsetfillcolor{diafillcolor}
\pgfsetfillopacity{1.000000}
\node[anchor=base west,inner sep=0pt,outer sep=0pt,color=dialinecolor] at (27.854232\du,13.283044\du){$B_{2}\left(x^{2}\right)$};
% setfont left to latex
\definecolor{dialinecolor}{rgb}{0.000000, 0.000000, 0.000000}
\pgfsetstrokecolor{dialinecolor}
\pgfsetstrokeopacity{1.000000}
\definecolor{diafillcolor}{rgb}{0.000000, 0.000000, 0.000000}
\pgfsetfillcolor{diafillcolor}
\pgfsetfillopacity{1.000000}
\node[anchor=base west,inner sep=0pt,outer sep=0pt,color=dialinecolor] at (26.427714\du,16.103660\du){$B_{3}\left(x^{3}\right)$};
\pgfsetlinewidth{0.050000\du}
\pgfsetdash{}{0pt}
\definecolor{diafillcolor}{rgb}{1.000000, 1.000000, 1.000000}
\pgfsetfillcolor{diafillcolor}
\pgfsetfillopacity{1.000000}
\pgfpathellipse{\pgfpoint{23.161941\du}{8.122777\du}}{\pgfpoint{0.050622\du}{0\du}}{\pgfpoint{0\du}{0.046980\du}}
\pgfusepath{fill}
\definecolor{dialinecolor}{rgb}{0.000000, 0.000000, 0.000000}
\pgfsetstrokecolor{dialinecolor}
\pgfsetstrokeopacity{1.000000}
\pgfpathellipse{\pgfpoint{23.161941\du}{8.122777\du}}{\pgfpoint{0.050622\du}{0\du}}{\pgfpoint{0\du}{0.046980\du}}
\pgfusepath{stroke}
\pgfsetlinewidth{0.050000\du}
\pgfsetdash{}{0pt}
\definecolor{diafillcolor}{rgb}{1.000000, 1.000000, 1.000000}
\pgfsetfillcolor{diafillcolor}
\pgfsetfillopacity{1.000000}
\pgfpathellipse{\pgfpoint{22.727193\du}{7.875233\du}}{\pgfpoint{0.050622\du}{0\du}}{\pgfpoint{0\du}{0.046980\du}}
\pgfusepath{fill}
\definecolor{dialinecolor}{rgb}{0.000000, 0.000000, 0.000000}
\pgfsetstrokecolor{dialinecolor}
\pgfsetstrokeopacity{1.000000}
\pgfpathellipse{\pgfpoint{22.727193\du}{7.875233\du}}{\pgfpoint{0.050622\du}{0\du}}{\pgfpoint{0\du}{0.046980\du}}
\pgfusepath{stroke}
\pgfsetlinewidth{0.050000\du}
\pgfsetdash{}{0pt}
\definecolor{diafillcolor}{rgb}{1.000000, 1.000000, 1.000000}
\pgfsetfillcolor{diafillcolor}
\pgfsetfillopacity{1.000000}
\pgfpathellipse{\pgfpoint{22.333012\du}{7.645661\du}}{\pgfpoint{0.050622\du}{0\du}}{\pgfpoint{0\du}{0.046980\du}}
\pgfusepath{fill}
\definecolor{dialinecolor}{rgb}{0.000000, 0.000000, 0.000000}
\pgfsetstrokecolor{dialinecolor}
\pgfsetstrokeopacity{1.000000}
\pgfpathellipse{\pgfpoint{22.333012\du}{7.645661\du}}{\pgfpoint{0.050622\du}{0\du}}{\pgfpoint{0\du}{0.046980\du}}
\pgfusepath{stroke}
\pgfsetlinewidth{0.050000\du}
\pgfsetdash{}{0pt}
\definecolor{diafillcolor}{rgb}{1.000000, 1.000000, 1.000000}
\pgfsetfillcolor{diafillcolor}
\pgfsetfillopacity{1.000000}
\pgfpathellipse{\pgfpoint{21.491259\du}{16.759434\du}}{\pgfpoint{0.050622\du}{0\du}}{\pgfpoint{0\du}{0.046980\du}}
\pgfusepath{fill}
\definecolor{dialinecolor}{rgb}{0.000000, 0.000000, 0.000000}
\pgfsetstrokecolor{dialinecolor}
\pgfsetstrokeopacity{1.000000}
\pgfpathellipse{\pgfpoint{21.491259\du}{16.759434\du}}{\pgfpoint{0.050622\du}{0\du}}{\pgfpoint{0\du}{0.046980\du}}
\pgfusepath{stroke}
\pgfsetlinewidth{0.050000\du}
\pgfsetdash{}{0pt}
\definecolor{diafillcolor}{rgb}{1.000000, 1.000000, 1.000000}
\pgfsetfillcolor{diafillcolor}
\pgfsetfillopacity{1.000000}
\pgfpathellipse{\pgfpoint{21.790824\du}{16.433591\du}}{\pgfpoint{0.050622\du}{0\du}}{\pgfpoint{0\du}{0.046980\du}}
\pgfusepath{fill}
\definecolor{dialinecolor}{rgb}{0.000000, 0.000000, 0.000000}
\pgfsetstrokecolor{dialinecolor}
\pgfsetstrokeopacity{1.000000}
\pgfpathellipse{\pgfpoint{21.790824\du}{16.433591\du}}{\pgfpoint{0.050622\du}{0\du}}{\pgfpoint{0\du}{0.046980\du}}
\pgfusepath{stroke}
\pgfsetlinewidth{0.050000\du}
\pgfsetdash{}{0pt}
\definecolor{diafillcolor}{rgb}{1.000000, 1.000000, 1.000000}
\pgfsetfillcolor{diafillcolor}
\pgfsetfillopacity{1.000000}
\pgfpathellipse{\pgfpoint{21.196950\du}{17.078707\du}}{\pgfpoint{0.050622\du}{0\du}}{\pgfpoint{0\du}{0.046980\du}}
\pgfusepath{fill}
\definecolor{dialinecolor}{rgb}{0.000000, 0.000000, 0.000000}
\pgfsetstrokecolor{dialinecolor}
\pgfsetstrokeopacity{1.000000}
\pgfpathellipse{\pgfpoint{21.196950\du}{17.078707\du}}{\pgfpoint{0.050622\du}{0\du}}{\pgfpoint{0\du}{0.046980\du}}
\pgfusepath{stroke}
\end{tikzpicture}

\caption{Visualisierung zur Zerlegung der Eins}
\end{figure}

\begin{lemma}[Erweiterung von Lipschitz-Mengen]
	Sei $\Omega\subseteq\R^d$ Lipschitz. Dann existiert eine Erweiterung $E\colon H^1(\Omega;\R^d)\to H^1(\R^d;\R^d)$ mit $Ev\vert_\Omega = v$ und
	\begin{align*}
		\abs{\e(Ev)}_{0,\R^d}^2 \leq \frac{c_{K1}}{\sqrt{2}}\left(\abs{\e(v)}_{0,\Omega}^2+\abs{v}_{0,\Omega}^2\right)
	\end{align*}
\end{lemma}
\begin{proof}
	Seien $x^{i}$, $M$, $B_i$, $\tiB_i$, $r_i$, $\tir_i$, $\theta_i$, $A_i$ und $\psi_i$ wie im Lemma zur Partition der Eins.
	Es gibt nach Lemma \ref{le:ErweiterungVonEpigraphen} stetige lineare Erweiterungen
	\begin{align*}
		E_i\colon H^1_{\partial B_i}(B_i\cap\epi(\psi_i);\R^d)\to H^1(B_i;\R^d)
	\end{align*}
	die Gleichung \eqref{eq:ErweiterungEpigraphenUngleichung1} erfüllen.
	Nun setzen wir $\tiOmega\coloneqq\bigcup_i\tiB_i$ und definieren eine Erweiterung $\tilde{E}\colon H^1(\Omega)\to H^1(\tiOmega)$ für $v\in H^1(\Omega;\R^d)$ durch
	\begin{align*}
		\tilde{E}v \coloneqq\sum_iE_i((\theta_iv)\circ A_i)\circ A_i^{-1}
	\end{align*}
	wobei $E_0(\theta_0(v\circ A_0^{-1}))$ die Erweiterung von $\theta_0v$ außerhalb von $\Omega$ mit $0$ bezeichnet. Es gilt für $x\in\Omega$
	\begin{align*}
		(\tilde{E}v)(x) 
		&=\sum_iE_i((\theta_iv)\circ A_i)\circ A_i^{-1}(x) \\
		&= \sum_i(\theta_iv)\circ A_i(A_i^{-1}(x)) \\
		&= \sum_i\theta_i(x)v(x) \\
		&= v(x)
	\end{align*}
	wegen $A_i^{-1}(x)\in B_i\cap\epi(\psi_i)$ falls $x\in B_i$.
	Nun folgt
	\begin{align*}
		&\abs{\e(\tilde{E}v)}_{0,\tilde{\Omega}}^2 \\
		&\leq \sum_i\abs{\e(E_iv)}_{0,B_i}^2 \\
		&\leq c_E\sum_i\abs{\e(\theta_iv)}_{0,B_i\cap\Omega}^2 \\
		&=c_E\sum_{i,j,k}\abs{\e_{jk}(\theta_iv)}_{0,B_i\cap\Omega}^2 \\
		&=c_E\sum_{i,j,k}\abs{\theta_i\e_{jk}(v)+\frac{1}{2}v_j\partial_k\theta_i + \frac{1}{2}v_k\partial_j\theta_i}_{0,B_i\cap\Omega}^2 \\
		&\leq c_E\Big(\sum_{i,j,k}\abs{\theta_i\e_{jk}(v)}_{0,\Omega}^2+\frac{1}{2}\sum_{i,j,k}\abs{v_j}_{0,\Omega}^2\abs{\partial_k\theta_i}_{W^{0,\infty}(\Omega)}^2 \\
		&\quad+\frac{1}{2}\sum_{i,j,k}\abs{v_k}_{0,\Omega}^2\abs{\partial_j\theta_i}_{W^{0,\infty}(\Omega)}^2\Big) \\
		&\leq c_E\left(\sum_i\abs{\theta_i}_{W^{0,\infty}(\Omega)}+(N+1)\max_i\abs{\theta_i}_{W^{1,\infty}(\Omega)}^2\right)\left(\abs{\e(v)}_{0,\Omega}^2 + \abs{v}_{0,\Omega}^2\right) \\
		&=\frac{c_K}{\sqrt{2}}\left(\abs{\e(v)}_{0,\Omega}^2+\abs{v}_{0,\Omega}^2\right)
	\end{align*}
\end{proof}


\subsubsection*{Eine Charakterisierung von Starrkörperbewegungen}

TODO: auf $d\leq 3$ ausweiten

\subsubsection*{Kornsche Ungleichung mit Randbedingungen}


\section{Implementierung}


\subsubsection*{Diskretisiertes Problem}
Das diskrete Problem lautet dann:
Finde $u_h\in\cS_h$, so dass
\begin{align*}
	a(v_h,u_h) &= b(v_h) &&,\text{für alle }v_h\in\cS_h\text{ mit }v_h=0\text{ auf }\Gamma_D \\
	u_h &= w_h &&,\text{auf }\Gamma_D
\end{align*}

%Das diskrete Problem lautet dann:
%Finde $u_h\in\cS_h$, so dass $Mu_h=w_h$ auf $\Gamma_D$ und für alle $v_h\in\cS_h$ mit $Mv_h=0$ auf $\Gamma_D$ gilt
%\begin{align*}
%	a(v_h,u_h) = b(v_h)
%\end{align*}
Wir schreiben $u_h=\sum_i\hu_i\phi_i$ und $v_h=\sum_j\hv_j\phi_j$ und können die rechte Seite umschreiben zu
\begin{align*}
	a(v_h,u_h)
	= a(\sum_{j}\hv_j\phi_j,\sum_{i}\hu_i\phi_i)
	= \sum_{i,j}\underbrace{a(\phi_j,\phi_i)}_{\eqqcolon \hA_{ji}}\hu_i\hv_j
	= \sum_{i,j}\hA_{ji}\hu_i\hv_j
	= \hv^\top \hA\hu
\end{align*}
mit Steifheits-Matrix $\hA_{ij} = a(\phi_j,\phi_i)$.
und die linke Seite lautet 
% (mit leichtem überladen des Symbols $b$)
\begin{align*}
	b(v_h)
	= b(\sum_j \hv_j\phi_j)
	= \sum \underbrace{b(\phi_j)}_{\eqqcolon \hb_j}\hv_j
	= \hv^\top \hb
\end{align*}
mit Load-Vektor $\hb_i=b(\phi_i)$.
%Damit lautet unser neues Problem: Finde $\hu=(\hu_i)_i\in\R^{nd}$, so dass für alle $\hv=(\hv_i)_i\in\R^{nd}$ gilt
%\begin{align*}
%	\hv^\top\hA\hu=\hv^\top\hb
%\end{align*}
%Oder: Finde $\hu\in\R^{nd}$, so dass
%\begin{align*}
%	\hA\hu = \hb
%\end{align*}
Sind $x^{(i_1)},\dots,x^{(i_k)}\in\cK_D$ die Dirichlet-Knoten, so erhalten wir für die Randbedingung
\begin{align*}
	 \hu\big\vert_{\cK_D}
	 \coloneqq \vect{u_{(d-1)i_1+1} \\ \vdots \\ u_{di_k}}
	 =\vect{w_h(x^{(i_1)}) \\ \vdots \\ w_h(x^{(i_k)})}
	 = \vect{u_h(x^{(i_1)}) \\ \vdots \\ u_h(x^{(i_k)})}
	 \eqqcolon \hw\in\R^{kd}
\end{align*}
Damit lautet unser neues Problem: Finde $\hu=(\hu_i)_i\in\R^{nd}$, so dass gilt
\begin{align*}
	\hv^\top\hA\hu&=\hv^\top\hb &&,\text{für alle }\hv=(\hv_i)_i\in\R^{nd}\text{ mit }\hv \big\vert_{\cK_D}=0 \\
	\hu\big\vert_{\cK_D} &= \hw
\end{align*}
Es genügt, ein $\hu\in\R^{nd}$ zu finden (warum existiert so etwas? dies ist nicht klar), so dass
\begin{align*}
	\hA\hu &= \hb \\
	\hu\big\vert_{\cK_D} &= \hw
\end{align*}
Weil wir das Implementiert haben betrachten wir das allgemeinere System
\begin{align*}
	\hA\hu &= \hb \\
	\hu\big\vert_{\cK_D} &= \hw
\end{align*}
Wir lösen dies mit Lagrange-Multiplikatoren als System
\begin{align*}
	\begin{bmatrix}
		\hA & B^\top \\
		B & 0
	\end{bmatrix}
	\vect{\hu \\ \lambda}
	= \vect{\hb \\ \hw}
\end{align*}

 Wir definieren die Lagrangefunktion $L\colon\R^{nd}\times\R^{ld}\to\R$
\begin{align*}
	L(\hv,\lambda)\coloneqq W(\hv)+\lambda^\top(B\hv-\hw)
	= \hv^\top\hA\hv-\hb^\top\hv+\lambda^\top(B\hv-\hw)
\end{align*}
Wir rechnen
\begin{align*}
	\nabla L(\hv,\lambda)
	&= \begin{bmatrix}
		\nabla_{\hv}L(\hv,\lambda) \\
		\nabla_{\lambda}L(\hv,\lambda)
	\end{bmatrix} \\
	&= \begin{bmatrix}
		\hA\hv-\hb+B^\top\lambda \\
		B\hv-\hw
	\end{bmatrix} \\
	&= \begin{bmatrix}
		\hA\hv+B^\top\lambda \\
		B\hv
	\end{bmatrix}
	-\vect{\hb \\ \hw} \\
	&= \begin{bmatrix}
		\hA & B^\top \\
		B & 0
	\end{bmatrix}
	\vect{\hv \\ \lambda}
	-\vect{\hb \\ \hw}
\end{align*}
Nun ist $\nabla L=0$ ist ein notwendiges Kriterium (Satz 2.36 Geiger, Kanzow) dafür, dass ein Minimum des Optimierungsproblems vorliegt. Also besitzt das System
\begin{align*}
	\begin{bmatrix}
		\hA & B^\top \\
		B & 0
	\end{bmatrix}
	\vect{\hv \\ \lambda}
	=\vect{\hb \\ \hw}
\end{align*}
eine Lösung. Weiterhin ist dies aufgrund der Konvexität ein hinreichendes Optimalitätskriterium (Satz 2.46, Geiger, Kanzow) und somit ist die Lösung des Systems eindeutig. Kürzer: Korollar 2.47 Geiger-Kanzow.

Das diskrete Problem lautet dann:
Finde $u_h\in\cS_h$, so dass $Mu_h=w_h$ auf $\Gamma_D$ und für alle $v_h\in\cS_h$ mit $Mv_h=0$ auf $\Gamma_D$ gilt
\begin{align*}
	a(v_h,u_h) = b(v_h)
\end{align*}

\subsubsection*{Implementierung des Hooke-Tensors}


(in Braess:)
\begin{align*}
	\vect{\sigma_{11} \\ \sigma_{22} \\ \sigma_{33} \\ \sigma_{12} \\ \sigma_{13} \\ \sigma_{23}}
	= \frac{E}{(1+\nu)(1-2\nu)}\begin{bmatrix}
		1-\nu & \nu & \nu & & & \\
		\nu & 1-\nu & \nu & & & \\
		\nu & \nu & 1-\nu & & & \\
		& & & 1-2\nu & & \\
		& & & & 1-2\nu & \\
		& & & & & 1-2\nu \\
	\end{bmatrix}
	\vect{\e_{11} \\ \e_{22} \\ \e_{33} \\ \e_{12} \\ \e_{13} \\ \e_{23}}
\end{align*}

\subsubsection*{Berechnung der Steifheitsmatrix}
TODO: Vereinheitlichung der Bezeichnungen $\varphi_i$ und $\varphi_{x^{(i)}}$

\subsubsection*{Berechnung des Load-Vektors}


\subsubsection*{Randbedingungen}



\section{A posteriori Fehlerschätzer}
\subsection*{Residuale Schätzer}

Wir folgen im folgenden im wesentlichen \cite{Bra-2007, Nei-2004}

Im folgenden betrachten wir allgemeine lineare elliptische Probleme in einer Dimension
Wir haben den Operator $\cA$ einer gleichmäßig elliptischen Differenzialgleichung gegeben durch
\begin{align*}
	\cA v = \sum_{i,j}\partial_i(A_{ij}\partial_jv) = \diver A\nabla v
\end{align*}
mit $A_{ij}=A_{ji}$. So dass
\begin{align*}
	\lambda_{min}\abs{\eta}^2\leq A_{ij}\eta_i\eta_j\leq \lambda_{max}\abs{\eta}^2
\end{align*}
Wir definieren das Residuum
\begin{align*}
	R(v)\coloneqq f+\diver(A\nabla v)
\end{align*}


	für $E\subseteq\Gamma_D$ ist
	\begin{align*}
		\int_{E}((\sigma(e)\cdot n_T)\cdot z\dif s
		&= \int_{E}((\sigma(\underbrace{u-u_h}_{=w-w=0})\cdot z)\cdot z\dif s \\
		&= \int_{E}0\cdot z)\cdot z\dif s \\
		&= \int_{E}(R_E\cdot z)\dif s \\
		&= \inner{R_E,z}_{0,E}
	\end{align*}
	für $E\subseteq\Gamma_N$ ist
	\begin{align*}
		\inner{\sigma(e)\cdot n_T,z}_{0,E}
		&= \inner{\sigma(u-u_h)\cdot n,z}_{0,E}\\
		&= \inner{g-\sigma(u_h)\cdot n,z}_{0,E} \\
		&= \inner{R_E,z}_{0,E}
	\end{align*} $E$ kommt als Integrationsgebiet genau zweimal in der Summe vor und es gilt
	\begin{align*}
		\inner{\sigma(e)\vert_{T_1}\cdot n_1,z}_{0,E}+\inner{\sigma(e)\vert_{T_2}\cdot n_2}_{0,E}
		&= 
	\end{align*}
	
	
	Für den zweiten Term gilt
	\begin{align*}
		-\sum_T\int_T(\diver \sigma(e))\cdot z\dif x
		&= -\sum_T\inner{\diver \sigma(e),z}_{0,T} \\
		&= -\sum_T\inner{\diver \sigma(u-u_h),z}_{0,T} \\
		&= -\sum_T\inner{\diver \sigma(u),z}_{0,T} +\sum_T\inner{\diver \sigma(u_h),z}_{0,T} \\
		&= \sum_T\inner{f,z}_{0,T} +\sum_T\inner{\diver \sigma(u_h),z}_{0,T} \\
		&= \sum_T\inner{f+\diver \sigma(u_h),z}_{0,T} \\
		&= \sum_T\inner{R_T,z}_{0,T}
	\end{align*}

\subsubsection*{Obere Abschätzung des Fehlers}

\subsubsection*{Obere Abschätzung des Schätzers}

 die Umgebung $\omega_T$ vom Dreieck $T$ durch
\begin{align*}
	\omega_T\coloneqq\bigcup\{T'\in\cT\vert T\text{ und }T'\text{ haben mindestens eine Kante gemeinsam }\}
\end{align*}

\begin{theorem}
	(nach \cite{Bra-2007}
	Sei $\cT$ eine quasiuniforme Triangluierung mit Regularitätsparameter $\kappa$. Dann gibt es ein von $\Omega$ und $\kappa$ abhängiges $c>0$, so dass für alle $T\in\cT$ gilt die obere Abschätzung
	\begin{align*}
		\norm{u-u_h}_{1,\Omega}\leq c\left(\sum_{T\in\cT}\eta_{T,R}^2\right)^{1/2}
	\end{align*}
	und die untere Abschätzung
	\begin{align*}
		\eta_{T,R}\leq c\left(\norm_{u-u_h}_{1,\omega_T}^2+\sum_{T'\subseteq\omega_T}h_T^2\norm{f-P_hf}_{0,T}^2\right)^{1/2}
	\end{align*}
\end{theorem}
\begin{proof}
	(Zur oberen Abschätzung)
	Es gilt für $v\in H_0^1(\Omega)$
	\begin{align*}
		\inner{\grad (u-u_h),\grad v}_{0,\Omega}
		&= 
	\end{align*}
\end{proof}


\subsection*{lokales Neumann-Problem}
TODO: ist folgendes legitim (aus Braess): Man definiert $\sigma_h$ an den Knotenpunkten als gewichtetes Mittel von $\partial u_h$ an den angrenzenden Dreiecken und dann durch lineare Interpolation auf den Dreiecken. Dann liefert $\abs{\sigma_h-\partial u_h}$ einen Fehlerschätzer (die Wahl von $\sigma$ als Symbol) ist ein wenig unglücklich
\subsection*{Hierarchische Schätzer}

\newpage

\begin{thebibliography}{00}

\bibitem{Alb-2002}
\newblock Alberty, J., C. Carstensen, S. A. Funken, and R. Klose.
\newblock "Matlab Implementation of the Finite Element Method in Elasticity."
\newblock {\em Computing 69}, no. 3 (2002): 239-263. 

\bibitem{Ban-2003}
\newblock Bangerth, Wolfgang, and Rolf Rannacher.
\newblock {\em Adaptive Finite Element Methods for Differential Equations.}
\newblock Basel [u.a.]: Birkhäuser, 2003. S.130f.

\bibitem{Bra-2007}
\newblock Braess, Dietrich.
\newblock {\em Finite Elemente: Theorie, Schnelle Löser Und Anwendungen in Der Elastizitätstheorie.}
\newblock 4., überarb. und erw. Aufl.
\newblock Berlin [u.a.]: Springer, 2007.

\bibitem{Cia-1988}
\newblock Ciarlet, Philippe G.
\newblock {\em Studies in Mathematics and Its Applications. Mathematical Elasticity. 1, Three-dimensional Elasticity.}
\newblock Amsterdam [u.a.]: North-Holland, 1988.

\bibitem{Cia-1997}
\newblock Ciarlet, Philippe G.
\newblock {\em Studies in Mathematics and Its Applications. Mathematical Elasticity. 2, Theory of Plates.}
\newblock Amsterdam [u.a.]: North-Holland, 1997. 

\bibitem{Duv-1976}
\newblock Lions, Jacques Louis, and Georges Duvaut.
\newblock {\em Inequalities in Mechanics and Physics.}
\newblock Berlin, Heidelberg: Springer, 1976. 

\bibitem{Kik-1988}
\newblock Kikuchi, Noboru, and John Tinsley Oden.
\newblock {\em Contact Problems in Elasticity: A Study of Variational Inequalities and Finite Element Methods.}
\newblock Philadelphia: SIAM, 1988. 

\bibitem{Lif-1959}
\newblock Lifshitz, Evgenii Mikhailovich, and Lev Davidovich Landau.
\newblock {\em Course of Theoretical Physics.}
\newblock Pergamon, 1959.

\bibitem{Nei-2004}
\newblock Neittaanmäki, Pekka, and Sergey R. Repin.
\newblock {\em Reliable Methods for Computer Simulation: Error Control and Posteriori Estimates}.
\newblock Oxford: Elsevier Science \& Technology, 2004.


\end{thebibliography}


\end{document}
